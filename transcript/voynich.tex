% BibTeX
% XeLaTeX

\documentclass{scrarticle}

\usepackage[english]{babel}
\usepackage[natbib,authordate,backend=bibtex]{biblatex-chicago}
   \bibliography{references}
\usepackage[makeroom]{cancel}
\usepackage[a4paper,top=2cm,bottom=2cm,left=3cm,right=3cm,marginparwidth=1.75cm]{geometry}
\usepackage{fontspec}
   \newfontfamily\eva{fairfax_eva_hd.ttf}
\usepackage{tabto}
\usepackage{xcolor}

\newenvironment{locator}{\tiny\sffamily}

\deffootnote{1.5em}{1em}{\makebox[1.5em][l]{\thefootnotemark}}
   \setlength{\skip\footins}{1.5em}
   \setlength{\footnotesep}{1em}

\title{Beinecke MS 408}
\subtitle{A Transcription Based on a Revised Version\\of Takeshi Takahashi's Transliteration}
\author{Alexander Max Bauer}
\date{}

\begin{document}
\maketitle


%%%%%%%%%%%%%%%%
% INTRODUCTION %
%%%%%%%%%%%%%%%%
\section{Introduction}\label{sec:introduction}
In Section \ref{sec:transcription} of this document, mainly to promote easier readability and comparability, a transcription of \textit{Beinecke MS 408} is currently being developed, based on a transliteration by \citet{takahashi_voynich_2004}.
To achieve this, Takahashi's transliteration is obtained in an automatically capitalized version of the \textit{European} or \textit{Extensible Voynich Alphabet} (EVA) using the \textit{Interlinear Transcription Archive Extractor} by \citet{schwerdtfeger_voynich_2004}.\footnote{\citet{schwerdtfeger_voynich_2004} builds on an online archive of Voynich transcriptions maintained by Jorge \citet{stolfi_voynich_1998}. For a general overview of transcription efforts, see \citet{zandbergen_text_2023}.}
It is then converted back into glyphs using a font by \citet{bettencourt_voynich_2019}.

Using the manuscript's reproduction in \citet{clemens_voynich_2016} and the scans provided by Yale University's \citet{beinecke_voynich_2004}, Takahashi's transliteration is compared with what I am able to make out of the manuscript.
Changes to Takahashi's reading are highlighted in \textcolor{orange}{orange}:

\begin{itemize}
   \item Glyphs that I interpret in another way than Takahashi or that I add to the corpus are printed in \textcolor{orange}{orange}, showing my interpretation.
   \item When a glyph has been deleted by me without replacement, this is indicated by \textcolor{orange}{$\cancel{x}$}.
   \item Removed spaces are marked with \textcolor{orange}{$\frown$} and added spaces are indicated by \textcolor{orange}{$\smile$}.
   \item Added paragraphs are represented by \textcolor{orange}{{\P}}.
\end{itemize}

For the sake of simplicity, places where the writing is interrupted by illustrations are also marked by spaces.
Especially the recognition of spaces in running text is, as Takahahashi has already noted, sometimes very difficult; therefore, my adjustments in this -- just as in any other -- regard should be taken with a certain amount of caution.
In total, I made changes in $299$ places of Takahashi's transliteration.

Each transcription line is preceded by a locator in square brackets, informing the reader about the element a line is part of (i.e., paragraph, title, or label) and assigning the line a number.
For example, a line located at [P2, 9] is part of the page's second paragraph (P2) and is counted as the overall ninth line (9) on this page.

Additionally, in Section \ref{sec:analysis}, some basic analyses are presented.
First, in Section \ref{sec:frequency}, glyph frequencies are counted.
In Section \ref{sec:entropy}, then, entropy is calculated.


%%%%%%%%%%%%%%%%%
% TRANSCRIPTION %
%%%%%%%%%%%%%%%%%
\clearpage
\section{Transcription}\label{sec:transcription}


%%%%%%
% 1R %
%%%%%%
\subsection*{Folio 1, Recto}
\begin{locator}[P1, 1]\end{locator} {\eva fachys ykal ar ataiin Shol Shory cThres y kor Sholdy}\\
\begin{locator}[P1, 2]\end{locator} {\eva sory cKhar or y kair chtaiin Shar are cThar cThar dan}\\
\begin{locator}[P1, 3]\end{locator} {\eva syaiir Sheky or ykaiin Shod cThoary cThes daraiin s\textcolor{orange}{y}}\\
\begin{locator}[P1, 4]\end{locator} {\eva \textcolor{orange}{d}oiin oteey oteos roloty cTh\textcolor{orange}{i}ar daiin o\textcolor{orange}{k}aiin or okan}\\
\begin{locator}[P1, 5]\end{locator} {\eva dair\textcolor{orange}{$\frown$}y chear cThaiin cPhar cFhaiin}\\
\begin{locator}[T1, 6]\end{locator} {\eva ydaraiShy}\\

\vspace{1em}
\noindent\begin{locator}[P2, 7]\end{locator} {\eva \textcolor{orange}{ü} odar \textcolor{orange}{S$\frown$}y Shol cPhoy oydar Sh*\textcolor{orange}{$\frown$}s cFhoaiin Shodary}\\
\begin{locator}[P2, 8]\end{locator} {\eva yShey Shody okchoy otchol chocThy oschy dain chor kos}\\
\begin{locator}[P2, 9]\end{locator} {\eva daiin Shos cFhol Shody}\\
\begin{locator}[T2, 10]\end{locator} {\eva dain os teody}\\

\vspace{1em}
\noindent\begin{locator}[P3, 11]\end{locator} {\eva \textcolor{orange}{ü} ydain cPhesaiin ol\textcolor{orange}{$\frown$}s cPhey ytain ShoShy cPhodales}\\
\begin{locator}[P3, 12]\end{locator} {\eva okSho kShoy otairin oteol okan Shodain scKhey daiin}\\
\begin{locator}[P3, 13]\end{locator} {\eva Shoy cKhey kodaiin cPhy cPhodaiils cThey She oldain d}\\
\begin{locator}[P3, 14]\end{locator} {\eva dain oiin chol odaiin chodain chdy okain dan cThy kod}\\
\begin{locator}[P3, 15]\end{locator} {\eva daiin ShcKhey \textcolor{orange}{cKh}or chor Shey kol chol chol kor chal}\\
\begin{locator}[P3, 16]\end{locator} {\eva Sho chol Shodan kShy kchy dor chodaiin Sho kchom}\\
\begin{locator}[P3, 17]\end{locator} {\eva ycho tchey ch\textcolor{orange}{e}kain Sheo pShol dydyd cThy daicThy}\\
\begin{locator}[P3, 18]\end{locator} {\eva yto Shol She kodShey cPhealy dasain dain cKhyds}\\
\begin{locator}[P3, 19]\end{locator} {\eva dchar ShcThaiin okaiir\textcolor{orange}{$\frown$}chey rchy potol cThols d\textcolor{orange}{a}octa}\\
\begin{locator}[P3, 20]\end{locator} {\eva Shok chor chey dain cKhey}\\
\begin{locator}[T3, 21]\end{locator} {\eva otol daiiin}\\

\vspace{1em}
\noindent\begin{locator}[P4, 22]\end{locator} {\eva cPho Shaiin Shokcheey chol tShodeesy Shey pydeey chy ro d\textcolor{orange}{ar}}\\
\begin{locator}[P4, 23]\end{locator} {\eva *doin chol dain cThal dar Shear kaiin dar Shey cThar}\\
\begin{locator}[P4, 24]\end{locator} {\eva cho*o kaiin Shoaiin okol daiin far cThol daiin cTholdar}\\
\begin{locator}[P4, 25]\end{locator} {\eva ycheey okay oky daiin okchey kokaiin \textcolor{orange}{of}chol k\textcolor{orange}{ad}chy dal}\\
\begin{locator}[P4, 26]\end{locator} {\eva d\textcolor{orange}{che}o Shody koShey cThy okchey keey keey dal chtor}\\
\begin{locator}[P4, 27]\end{locator} {\eva \textcolor{orange}{ch}o chol chok choty chotey}\\
\begin{locator}[T4, 28]\end{locator} {\eva dchaiin}\\


%%%%%%
% 1V %
%%%%%%
\clearpage
\subsection*{Folio 1, Verso}
\begin{locator}[P1, 1]\end{locator} {\eva kchsy chadaiin ol oltchey char cFhar am}\\
\begin{locator}[P1, 2]\end{locator} {\eva ytee\textcolor{orange}{y} char or ochy dcho lkody okodar chody}\\
\begin{locator}[P1, 3]\end{locator} {\eva do cKhy cKho cKhy Shy dkSheey cThy kotchody dal}\\
\begin{locator}[P1, 4]\end{locator} {\eva dol chokeo dair dam sochey chokody}\\

\vspace{1em}
\noindent\begin{locator}[P2, 5]\end{locator} {\eva potoy Shol dair cPhoal dar chey tody otoaiin ShoShy}\\
\begin{locator}[P2, 6]\end{locator} {\eva choky chol cThol Shol okal dolchey chodo lol chy cThy}\\
\begin{locator}[P2, 7]\end{locator} {\eva qo ol choee\textcolor{orange}{s} cheol dol cThey ykol dol dolo ykol do lchody}\\
\begin{locator}[P2, 8]\end{locator} {\eva okol\textcolor{orange}{$\frown$}Shol kol kechy chol ky chol cThol chody chol daiin}\\
\begin{locator}[P2, 9]\end{locator} {\eva Shor okol chol dol ky dar Shol dchor otcho dar Shody}\\
\begin{locator}[P2, 10]\end{locator} {\eva taor chotchey dal chody schody pol chodar}\\


%%%%%%
% 2R %
%%%%%%
\clearpage
\subsection*{Folio 2, Recto}
\begin{locator}[P1, 1]\end{locator} {\eva kydainy ypchol daiin otchal ypchaiin cKholsy}\\
\begin{locator}[P1, 2]\end{locator} {\eva dorchory chkar s Shor cThy cTh}\\
\begin{locator}[P1, 3]\end{locator} {\eva qotaiin cThey y chor chy ydy chaiin}\\
\begin{locator}[P1, 4]\end{locator} {\eva chaindy chtod dy cPhy dals chokaiin d}\\
\begin{locator}[P1, 5]\end{locator} {\eva otochor al Shodaiin chol dan ytchaiin dan}\\
\begin{locator}[P1, 6]\end{locator} {\eva saiin daind dkol sor ytoldy dchol dchy cThy}\\
\begin{locator}[P1, 7]\end{locator} {\eva Shor cKhy daiiny chol dan}\\

\vspace{1em}
\noindent\begin{locator}[L1, 8]\end{locator} {\eva \textcolor{orange}{ytoaile*}}\\
\begin{locator}[L2, 9]\end{locator} {\eva \textcolor{orange}{***an}}\\

\vspace{1em}
\noindent\begin{locator}[P2, 10]\end{locator} {\eva kydain Shaiin qoy s Shol fodan ykSh olSheey daiildy}\\
\begin{locator}[P2, 11]\end{locator} {\eva dlsSho kol Sheey qokey ykody so chol yky dain daiirol}\\
\begin{locator}[P2, 12]\end{locator} {\eva qoky cholaiin Shol Sheky daiin cThey keol saiin saiin}\\
\begin{locator}[P2, 13]\end{locator} {\eva ychain dal chy dalor Shan dan olsaiin Sheey cKhor}\\
\begin{locator}[P2, 14]\end{locator} {\eva okol chy chor cThor yor an chan saiin chety chyky sal}\\
\begin{locator}[P2, 15]\end{locator} {\eva Sho ykeey chey daiin chcThy}\\


%%%%%%
% 2V %
%%%%%%
\clearpage
\subsection*{Folio 2, Verso}
\begin{locator}[P1, 1]\end{locator} {\eva kooiin cheo pchor otaiin o dain chor dair Shty}\\
\begin{locator}[P1, 2]\end{locator} {\eva kcho kchy Sho Shol qotcho loeees qoty chor daiin}\\
\begin{locator}[P1, 3]\end{locator} {\eva otchy chor lShy chol chody chodain chcThy daiin}\\
\begin{locator}[P1, 4]\end{locator} {\eva Sho cholo cheor chodaiin}\\

\vspace{1em}
\noindent\begin{locator}[P2, 5]\end{locator} {\eva kchor Shy daiiin chcKhoy s Shey dor chol daiin}\\
\begin{locator}[P2, 6]\end{locator} {\eva dor chol chor chol keol chy chty daiin otchor chan}\\
\begin{locator}[P2, 7]\end{locator} {\eva daiin chotchey qoteeey chokeos chees chr cheaiin}\\
\begin{locator}[P2, 8]\end{locator} {\eva chokoiShe chor cheol chol dolody}\\


%%%%%%
% 3R %
%%%%%%
\clearpage
\subsection*{Folio 3, Recto}
\begin{locator}[P1, 1]\end{locator} {\eva tSheos qopal chol cThol daimm}\\
\begin{locator}[P1, 2]\end{locator} {\eva ycheor chor dam qotcham cham}\\
\begin{locator}[P1, 3]\end{locator} {\eva ochor qocheor chol daiin cThy}\\
\begin{locator}[P1, 4]\end{locator} {\eva schey chor chal cham cham cho}\\
\begin{locator}[P1, 5]\end{locator} {\eva qokol chololy s cham cThol}\\
\begin{locator}[P1, 6]\end{locator} {\eva ychtaiin chor cThom otal\textcolor{orange}{$\frown$}dam}\\
\begin{locator}[P1, 7]\end{locator} {\eva otchol qodaiin chom Shom damo}\\
\begin{locator}[P1, 8]\end{locator} {\eva ySheor chor chol oky damo}\\
\begin{locator}[P1, 9]\end{locator} {\eva Sho *or Sheoldam otchody ol}\\
\begin{locator}[P1, 10]\end{locator} {\eva ydas chol cThom}\\

\vspace{1em}
\noindent\begin{locator}[P2, 11]\end{locator} {\eva pcheol Shol sols Sheol Shey}\\
\begin{locator}[P2, 12]\end{locator} {\eva okadaiin qokchor qoschodam ocThy}\\
\begin{locator}[P2, 13]\end{locator} {\eva qokeey qot Shey qokody qok\textcolor{orange}{Sh}ey cheody}\\
\begin{locator}[P2, 14]\end{locator} {\eva chor qodair okeey qokeey}\\

\vspace{1em}
\noindent\begin{locator}[P3, 15]\end{locator} {\eva tSheoarom Shor or chor olchsy chom otchom oporar}\\
\begin{locator}[P3, 16]\end{locator} {\eva oteol chol s cheol ekShy qokeom qokol daiin soleeg}\\
\begin{locator}[P3, 17]\end{locator} {\eva soeom okeom yteody qokeeo\textcolor{orange}{$\frown$}dal sam}\\

\vspace{1em}
\noindent\begin{locator}[P4, 18]\end{locator} {\eva pcheoldom Shodaiin qopchor qopol opchol qoty otolom}\\
\begin{locator}[P4, 19]\end{locator} {\eva otchor ol cheor qoeor dair qoteol qosaiin chor cThy}\\
\begin{locator}[P4, 20]\end{locator} {\eva ycheor chol odaiin chol s aiin okolor am}\\


%%%%%%
% 3V %
%%%%%%
\clearpage
\subsection*{Folio 3, Verso}
\begin{locator}[P1, 1]\end{locator} {\eva koaiin cPhor qotoy Sha cKhol ykoaiin s oly}\\
\begin{locator}[P1, 2]\end{locator} {\eva daiidy qot\textcolor{orange}{ch}ol okchor okor olytol dol dar}\\
\begin{locator}[P1, 3]\end{locator} {\eva okom chol Shol seees chom cheeykam okai}\\
\begin{locator}[P1, 4]\end{locator} {\eva qodar \textcolor{orange}{ch}s \textcolor{orange}{ch}y kcheol okal do\textcolor{orange}{$\frown$}r chear een}\\
\begin{locator}[P1, 5]\end{locator} {\eva y\textcolor{orange}{ch}ear otchal \textcolor{orange}{ch}or \textcolor{orange}{ch}ar cKhy}\\
\begin{locator}[P1, 6]\end{locator} {\eva or cheor kor chodaly chom}\\

\vspace{1em}
\noindent\begin{locator}[P2, 7]\end{locator} {\eva tchor otcham chor cFham s}\\
\begin{locator}[P2, 8]\end{locator} {\eva ykchy kchom chor ch\textcolor{orange}{cKh}ol oka}\\
\begin{locator}[P2, 9]\end{locator} {\eva ytcheear okeol cThodoaly chor cThy}\\
\begin{locator}[P2, 10]\end{locator} {\eva ochor daiin qokShol daiim chol okary}\\
\begin{locator}[P2, 11]\end{locator} {\eva Sho ShocKho cKhy tchor chodaiin chom}\\
\begin{locator}[P2, 12]\end{locator} {\eva oSh chodair ytchy tchor kcham s}\\
\begin{locator}[P2, 13]\end{locator} {\eva Shar Shkaiin qokchy yty cThal chky}\\
\begin{locator}[P2, 14]\end{locator} {\eva dain Sheam yteam}\\


%%%%%%
% 4R %
%%%%%%
\clearpage
\subsection*{Folio 4, Recto}
\begin{locator}[P1, 1]\end{locator} {\eva kodalchy chpady Sheol ol Sheey qotey doiin chor ytoy}\\
\begin{locator}[P1, 2]\end{locator} {\eva dchor chol Shol cThol Shtchy chaiin \textcolor{orange}{*}s choraiin chom}\\
\begin{locator}[P1, 3]\end{locator} {\eva otchol chol chy chaiin qotaiin daiin Shain}\\
\begin{locator}[P1, 4]\end{locator} {\eva qotchol chy yty daiin okaiin cThy}\\

\vspace{1em}
\noindent\begin{locator}[P2, 5]\end{locator} {\eva pydaiin qotchy dy tydy}\\
\begin{locator}[P2, 6]\end{locator} {\eva chor Shytchy dy tche\textcolor{orange}{e}y}\\
\begin{locator}[P2, 7]\end{locator} {\eva qotaiin cThol daiin cThom}\\
\begin{locator}[P2, 8]\end{locator} {\eva Shor Shol Shol cThy cPholdy}\\
\begin{locator}[P2, 9]\end{locator} {\eva daiin cKhochy tchy koraiin}\\
\begin{locator}[P2, 10]\end{locator} {\eva odal Shor ShyShol cPhaiin}\\
\begin{locator}[P2, 11]\end{locator} {\eva qotchoiin Sheyr qoty}\\
\begin{locator}[P2, 12]\end{locator} {\eva soiin chaiin chaiin}\\
\begin{locator}[P2, 13]\end{locator} {\eva daiin cThey}\\


%%%%%%
% 4V %
%%%%%%
\clearpage
\subsection*{Folio 4, Verso}
\begin{locator}[P1, 1]\end{locator} {\eva pchooiin kSheo kchoy chopchy dolds dlod}\\
\begin{locator}[P1, 2]\end{locator} {\eva ol chey chy cThy Shkchor Sheo cheory choldy}\\
\begin{locator}[P1, 3]\end{locator} {\eva Sho Sho chaiin Shaiin daiin qodaiin o ar am}\\
\begin{locator}[P1, 4]\end{locator} {\eva qokShy qocThy choteol daiin cThey choaiin}\\
\begin{locator}[P1, 5]\end{locator} {\eva Shor Sheey cto otoiin Shey qotchoiin chodain}\\
\begin{locator}[P1, 6]\end{locator} {\eva ytchoy Shokchy cPhody}\\

\vspace{1em}
\noindent\begin{locator}[P2, 7]\end{locator} {\eva torchy Sheeor chor Shokchy cPhydy}\\
\begin{locator}[P2, 8]\end{locator} {\eva olaen chor cThol Sho otor cThory}\\
\begin{locator}[P2, 9]\end{locator} {\eva qooko iiincheom chcThy Shoky daiin}\\
\begin{locator}[P2, 10]\end{locator} {\eva otaiin Sheo okeody chol chokeody}\\
\begin{locator}[P2, 11]\end{locator} {\eva Sho kcheor Shody Shtaiin qotol daiin}\\
\begin{locator}[P2, 12]\end{locator} {\eva qokoy Sho okeol s keey Shar char ody}\\
\begin{locator}[P2, 13]\end{locator} {\eva Shody s cheor chokody Shodaiin qoty}\\
\begin{locator}[P2, 14]\end{locator} {\eva ochody chykey chtody}\\


%%%%%%
% 5R %
%%%%%%
\clearpage
\subsection*{Folio 5, Recto}
\begin{locator}[P1, 1]\end{locator} {\eva kShody fchoy chkoy oaiin oar olsy chody dkShy dy}\\
\begin{locator}[P1, 2]\end{locator} {\eva ochey okey qokaiin Sho cKhoy cThey chey oka*or otol}\\
\begin{locator}[P1, 3]\end{locator} {\eva qoaiin otan chy daiin oteeen cho cThy otchy qotcho dy}\\
\begin{locator}[P1, 4]\end{locator} {\eva otain Sheody chan s cheor chocThy}\\

\vspace{1em}
\noindent\begin{locator}[P2, 5]\end{locator} {\eva tShy Shody qoaiin cholols Sho qotcheo daiin Shodaiin}\\
\begin{locator}[P2, 6]\end{locator} {\eva Sho cheor chey qoeeey qoykeeey qoeor cThy ShotShy dy}\\
\begin{locator}[P2, 7]\end{locator} {\eva qotoeey keey cheo kchy Shody}\\


%%%%%%
% 5V %
%%%%%%
\clearpage
\subsection*{Folio 5, Verso}
\begin{locator}[P1, 1]\end{locator} {\eva kocheor chor ytchey pShod chols chodaiin ytoiiin daiin}\\
\begin{locator}[P1, 2]\end{locator} {\eva dchol \textcolor{orange}{S}y chol otaiin dain cThor chots ychopordg}\\
\begin{locator}[P1, 3]\end{locator} {\eva qotcho ytor daiin daiin otchor daii\textcolor{orange}{n} qo darchor do}\\
\begin{locator}[P1, 4]\end{locator} {\eva qotor Shees otol ykoiin Shol daiin cThor okchy taiin}\\
\begin{locator}[P1, 5]\end{locator} {\eva Shokeeol chor cheotol otchol daiin dal chol chotaiin}\\
\begin{locator}[P1, 6]\end{locator} {\eva otol chol dairodg}\\


%%%%%%
% 6R %
%%%%%%
\clearpage
\subsection*{Folio 6, Recto}
\begin{locator}[P1, 1]\end{locator} {\eva foar y Shol cholor cPhol chor chcKh chopchol otcham}\\
\begin{locator}[P1, 2]\end{locator} {\eva daiin chcKhy chor chor kar cThy cThor chotols}\\
\begin{locator}[P1, 3]\end{locator} {\eva poeear kShor choky os cheoee\textcolor{orange}{e}s ykeor ytaiin dar}\\
\begin{locator}[P1, 4]\end{locator} {\eva dar cho s Sheor chocThy otcham yaiir chy}\\
\begin{locator}[P1, 5]\end{locator} {\eva tar okoiin Shees ytaly cThaiin odam}\\
\begin{locator}[P1, 6]\end{locator} {\eva or al daiin cKham okom cThaiin ydaiin}\\
\begin{locator}[P1, 7]\end{locator} {\eva daiin qodaiin cho s chol okaiin s}\\
\begin{locator}[P1, 8]\end{locator} {\eva ychol cKhor pchar Sheo cKhaiin}\\
\begin{locator}[P1, 9]\end{locator} {\eva dar Sh\textcolor{orange}{e}ol skaiiodar otaiin chory}\\
\begin{locator}[P1, 10]\end{locator} {\eva tchor cTheod chy Shor odShe od}\\
\begin{locator}[P1, 11]\end{locator} {\eva ychar olchad ol chokaiin}\\
\begin{locator}[P1, 12]\end{locator} {\eva or Shol cThom chor cThy}\\
\begin{locator}[P1, 13]\end{locator} {\eva qocThol \textcolor{orange}{y}odaiin cThy}\\
\begin{locator}[P1, 14]\end{locator} {\eva ySho taiin y kaiim}\\


%%%%%%
% 6V %
%%%%%%
\clearpage
\subsection*{Folio 6, Verso}
\begin{locator}[P1, 1]\end{locator} {\eva koar y sar \textcolor{orange}{c}heekar qoar Shor chapchy s chear char otchy}\\
\begin{locator}[P1, 2]\end{locator} {\eva oees chor chcKhy qoekchar cheas odaii\textcolor{orange}{i}n kchey chor chaiin}\\
\begin{locator}[P1, 3]\end{locator} {\eva qoair cKhy chol oochocKhy chekchoy cKhy okol rychos}\\
\begin{locator}[P1, 4]\end{locator} {\eva y ShcKhy ytchoy sos y\textcolor{orange}{$\frown$}dady dchy dey okody ytody}\\
\begin{locator}[P1, 5]\end{locator} {\eva dair \textcolor{orange}{Sha} chodam dam okor oty doldom}\\

\vspace{1em}
\noindent\begin{locator}[P2, 6]\end{locator} {\eva tchody ShocThol chocThey s}\\
\begin{locator}[P2, 7]\end{locator} {\eva ychos ychol daiin cThol dol}\\
\begin{locator}[P2, 8]\end{locator} {\eva ychor chor okchey qokom}\\
\begin{locator}[P2, 9]\end{locator} {\eva oeeo dal chor cThom s}\\
\begin{locator}[P2, 10]\end{locator} {\eva qokch\textcolor{orange}{od} ychear kchdy}\\
\begin{locator}[P2, 11]\end{locator} {\eva lor char otam cThom dy}\\
\begin{locator}[P2, 12]\end{locator} {\eva ytchos Shy qokam cThy}\\
\begin{locator}[P2, 13]\end{locator} {\eva yodaiin cThy s chor oees or}\\
\begin{locator}[P2, 14]\end{locator} {\eva qokor chol cThol tchalody}\\
\begin{locator}[P2, 15]\end{locator} {\eva chocKhy s os chy sain or}\\
\begin{locator}[P2, 16]\end{locator} {\eva ochy cThar cThar cThy}\\
\begin{locator}[P2, 17]\end{locator} {\eva y chaiir cKhal cThodam dy}\\
\begin{locator}[P2, 18]\end{locator} {\eva ytcho\textcolor{orange}{$\frown$}cThol ches cThor}\\
\begin{locator}[P2, 19]\end{locator} {\eva ocholy kchos chy dor}\\
\begin{locator}[P2, 20]\end{locator} {\eva dchor choldar okol daiin}\\
\begin{locator}[P2, 21]\end{locator} {\eva ycheor chor ocTham}\\


%%%%%%
% 7R %
%%%%%%
\clearpage
\subsection*{Folio 7, Recto}
\begin{locator}[P1, 1]\end{locator} {\eva fchodaiin Shopchey qko Shey qoos Sheey ch\textcolor{orange}{a}rochy}\\
\begin{locator}[P1, 2]\end{locator} {\eva dcheey keor Shor dold dchey kchey otchy cheody}\\
\begin{locator}[P1, 3]\end{locator} {\eva oeeees cheodaiin Sheey ytcheey qotchy chald}\\
\begin{locator}[P1, 4]\end{locator} {\eva qokcho cho lochey daiin ychey kchos odaiin}\\
\begin{locator}[P1, 5]\end{locator} {\eva oaiir otaiin}\\

\vspace{1em}
\noindent\begin{locator}[P2, 6]\end{locator} {\eva kSholo\textcolor{orange}{Sh}ey qotoees chkoldy otchor choaiin}\\
\begin{locator}[P2, 7]\end{locator} {\eva dShoy cThol chol otchol dain Shody Shol chotchy}\\
\begin{locator}[P2, 8]\end{locator} {\eva okchey deeeese choty qokchy Shol keey choty dain}\\
\begin{locator}[P2, 9]\end{locator} {\eva qokechy olchoiin chol cPhey ShcKhy chochy kchod}\\
\begin{locator}[P2, 10]\end{locator} {\eva schain chor daiin chcKhy}\\


%%%%%%
% 7V %
%%%%%%
\clearpage
\subsection*{Folio 7, Verso}
\begin{locator}[P1, 1]\end{locator} {\eva polyShy Shey tchody qopchy otShol dy\textcolor{orange}{$\frown$}daiin tShodody}\\
\begin{locator}[P1, 2]\end{locator} {\eva chochy cThy daiin qoky chcPhhy daiin cThol cThy cThd}\\
\begin{locator}[P1, 3]\end{locator} {\eva qokchy dykchy chkeey kShy ky ty dor cheey ol\textcolor{orange}{$\frown$}cheol\textcolor{orange}{$\frown$}dy}\\
\begin{locator}[P1, 4]\end{locator} {\eva choteeen oeear choschy dain Sho\textcolor{orange}{$\frown$}kShy Shol deees\textcolor{orange}{$\frown$}dol}\\
\begin{locator}[P1, 5]\end{locator} {\eva dchodaiin qotchy cheey tcheey}\\

\vspace{1em}
\noindent\begin{locator}[P2, 6]\end{locator} {\eva kchor Sheod Sheodaiin Shodaiin okSho\textcolor{orange}{$\frown$}lShol dai\textcolor{orange}{r} qos}\\
\begin{locator}[P2, 7]\end{locator} {\eva okSho\textcolor{orange}{$\frown$}deeen chor\textcolor{orange}{$\frown$}cheor odaiin Shotch\textcolor{orange}{*} dol dol dor aiin}\\
\begin{locator}[P2, 8]\end{locator} {\eva qoteeeo rcho\textcolor{orange}{$\frown$}cheeody qotchey tey okchor daiin}\\
\begin{locator}[P2, 9]\end{locator} {\eva Sho keeo daiir chokchy dor deol dy dol daiin}\\


%%%%%%
% 8R %
%%%%%%
\clearpage
\subsection*{Folio 8, Recto}
\begin{locator}[P1, 1]\end{locator} {\eva pShol chor otShal chopy cPhol chody Shy cFhodar Shor}\\
\begin{locator}[P1, 2]\end{locator} {\eva tchty Sh kcheals Sho okche do dchy dain al}\\
\begin{locator}[P1, 3]\end{locator} {\eva chodar Shy ry chodaiin Shokchy chor dy}\\
\begin{locator}[P1, 4]\end{locator} {\eva qotor chor chor Sheey dchol Shesed chofchy dam}\\
\begin{locator}[P1, 5]\end{locator} {\eva okchey do r cheeey dy ky scho chky ckooaiin ch\textcolor{orange}{o} taiin}\\
\begin{locator}[P1, 6]\end{locator} {\eva toSh ckcheey koltoldy Shy cho\textcolor{orange}{e}ty cheeody sol}\\
\begin{locator}[P1, 7]\end{locator} {\eva choto kchoan choor dain}\\
\begin{locator}[T1, 8]\end{locator} {\eva \textcolor{orange}{d}cho daiin}\\

\vspace{1em}
\noindent\begin{locator}[P2, 9]\end{locator} {\eva tchoep Sho pcheey pchey ofchey dSheey Shol\textcolor{orange}{$\frown$}daiin Shor}\\
\begin{locator}[P2, 10]\end{locator} {\eva daiin cheey teeodan dy cheocThy okSheo dol dair\textcolor{orange}{g}}\\
\begin{locator}[P2, 11]\end{locator} {\eva Shol cheodaiin daiin do ytchody chot choty otariin}\\
\begin{locator}[P2, 12]\end{locator} {\eva qochodaiin Shotokody chotol}\\
\begin{locator}[T2, 13]\end{locator} {\eva okokchod\textcolor{orange}{g}}\\

\vspace{1em}
\noindent\begin{locator}[P3, 14]\end{locator} {\eva cTho cThey Shol chofydy Sho chey kShey lody cholal}\\
\begin{locator}[P3, 15]\end{locator} {\eva dchey cKhol chol chey kchs chy \textcolor{orange}{cT}odaiin dol daiiirchy cKhy}\\
\begin{locator}[P3, 16]\end{locator} {\eva ychey kchokchy chsey kchy scheaiin cThaichar cThy dar}\\
\begin{locator}[P3, 17]\end{locator} {\eva chol dchy qokar chl aiin chean c\textcolor{orange}{K}y char chaiin}\\
\begin{locator}[P3, 18]\end{locator} {\eva okar cPhaiin chaiin el daiin chor cha rchealcham}\\
\begin{locator}[P3, 19]\end{locator} {\eva sair cheain cPhol dar Shol kaiin Shol kaiin daikam}\\
\begin{locator}[P3, 20]\end{locator} {\eva or chokesey Shey okal chal}\\
\begin{locator}[T3, 21]\end{locator} {\eva schol sair}\\


%%%%%%
% 8V %
%%%%%%
\clearpage
\subsection*{Folio 8, Verso}
\begin{locator}[P1, 1]\end{locator} {\eva cThod soocTh sol Shol otol chol opcheaiin opydaiin saiin}\\
\begin{locator}[P1, 2]\end{locator} {\eva ShcThal sar chor Sheaiin Shor chykchy otaiin cty}\\
\begin{locator}[P1, 3]\end{locator} {\eva qody cheal sy chory chear Shol chaiin Shaiin dolar}\\
\begin{locator}[P1, 4]\end{locator} {\eva dShol Shol d\textcolor{orange}{a}l chean cThar Shealy daiin chary}\\
\begin{locator}[P1, 5]\end{locator} {\eva chol chol dar otchar etaiin cThol dar}\\
\begin{locator}[P1, 6]\end{locator} {\eva daiin cThan ytchy chey kaiin dain ar}\\
\begin{locator}[P1, 7]\end{locator} {\eva Sho kchol dar Shey cThar chotain ry}\\
\begin{locator}[P1, 8]\end{locator} {\eva okchol kSh chol chol chol cThaiin dain}\\
\begin{locator}[P1, 9]\end{locator} {\eva Shol orchl chokchy chol cThor chaiin}\\
\begin{locator}[P1, 10]\end{locator} {\eva scharchy oeesody kchey pchy cPharom}\\
\begin{locator}[P1, 11]\end{locator} {\eva sorain}\\

\vspace{1em}
\noindent\begin{locator}[P2, 12]\end{locator} {\eva pchar cho rol dal Shear ch\textcolor{orange}{ch}otaiin chal daiin}\\
\begin{locator}[P2, 13]\end{locator} {\eva kchor otchar oky chokain keoky otorchy satar}\\
\begin{locator}[P2, 14]\end{locator} {\eva Shor okol lokaiin Shol kol char cThey tchy cKham}\\
\begin{locator}[P2, 15]\end{locator} {\eva or chol cheen chcky chor cheain char cheeky chor ry}\\
\begin{locator}[P2, 16]\end{locator} {\eva chor chear chear oteey dchor chodey cho raiin}\\
\begin{locator}[P2, 17]\end{locator} {\eva dain chear daiin}\\


%%%%%%
% 9R %
%%%%%%
\clearpage
\subsection*{Folio 9, Recto}
\begin{locator}[P1, 1]\end{locator} {\eva tydlo choly cThor orchey s Shy odaiin sary Shor cThy}\\
\begin{locator}[P1, 2]\end{locator} {\eva oykeey chol ytaiin okchody toeoky okoiin dy or chaiin}\\
\begin{locator}[P1, 3]\end{locator} {\eva toiin cPhy qotod otaiin cThy okor chey cThod ram}\\
\begin{locator}[P1, 4]\end{locator} {\eva yShy chokcho chcThod Shor Shaiin otar dor ytol dayty}\\
\begin{locator}[P1, 5]\end{locator} {\eva daiin chor sor cThy chokoiin Shol dSholdy otchol ot dy}\\

\vspace{1em}
\noindent\begin{locator}[P2, 6]\end{locator} {\eva pShoain cThyaiin okaiir cFhodoral Shar sy Shydal chdy}\\
\begin{locator}[P2, 7]\end{locator} {\eva or chol chytchy tchol ytor qotol chyky chodar aiin}\\
\begin{locator}[P2, 8]\end{locator} {\eva qotcho qokchy cThey koraiin okain d dal s olSho cThy}\\
\begin{locator}[P2, 9]\end{locator} {\eva ocho cThy choc\textcolor{orange}{T}oy chodykchy saiin dchy daiin}\\
\begin{locator}[T1, 10]\end{locator} {\eva ytchas oraiin chk\textcolor{orange}{o}r}\\


%%%%%%
% 9V %
%%%%%%
\clearpage
\subsection*{Folio 9, Verso}
\begin{locator}[P1, 1]\end{locator} {\eva fochor oporody opy Shor daiin qopchypcho qofol Shol cFhol daiin}\\
\begin{locator}[P1, 2]\end{locator} {\eva dchor qoaiin chkaiin cThor chol chor cPhol dy oty qokaiin dy}\\
\begin{locator}[P1, 3]\end{locator} {\eva ykey chor ykaiin daiin cThy otaiin oky oeees daiin}\\
\begin{locator}[P1, 4]\end{locator} {\eva ytey tchy kaiin cThor otol oty toldy}\\

\vspace{1em}
\noindent\begin{locator}[P2, 5]\end{locator} {\eva pchor ypcheey qotor ypchy olcFholy te ar chty daiiin}\\
\begin{locator}[P2, 6]\end{locator} {\eva odol choy kSheody chody dain otchy cThod yk\textcolor{orange}{o}}\\
\begin{locator}[P2, 7]\end{locator} {\eva qochol chol ctchy daiin otal dor daim}\\
\begin{locator}[P2, 8]\end{locator} {\eva soiin daiin qokcho rokyd daly}\\
\begin{locator}[P2, 9]\end{locator} {\eva daiin chy tor chyty dary ytoldy}\\
\begin{locator}[P2, 10]\end{locator} {\eva oty kchol chol chy kyty}\\
\begin{locator}[P2, 11]\end{locator} {\eva ychor chShoty oky kaiin}\\
\begin{locator}[P2, 12]\end{locator} {\eva chkaiin cKhy chor}\\


%%%%%%%
% 10R %
%%%%%%%
\clearpage
\subsection*{Folio 10, Recto}
\begin{locator}[P1, 1]\end{locator} {\eva pchocThy Shor ocThody chorchy pchodol chopchal ypch kom}\\
\begin{locator}[P1, 2]\end{locator} {\eva dchey cThoor char chty os chair otytchol oky daiin etyd}\\
\begin{locator}[P1, 3]\end{locator} {\eva qotor otchy daiin chocThy qotchy chol or yty dy dy}\\
\begin{locator}[P1, 4]\end{locator} {\eva sor chaiin chcThy cTho cKhy or aiin chtchor doiir ody}\\
\begin{locator}[P1, 5]\end{locator} {\eva qokchy qotchol chol cThy}\\

\vspace{1em}
\noindent\begin{locator}[P2, 6]\end{locator} {\eva ych\textcolor{orange}{s}or cThy chor cThaiin qocTholy dy chy taiin Shy}\\
\begin{locator}[P2, 7]\end{locator} {\eva dchy qokchol ykchaiin yty daiin cTh dain dair am}\\
\begin{locator}[P2, 8]\end{locator} {\eva qotchor chor otol chol cholor chol daiin dar}\\
\begin{locator}[P2, 9]\end{locator} {\eva oykchor Shor chor chy kaiiin dy chodaiin}\\
\begin{locator}[P2, 10]\end{locator} {\eva oqot\textcolor{orange}{o}r otor cFhy cThor osain ytoiin}\\
\begin{locator}[P2, 11]\end{locator} {\eva rotcho Shor qoty qotor cThy d otar}\\
\begin{locator}[P2, 12]\end{locator} {\eva rodaiin daiin qotchy qotor}\\


%%%%%%%
% 10V %
%%%%%%%
\clearpage
\subsection*{Folio 10, Verso}
\begin{locator}[P1, 1]\end{locator} {\eva paiin daiin Sheo pcheey qoty daiin cThor otydy sain}\\
\begin{locator}[P1, 2]\end{locator} {\eva dain daiin cKhy chcThor choiin qot chodaiin cThy daiin}\\
\begin{locator}[P1, 3]\end{locator} {\eva dSho ytey kchol olty chol dy}\\

\vspace{1em}
\noindent\begin{locator}[P2, 4]\end{locator} {\eva qotchytor Shoiin daiin qotchey ShcThey ytor dain}\\
\begin{locator}[P2, 5]\end{locator} {\eva Sho ykeey daiin qotchy qotor chol daiin qokchyky}\\
\begin{locator}[P2, 6]\end{locator} {\eva Shoiin chor ShcThy qoty qotoiin qokol choraiin}\\
\begin{locator}[P2, 7]\end{locator} {\eva qokol chyky chol cheky daiin dain chcKhan}\\


%%%%%%%
% 11R %
%%%%%%%
\clearpage
\subsection*{Folio 11, Recto}
\begin{locator}[P1, 1]\end{locator} {\eva tShol schoal cFhy Shfydaiin cPhy Shey tchody Shoyty}\\
\begin{locator}[P1, 2]\end{locator} {\eva socThody qodor y kShy daiin ytchy ytchoky kchol daiin}\\
\begin{locator}[P1, 3]\end{locator} {\eva qoty chol cThy dor ykychy choty dain chaiin daiin ded}\\
\begin{locator}[P1, 4]\end{locator} {\eva dchol chy kchy dy daiin}\\

\vspace{1em}
\noindent\begin{locator}[P2, 5]\end{locator} {\eva tchol Shor Shor dky \textcolor{orange}{S}cPhy daiin cThy dy chodl daiin}\\
\begin{locator}[P2, 6]\end{locator} {\eva odl ds y otol chaiin ykchor dair chody cThy s daiin}\\
\begin{locator}[P2, 7]\end{locator} {\eva qotchy okchol cThy dy}\\


%%%%%%%
% 11V %
%%%%%%%
\clearpage
\subsection*{Folio 11, Verso}
\begin{locator}[P1, 1]\end{locator} {\eva poldchody ShcPhy Shordy qoty Shol cPhar dan y}\\
\begin{locator}[P1, 2]\end{locator} {\eva Shol dy chcKhy ShcThy daiin dam ykchy dain dchy}\\
\begin{locator}[P1, 3]\end{locator} {\eva otchor dy kchy tchy \textcolor{orange}{d}ar qokchd oky chol dy dy}\\
\begin{locator}[P1, 4]\end{locator} {\eva qokchor chololer chyky dchy qoky cTho tchey tan}\\
\begin{locator}[P1, 5]\end{locator} {\eva soydy qoteey qot\textcolor{orange}{$\smile$}chor dy ddy cThor Shy arg}\\
\begin{locator}[P1, 6]\end{locator} {\eva ycheor kSho dor cThey schold}\\


%%%%%%
% 12 %
%%%%%%
\clearpage
\subsection*{Folio 12}
Folio 12 is missing from the manuscript.


%%%%%%%
% 13R %
%%%%%%%
\clearpage
\subsection*{Folio 13, Recto}
\begin{locator}[P1, 1]\end{locator} {\eva torShor opchy Shol dy qopchy Shol opchor dypchy dchm}\\
\begin{locator}[P1, 2]\end{locator} {\eva dchol chol dol Shkchy ydal Shy ykchy qot\textcolor{orange}{$\smile$}ey daiin s y}\\
\begin{locator}[P1, 3]\end{locator} {\eva s y dchor Shaiin oeees ykor chor ytShy ykchy kchy dar}\\
\begin{locator}[P1, 4]\end{locator} {\eva qodchy ytchy otchor}\\

\vspace{1em}
\noindent\begin{locator}[P2, 5]\end{locator} {\eva Shorodo Shy tShy kchol dpchy qopchy otchol cFhol dy}\\
\begin{locator}[P2, 6]\end{locator} {\eva tchor dor daiin qotchol okchy okchor oiin chcKhy d}\\
\begin{locator}[P2, 7]\end{locator} {\eva dchy qoky chol dy qokhy d oldy okchor doaiin}\\
\begin{locator}[P2, 8]\end{locator} {\eva Shochy qokchy torchy kcc\textcolor{orange}{K}y s okchey daiin}\\
\begin{locator}[P2, 9]\end{locator} {\eva oldy Shey chol doiin ykoly okchal daldy}\\
\begin{locator}[P2, 10]\end{locator} {\eva sotchy kchy okorory}\\


%%%%%%%
% 13V %
%%%%%%%
\clearpage
\subsection*{Folio 13, Verso}
\begin{locator}[P1, 1]\end{locator} {\eva koair chtoiin otchy kchod otol otchy ocThos}\\
\begin{locator}[P1, 2]\end{locator} {\eva oko qokol chodal otchol cPhol choty}\\
\begin{locator}[P1, 3]\end{locator} {\eva qokchy qokod chy otchy cThody}\\
\begin{locator}[P1, 4]\end{locator} {\eva ols chey okos oaiin okShy qoc\textcolor{orange}{K}y}\\
\begin{locator}[P1, 5]\end{locator} {\eva qoky daiin}\\

\vspace{1em}
\noindent\begin{locator}[P2, 6]\end{locator} {\eva foldaiin olcPhy Shol dy oty Shor qotyd dairo d}\\
\begin{locator}[P2, 7]\end{locator} {\eva dain okal chy qokchory dchy koky daiin Sho\textcolor{orange}{n}}\\
\begin{locator}[P2, 8]\end{locator} {\eva otchy daiin y dain ykol okchy okald d ytaiin}\\
\begin{locator}[P2, 9]\end{locator} {\eva tchtod otal cThor ytal y cho t\textcolor{orange}{o}l Sho qocThy}\\
\begin{locator}[P2, 10]\end{locator} {\eva y ol chy kchey kchor dal}\\


%%%%%%%
% 14R %
%%%%%%%
\clearpage
\subsection*{Folio 14, Recto}
\begin{locator}[P1, 1]\end{locator} {\eva pchodaiin chopol Shoiin daiin dain}\\
\begin{locator}[P1, 2]\end{locator} {\eva o ykeey soiiin chok qokchy da okol}\\
\begin{locator}[P1, 3]\end{locator} {\eva ydaiin olchy kchor daiin olol}\\
\begin{locator}[P1, 4]\end{locator} {\eva ochkch\textcolor{orange}{o}r kol Shy daiin dorody}\\
\begin{locator}[P1, 5]\end{locator} {\eva qokchol dar dal\textcolor{orange}{o} qotolo}\\
\begin{locator}[P1, 6]\end{locator} {\eva ychol oir okor choor ocKhy}\\
\begin{locator}[P1, 7]\end{locator} {\eva otcho dain chcKhy}\\

\vspace{1em}
\noindent\begin{locator}[P2, 8]\end{locator} {\eva soShy fchol Shor cheor ykaiin s}\\
\begin{locator}[P2, 9]\end{locator} {\eva sody chody otchody qotchy koiin sy Shoty dy}\\
\begin{locator}[P2, 10]\end{locator} {\eva qotchor chod Shoty chody dol dy dy okchy dy}\\
\begin{locator}[P2, 11]\end{locator} {\eva dchokchy schol dy Shey dar qoty ykeey ky}\\
\begin{locator}[P2, 12]\end{locator} {\eva oeeen chey \textcolor{orange}{k}eor chey tchy ky chodalg}\\
\begin{locator}[P2, 13]\end{locator} {\eva sodaiin chy kchy kchy ykeody}\\


%%%%%%%
% 14V %
%%%%%%%
\clearpage
\subsection*{Folio 14, Verso}
\begin{locator}[P1, 1]\end{locator} {\eva pdychoiin yfodain otyShy dy ypchor daiin kol ydain}\\
\begin{locator}[P1, 2]\end{locator} {\eva okchor dchy tShy oky chy cThy otchy ty chol daiin}\\
\begin{locator}[P1, 3]\end{locator} {\eva ychy dy daiin chcThy ykykaiin dytchy y\textcolor{orange}{K}chy ky dy}\\
\begin{locator}[P1, 4]\end{locator} {\eva yty\textcolor{orange}{$\smile$}chy kSho ykShy ShokShor yty darody dyotyds}\\
\begin{locator}[P1, 5]\end{locator} {\eva okShy daiin okchor chky qotchy daiin cThor oty}\\
\begin{locator}[P1, 6]\end{locator} {\eva qoty choky cThy chokchy dydydy chcKhy dchyd n}\\
\begin{locator}[P1, 7]\end{locator} {\eva oykShy choty dydy odyd otchy okchy dShy dardy}\\
\begin{locator}[P1, 8]\end{locator} {\eva chokShor daiin okShody daiin dol dair dam}\\
\begin{locator}[P1, 9]\end{locator} {\eva dykchy cTholdg dchcKhy}\\


%%%%%%%
% 15R %
%%%%%%%
\clearpage
\subsection*{Folio 15, Recto}
\begin{locator}[P1, 1]\end{locator} {\eva tShor Shey tchaly Shy chtols Shey daiin}\\
\begin{locator}[P1, 2]\end{locator} {\eva otchor qokchor oly okor Shy koly}\\
\begin{locator}[P1, 3]\end{locator} {\eva qokaiin qotchy tydy daiin chol cThy}\\
\begin{locator}[P1, 4]\end{locator} {\eva scheaiin chodaiin chl sol cKhaiin sal}\\
\begin{locator}[P1, 5]\end{locator} {\eva qotchy r Shor cThy daiin cThy dy}\\
\begin{locator}[P1, 6]\end{locator} {\eva dchy kokaiin chdy saiin okear}\\
\begin{locator}[P1, 7]\end{locator} {\eva daiin Shkaiin cThy Sho keocThy}\\
\begin{locator}[P1, 8]\end{locator} {\eva ShocThy tol kaiin s dain cTholy}\\
\begin{locator}[P1, 9]\end{locator} {\eva ocThain qokaiin chos odaiin cThl s y}\\
\begin{locator}[P1, 10]\end{locator} {\eva ychain chcKhhy okShy saiiin dolchds}\\
\begin{locator}[P1, 11]\end{locator} {\eva okaiin otaiin chl sy chor choross}\\
\begin{locator}[P1, 12]\end{locator} {\eva qotor Shor tcheor chy cThaiin Shan}\\
\begin{locator}[P1, 13]\end{locator} {\eva ykShol dor Sheey cThy dain sky Shor Shoty}\\
\begin{locator}[P1, 14]\end{locator} {\eva otcho kchy chol daiin cThar ytol dor dom}\\
\begin{locator}[P1, 15]\end{locator} {\eva qotchor chaiin chy kol\textcolor{orange}{$\smile$}daky}\\


%%%%%%%
% 15V %
%%%%%%%
\clearpage
\subsection*{Folio 15, Verso}
\begin{locator}[P1, 1]\end{locator} {\eva poror orShy choiin dtchan opchor dy}\\
\begin{locator}[P1, 2]\end{locator} {\eva *chor or oro r aiin cThy \textcolor{orange}{t}ain dar}\\
\begin{locator}[P1, 3]\end{locator} {\eva cThor daiin qokor okeor okaiin}\\
\begin{locator}[P1, 4]\end{locator} {\eva doiin choky Shol qoky qotchod}\\
\begin{locator}[P1, 5]\end{locator} {\eva otchor chor chor ytchor cThy s}\\
\begin{locator}[P1, 6]\end{locator} {\eva qotchey choty kaiin otchy r aiin}\\
\begin{locator}[P1, 7]\end{locator} {\eva coy choiin Sho s chy s chy tor ols}\\
\begin{locator}[P1, 8]\end{locator} {\eva ytchor chor ol oiin oty Shol daiin}\\
\begin{locator}[P1, 9]\end{locator} {\eva otcholocThol chol chol chody kan}\\
\begin{locator}[P1, 10]\end{locator} {\eva sor chor cThoiin cThy qokaiin}\\
\begin{locator}[P1, 11]\end{locator} {\eva soloiin cheor chol daiin cThy}\\
\begin{locator}[P1, 12]\end{locator} {\eva daiin cThor chol chor}\\


%%%%%%%
% 16R %
%%%%%%%
\clearpage
\subsection*{Folio 16, Recto}
\begin{locator}[P1, 1]\end{locator} {\eva pocheody qopchey sykaiin opchy dor ychy daiin dy chor orom}\\
\begin{locator}[P1, 2]\end{locator} {\eva ychykchy otly kol Shor ody otody qoy oeesordy}\\
\begin{locator}[P1, 3]\end{locator} {\eva ydor Sheal okchy qoy koiin choky ykair}\\
\begin{locator}[T1, 4]\end{locator} {\eva dainod ychealod}\\

\vspace{1em}
\noindent\begin{locator}[P2, 5]\end{locator} {\eva tchor chor chs ykch ShocThy opchy ty ky}\\
\begin{locator}[P2, 6]\end{locator} {\eva oShaiin dyky oeees deeeod aiin dtoaiin}\\
\begin{locator}[P2, 7]\end{locator} {\eva daiin dalchy dyky schy saiin doal qoky}\\
\begin{locator}[P2, 8]\end{locator} {\eva Shotchy ydain yky Shody otol daiin}\\
\begin{locator}[P2, 9]\end{locator} {\eva saiin ytaiin}\\

\vspace{1em}
\noindent\begin{locator}[P3, 10]\end{locator} {\eva toror dalydal opchy fchol ypchoc\textcolor{orange}{F}y okal}\\
\begin{locator}[P3, 11]\end{locator} {\eva sokchy qokol choty\textcolor{orange}{$\smile$}okchy cThy chy kchy}\\
\begin{locator}[P3, 12]\end{locator} {\eva dychokchy ShcThy ShtShy Sho tchokyd}\\
\begin{locator}[P3, 13]\end{locator} {\eva qokchor dl dy Shey}\\


%%%%%%%
% 16V %
%%%%%%%
\clearpage
\subsection*{Folio 16, Verso}
\begin{locator}[P1, 1]\end{locator} {\eva pchraiin otchor chpchol chpchey s pchocty}\\
\begin{locator}[P1, 2]\end{locator} {\eva ytchor y ky chokchy qokchocThor Shory}\\
\begin{locator}[P1, 3]\end{locator} {\eva ykchy dy choy qoty chy kchy koShet}\\
\begin{locator}[P1, 4]\end{locator} {\eva dchol chcThody cPhod chotol dal}\\
\begin{locator}[P1, 5]\end{locator} {\eva ytchy chyty chor chol ytchy dan}\\
\begin{locator}[P1, 6]\end{locator} {\eva sor chk\textcolor{orange}{o}r oty chk\textcolor{orange}{o}r chol dairin}\\

\vspace{1em}
\noindent\begin{locator}[P2, 7]\end{locator} {\eva pchocThy chypchy qotchy chcFhhy sy}\\
\begin{locator}[P2, 8]\end{locator} {\eva daiin chol y daiin chcThy qotchar chor Sholo}\\
\begin{locator}[P2, 9]\end{locator} {\eva dShy okaiin okaiin chol chor cThor ty chody}\\
\begin{locator}[P2, 10]\end{locator} {\eva qokchy chydy ykchy chcKhy otain cThor cThy}\\
\begin{locator}[P2, 11]\end{locator} {\eva okytaiin chkchy saiin}\\
\begin{locator}[P2, 12]\end{locator} {\eva daiin yky otor chody}\\
\begin{locator}[P2, 13]\end{locator} {\eva sokar oaorar}\\


%%%%%%%
% 17R %
%%%%%%%
\clearpage
\subsection*{Folio 17, Recto}
\begin{locator}[P1, 1]\end{locator} {\eva fShody daram ydarchom opydy ypod chop otchy dody oldcKhy}\\
\begin{locator}[P1, 2]\end{locator} {\eva ydair \textcolor{orange}{c}hoky okShy qodar cKhody dor otchol qodcThy ods}\\
\begin{locator}[P1, 3]\end{locator} {\eva chol or chy qodam okor chor okchom}\\

\vspace{1em}
\noindent\begin{locator}[P2, 4]\end{locator} {\eva tcho\textcolor{orange}{m} Shol qokol qor olaiin opydg som ypchy ypaim}\\
\begin{locator}[P2, 5]\end{locator} {\eva ychekchy cThy chor Shor cPhor cPhaldy dair cThey qody}\\
\begin{locator}[P2, 6]\end{locator} {\eva tSho qofcho qokcheor cheke\textcolor{orange}{y}}\\

\vspace{1em}
\noindent\begin{locator}[P3, 7]\end{locator} {\eva kSheo qokchy choldShy zepchy \textcolor{orange}{dShe} opchordy}\\
\begin{locator}[P3, 8]\end{locator} {\eva dchchy dychear schar ykchy}\\
\begin{locator}[P3, 9]\end{locator} {\eva soy chcKh\textcolor{orange}{o} o dar chypcham}\\
\begin{locator}[P3, 10]\end{locator} {\eva dar chear dSheor \textcolor{orange}{d}ain y mol}\\
\begin{locator}[P3, 11]\end{locator} {\eva otchol cThar okaiin chol daiiin}\\
\begin{locator}[P3, 12]\end{locator} {\eva ychody chotom}\\


%%%%%%%
% 17V %
%%%%%%%
\clearpage
\subsection*{Folio 17, Verso}
\begin{locator}[P1, 1]\end{locator} {\eva pchodol chor fchy opydaiin odaldy}\\
\begin{locator}[P1, 2]\end{locator} {\eva ycheey keeor cThodal okol odaiin okal}\\
\begin{locator}[P1, 3]\end{locator} {\eva oldaim odaiin okal oldaiin chocKhol olol}\\
\begin{locator}[P1, 4]\end{locator} {\eva kchor fchol cPhol olcheol okeeey}\\
\begin{locator}[P1, 5]\end{locator} {\eva ychol chol dolcheey tchol dar cKhy}\\
\begin{locator}[P1, 6]\end{locator} {\eva oekor or okaiin or otaiin d}\\
\begin{locator}[P1, 7]\end{locator} {\eva sor chkeey poiis cheor os s aiin}\\
\begin{locator}[P1, 8]\end{locator} {\eva qokeey kchar ol dy choldaiin sy}\\
\begin{locator}[P1, 9]\end{locator} {\eva ycheol Shol kchol choltaiin ol}\\
\begin{locator}[P1, 10]\end{locator} {\eva oytor okeor okar okol doiiram}\\
\begin{locator}[P1, 11]\end{locator} {\eva qokcheo qokoiir cTheol chol}\\
\begin{locator}[P1, 12]\end{locator} {\eva oy choy koaiin chcKhey ol chor}\\
\begin{locator}[P1, 13]\end{locator} {\eva ykeor chol chol cThol chkor Sheol}\\
\begin{locator}[P1, 14]\end{locator} {\eva olo r okeeol chodaiin okeol tchory}\\
\begin{locator}[P1, 15]\end{locator} {\eva ychor cThy cheeky cheo otor oteol}\\
\begin{locator}[P1, 16]\end{locator} {\eva okcheol chol okeol cThol otcheolom}\\
\begin{locator}[P1, 17]\end{locator} {\eva qoain sar She dol qopchaiin cThor}\\
\begin{locator}[P1, 18]\end{locator} {\eva otor cheeor ol chol dor chr or eees}\\
\begin{locator}[P1, 19]\end{locator} {\eva dain chey qoaiin cThor chol chom}\\
\begin{locator}[P1, 20]\end{locator} {\eva ykeey okeey cheor chol Sho ydaiin}\\
\begin{locator}[P1, 21]\end{locator} {\eva oal Sheor Sholor or ShecThy cPheor daiin}\\
\begin{locator}[P1, 22]\end{locator} {\eva qokeee dar chey keeor cheeol cTheey cThy}\\
\begin{locator}[P1, 23]\end{locator} {\eva chkeey okeor \textcolor{orange}{S}har okeom}\\


%%%%%%%
% 18R %
%%%%%%%
\clearpage
\subsection*{Folio 18, Recto}
\begin{locator}[P1, 1]\end{locator} {\eva pdrairdy darod\textcolor{orange}{c}f yoar ykchol dar o m chcKhy ocKhor dal}\\
\begin{locator}[P1, 2]\end{locator} {\eva otShol qokchol chykchy okchal daiin dy chol diiin}\\
\begin{locator}[P1, 3]\end{locator} {\eva qokchor chor chcKhy orchey qokchol dy ytchar g}\\
\begin{locator}[P1, 4]\end{locator} {\eva chor cThor okeor ykchol okain}\\

\vspace{1em}
\noindent\begin{locator}[P2, 5]\end{locator} {\eva tchor Shor cThaiin cThol chlol chom}\\
\begin{locator}[P2, 6]\end{locator} {\eva ychy kchor dair ytol chcThy dar dar dal}\\
\begin{locator}[P2, 7]\end{locator} {\eva oShor Shaiin cThy Sholdy doldy doldaiin}\\
\begin{locator}[P2, 8]\end{locator} {\eva qokchor cKhol olody okaldy dary}\\
\begin{locator}[P2, 9]\end{locator} {\eva chol chcThal okShal chykald}\\
\begin{locator}[P2, 10]\end{locator} {\eva dar Shor qokchol ol ydaiin}\\
\begin{locator}[P2, 11]\end{locator} {\eva sotchaiin chokchy ch\textcolor{orange}{cK}hol chor g}\\
\begin{locator}[P2, 12]\end{locator} {\eva ychair cThol daiin qokchy cThy}\\
\begin{locator}[P2, 13]\end{locator} {\eva or Shaiin cThar cThal okal dar}\\
\begin{locator}[T1, 14]\end{locator} {\eva ychekchy kchaiin}\\


%%%%%%%
% 18V %
%%%%%%%
\clearpage
\subsection*{Folio 18, Verso}
\begin{locator}[P1, 1]\end{locator} {\eva told Sh\textcolor{orange}{o}r ytShy otchdal dchal dchy ytd\textcolor{orange}{m}}\\
\begin{locator}[P1, 2]\end{locator} {\eva qoeees or oaiin Shy\textcolor{orange}{*} okShy qokchy qokchy s g}\\
\begin{locator}[P1, 3]\end{locator} {\eva or Shy qoky qoky chkchy qokShy qokam}\\
\begin{locator}[P1, 4]\end{locator} {\eva qotchy qokay qokchy ykcho ydl dar}\\
\begin{locator}[P1, 5]\end{locator} {\eva \textcolor{orange}{$\cancel{x}\frown$}ychoees ykchy qol kchy qotchol daiir om}\\
\begin{locator}[P1, 6]\end{locator} {\eva qotor chor otchy qokeees chy s ar ykar}\\
\begin{locator}[P1, 7]\end{locator} {\eva ychol dor chod qokol daiin qokol dar dy}\\
\begin{locator}[P1, 8]\end{locator} {\eva tolol Sh cPhoy daror ddy ytor ykam}\\
\begin{locator}[P1, 9]\end{locator} {\eva okchor qotchy qokchy ytol doky dy}\\
\begin{locator}[P1, 10]\end{locator} {\eva yk\textcolor{orange}{o} dShy dair ykol dom}\\


%%%%%%%
% 19R %
%%%%%%%
\clearpage
\subsection*{Folio 19, Recto}
\begin{locator}[P1, 1]\end{locator} {\eva pchor qodchy qotShy dy tchy qotchy qoky daiin dchydy}\\
\begin{locator}[P1, 2]\end{locator} {\eva dShy chor y tchy chol dytchy chordy daiin dyty s choy}\\
\begin{locator}[P1, 3]\end{locator} {\eva oscheor Shy tdaiin chol dor yky}\\
\begin{locator}[P1, 4]\end{locator} {\eva qokorar daiin chcKhy Shy kchor}\\
\begin{locator}[P1, 5]\end{locator} {\eva otchy tchy qoky daiin r}\\
\begin{locator}[P1, 6]\end{locator} {\eva y Shor Shy daiin otytchy daiin}\\
\begin{locator}[P1, 7]\end{locator} {\eva qokcho ky cThar dor chan dar}\\
\begin{locator}[P1, 8]\end{locator} {\eva or chor daky dal chor dorl Shy}\\
\begin{locator}[P1, 9]\end{locator} {\eva qotchor dy dor y tchykchy Shdaiin}\\
\begin{locator}[P1, 10]\end{locator} {\eva daiin cThor chol ykchor chordy}\\
\begin{locator}[P1, 11]\end{locator} {\eva qotchy qolody choldy cThyd}\\
\begin{locator}[P1, 12]\end{locator} {\eva ykchor chor daiin daiinol}\\
\begin{locator}[P1, 13]\end{locator} {\eva osocThor ytchor}\\


%%%%%%%
% 19V %
%%%%%%%
\clearpage
\subsection*{Folio 19, Verso}
\begin{locator}[P1, 1]\end{locator} {\eva pochaiin cThor chpcheos opchey py kchy}\\
\begin{locator}[P1, 2]\end{locator} {\eva qokchy kchol sor qokcho\textcolor{orange}{r} ykchy darom}\\
\begin{locator}[P1, 3]\end{locator} {\eva otchy chol daiin qotol ytol daii\textcolor{orange}{i}n}\\
\begin{locator}[P1, 4]\end{locator} {\eva ytch chcThy qotol daiin daiin}\\
\begin{locator}[P1, 5]\end{locator} {\eva qotchy qoteey daiin doty qot}\\
\begin{locator}[P1, 6]\end{locator} {\eva ychoy kchor cThol chocThy s}\\
\begin{locator}[P1, 7]\end{locator} {\eva ycho r chaiin cThor}\\

\vspace{1em}
\noindent\begin{locator}[P2, 8]\end{locator} {\eva toy tchey qo dchol qokchs dom}\\
\begin{locator}[P2, 9]\end{locator} {\eva ychor oky chor yt\textcolor{orange}{o}l chol oky ddor}\\
\begin{locator}[P2, 10]\end{locator} {\eva daiin chor daiin \textcolor{orange}{y}okor y okchan}\\
\begin{locator}[P2, 11]\end{locator} {\eva qotol d\textcolor{orange}{o}r okchor daiin cThor otam}\\
\begin{locator}[P2, 12]\end{locator} {\eva otch okchodShy daiin or otaiin dai\textcolor{orange}{r}}\\
\begin{locator}[P2, 13]\end{locator} {\eva yees ykchol oty ytor ytar ytchor ytaiin}\\
\begin{locator}[T1, 14]\end{locator} {\eva otcholcheaiin cThol}\\


%%%%%%%
% 20R %
%%%%%%%
\clearpage
\subsection*{Folio 20, Recto}
\begin{locator}[P1, 1]\end{locator} {\eva kdchody chopy cheey qotchol qotoeey dchor choiin}\\
\begin{locator}[P1, 2]\end{locator} {\eva chodey cThey chotol odaiir qotchy cThody chodchy}\\
\begin{locator}[P1, 3]\end{locator} {\eva qoteey cho chodaiin Sho qochy chey tcheodal daral}\\
\begin{locator}[P1, 4]\end{locator} {\eva ochol ol\textcolor{orange}{$\frown$}teey otolchey}\\

\vspace{1em}
\noindent\begin{locator}[P2, 5]\end{locator} {\eva pchocThy chokoaiin cpy cheeen opchey Shosaiin r}\\
\begin{locator}[P2, 6]\end{locator} {\eva choees okchor qotol cheey daiin chy choiin daiin}\\
\begin{locator}[P2, 7]\end{locator} {\eva ocholShod daiin choteol chol dol Shol otaiin}\\
\begin{locator}[P2, 8]\end{locator} {\eva schodain cheo r cheody}\\

\vspace{1em}
\noindent\begin{locator}[P3, 9]\end{locator} {\eva fchodees Shody qotchey qokchey qocPhy chokoldy}\\
\begin{locator}[P3, 10]\end{locator} {\eva ochoaiin chor sody pchodaiin chetody choky dchy toy}\\
\begin{locator}[P3, 11]\end{locator} {\eva dchod qoteeody ytchy qotShey dchaiin cholody}\\
\begin{locator}[P3, 12]\end{locator} {\eva Shoiin cheody otchey otchy tchy qoteey daiin dar}\\
\begin{locator}[P3, 13]\end{locator} {\eva ochaiin chor chor cheey tchey}\\


%%%%%%%
% 20V %
%%%%%%%
\clearpage
\subsection*{Folio 20, Verso}
\begin{locator}[P1, 1]\end{locator} {\eva faiis ar okoy Shy pofochey opchy qopy choldy opydy cPhy}\\
\begin{locator}[P1, 2]\end{locator} {\eva sos yk\textcolor{orange}{o}iin cheol chol choiin checThy otol chol chodaiin oty}\\
\begin{locator}[P1, 3]\end{locator} {\eva okchy Sho kchol Shol chcThy qoty chy tol\textcolor{orange}{$\frown$}Shy qotchy}\\
\begin{locator}[P1, 4]\end{locator} {\eva Sho or aiin Shol daiin}\\

\vspace{1em}
\noindent\begin{locator}[P2, 5]\end{locator} {\eva tShol folchol otor Shol Shos fShodchy otchy chcPhy dy}\\
\begin{locator}[P2, 6]\end{locator} {\eva doiiin chocKhy dar cheocKhy Shos cheos char cThaiin}\\
\begin{locator}[P2, 7]\end{locator} {\eva ShocThy Sho cThy daiin Sheoy tey s soaiin}\\
\begin{locator}[P2, 8]\end{locator} {\eva Shain choraly Sho ar chy daiin d s}\\
\begin{locator}[P2, 9]\end{locator} {\eva ykchy keody cho cThy chol Shd qoty d \textcolor{orange}{yro}}\\
\begin{locator}[P2, 10]\end{locator} {\eva Shokaiin chocThy chol daiin chy chor ety}\\
\begin{locator}[P2, 11]\end{locator} {\eva okoiin chey cPhol chor y}\\


%%%%%%%
% 21R %
%%%%%%%
\clearpage
\subsection*{Folio 21, Recto}
\begin{locator}[P1, 1]\end{locator} {\eva pchor oeeocKhy ofychey ypchey qopcheody otaiin chan}\\
\begin{locator}[P1, 2]\end{locator} {\eva saiin chcPhy oky Sheaiin qotchol oteos Sheey cThy daiin}\\
\begin{locator}[P1, 3]\end{locator} {\eva qotol Shy ol cheor chy qokchey chey keey dy}\\

\vspace{1em}
\noindent\begin{locator}[P2, 4]\end{locator} {\eva pchofychy daiin cThain otoloSheey qocThey tolchory}\\
\begin{locator}[P2, 5]\end{locator} {\eva ykeey daiin chosy qokoiin otol chol qotcheol okeoaiin}\\
\begin{locator}[P2, 6]\end{locator} {\eva dchor y kolyky chol kol qokeol cholol qoteeol d\textcolor{orange}{o}dy}\\
\begin{locator}[P2, 7]\end{locator} {\eva Sho\textcolor{orange}{s}or cheor chokeody cho cThor Shy}\\

\vspace{1em}
\noindent\begin{locator}[P3, 8]\end{locator} {\eva fchokShy otor \textcolor{orange}{S}heol ocPhal op\textcolor{orange}{S}he\textcolor{orange}{o}s cThodaiin oty}\\
\begin{locator}[P3, 9]\end{locator} {\eva okaiin Sho tShaiin chkaiin ShcThey cThody cThy s}\\
\begin{locator}[P3, 10]\end{locator} {\eva totchy keor chy ky qotaiin qotchol ty cTheey otaiin}\\
\begin{locator}[P3, 11]\end{locator} {\eva Shol chol Shol tchol chcThy otyky Shey yteol Shody}\\
\begin{locator}[P3, 12]\end{locator} {\eva ykee\textcolor{orange}{y} chor Sheey ySheol chor chol daiin chkaiin}\\


%%%%%%%
% 21V %
%%%%%%%
\clearpage
\subsection*{Folio 21, Verso}
\begin{locator}[P1, 1]\end{locator} {\eva toldShy chofchy qofShey ShcKhol odaiin Shey cKholy}\\
\begin{locator}[P1, 2]\end{locator} {\eva oeeesoy qokchy chody qotchy qokchy choty tchol daiin}\\
\begin{locator}[P1, 3]\end{locator} {\eva qotol keeees chotchy tcho choty chor qotol daiin dal}\\
\begin{locator}[P1, 4]\end{locator} {\eva Sho chodaiin choty chol daiin daiin chty chtol}\\
\begin{locator}[P1, 5]\end{locator} {\eva oSho deey cTho l Sho cThy daiin dait oky}\\
\begin{locator}[P1, 6]\end{locator} {\eva Sho tSho chotShol chol todaiin daiin}\\
\begin{locator}[P1, 7]\end{locator} {\eva ykcho lchol cholchaiin otchy s Sheaiin}\\
\begin{locator}[P1, 8]\end{locator} {\eva cho l kchochaiin}\\


%%%%%%%
% 22R %
%%%%%%%
\clearpage
\subsection*{Folio 22, Recto}
\begin{locator}[P1, 1]\end{locator} {\eva pol olShy fcholy Shol dpchy oty okoly daiin opchy s ocPhy}\\
\begin{locator}[P1, 2]\end{locator} {\eva ol oiin Shol o kor qokchol daiin otaiin cThor dain cKhydom}\\
\begin{locator}[P1, 3]\end{locator} {\eva qokol dykaiin okchy daiin cThol cTholo dar Shain}\\

\vspace{1em}
\noindent\begin{locator}[P2, 4]\end{locator} {\eva pchaiin ofchy daiin cFhy doroiin ypchol sy schor daiin}\\
\begin{locator}[P2, 5]\end{locator} {\eva ol daiin qokchy dar daiin chor oldor oky y choldchy}\\
\begin{locator}[P2, 6]\end{locator} {\eva y chokShchy cTheen}\\

\vspace{1em}
\noindent\begin{locator}[P3, 7]\end{locator} {\eva kchol Shol dSheor \textcolor{orange}{S}k\textcolor{orange}{o} chdoly ytaiin ol otchy cPhal}\\
\begin{locator}[P3, 8]\end{locator} {\eva dchor oty daiin cTholy qoky chotaiin chocThy doiiin dchor}\\
\begin{locator}[P3, 9]\end{locator} {\eva odaiin dain cThy cTheor oraiino}\\

\vspace{1em}
\noindent\begin{locator}[P4, 10]\end{locator} {\eva kchol chor daiin cThoiin dchor chey qokol dy opchol oldam}\\
\begin{locator}[P4, 11]\end{locator} {\eva doiin ycKhody qokchy oky otoldy yty dol or d\textcolor{orange}{ch}chy daiin}\\
\begin{locator}[P4, 12]\end{locator} {\eva odchaiin cThy okchy kchy dchol daiin ydaiin}\\
\begin{locator}[P4, 13]\end{locator} {\eva dchor dydain qocKhy yk\textcolor{orange}{o}lokain}\\


%%%%%%%
% 22V %
%%%%%%%
\clearpage
\subsection*{Folio 22, Verso}
\begin{locator}[P1, 1]\end{locator} {\eva pysaiinor ofchar oky tchy otdy sor Shy qod}\\
\begin{locator}[P1, 2]\end{locator} {\eva daiin ykaiin qotchy kchy ytchyd dShor ychy}\\
\begin{locator}[P1, 3]\end{locator} {\eva qoky kchorl otchy cThy otchyky ytchol otam}\\
\begin{locator}[P1, 4]\end{locator} {\eva otchaiin Shoty qoky saiin odaiin ytaiin}\\
\begin{locator}[P1, 5]\end{locator} {\eva dor ykcheor daiin}\\

\vspace{1em}
\noindent\begin{locator}[P2, 6]\end{locator} {\eva fShor Shy tchor otaiin}\\
\begin{locator}[P2, 7]\end{locator} {\eva ychor chor qokchol chory}\\
\begin{locator}[P2, 8]\end{locator} {\eva qotchy cThy qokol daiin dam}\\
\begin{locator}[P2, 9]\end{locator} {\eva okShor Shody chol t\textcolor{orange}{ch}ol otaiin daiin}\\
\begin{locator}[P2, 10]\end{locator} {\eva qokchy daiin s or ytal}\\
\begin{locator}[P2, 11]\end{locator} {\eva sokaiin oty dy}\\
\begin{locator}[P2, 12]\end{locator} {\eva ychky daiin cThy}\\
\begin{locator}[P2, 13]\end{locator} {\eva okchain chkoldy Shotoly}\\
\begin{locator}[P2, 14]\end{locator} {\eva qotchy olShly Shol daiin}\\
\begin{locator}[P2, 15]\end{locator} {\eva Sho cThy chocThy qokchy dory}\\
\begin{locator}[T1, 16]\end{locator} {\eva saiinchy daldalol}\\


%%%%%%%
% 23R %
%%%%%%%
\clearpage
\subsection*{Folio 23, Recto}
\begin{locator}[P1, 1]\end{locator} {\eva pydchdom chy fcholdy oty otchol Shy opy\textcolor{orange}{o}iin y yfchy daiin ololdy dal}\\
\begin{locator}[P1, 2]\end{locator} {\eva to ar chor daiin chkdain otchy lolchor daiin dam okchol daing}\\
\begin{locator}[P1, 3]\end{locator} {\eva dchar ykor ykaiin daiin cTho g}\\

\vspace{1em}
\noindent\begin{locator}[P2, 4]\end{locator} {\eva qokoldy okaiir ykaiil g qokeey ofchol dain yfchor olfchor otchald}\\
\begin{locator}[P2, 5]\end{locator} {\eva ychor qokchol ytom chol dair chol ar ol ol dol dain}\\

\vspace{1em}
\noindent\begin{locator}[P3, 6]\end{locator} {\eva tShol ykor qokaiin yky dar okol dchey daiidal dam ytcho ldals}\\
\begin{locator}[P3, 7]\end{locator} {\eva okar olchar Shaiin qokchol dar qokchol dairo r ol daiin alg}\\
\begin{locator}[P3, 8]\end{locator} {\eva qokShy char daiir qokaiin olol qoaiin ykchy s diil okchy}\\
\begin{locator}[P3, 9]\end{locator} {\eva okol ok Shy qokol dy d\textcolor{orange}{o}l dShe qokeees y oly daiin dal}\\
\begin{locator}[P3, 10]\end{locator} {\eva qok okaiin chkchy s yteair odal chal dy dar ykain}\\
\begin{locator}[P3, 11]\end{locator} {\eva ykyk\textcolor{orange}{o} dalory}\\


%%%%%%%
% 23V %
%%%%%%%
\clearpage
\subsection*{Folio 23, Verso}
\begin{locator}[P1, 1]\end{locator} {\eva podairol odaiiily Shoaldy opchol otol ol chcPhy qotchar s}\\
\begin{locator}[P1, 2]\end{locator} {\eva qot\textcolor{orange}{c}otor chy okcha\textcolor{orange}{l}dy qokchey dol otchol chal otch\textcolor{orange}{g}}\\
\begin{locator}[P1, 3]\end{locator} {\eva ycho okaiin okeol eees oldaiin okeeor daiin qotchol\textcolor{orange}{g}}\\
\begin{locator}[P1, 4]\end{locator} {\eva okchey dair Sholol oldal otchor \textcolor{orange}{d}ar}\\
\begin{locator}[P1, 5]\end{locator} {\eva dor chear Shor dol oaldary}\\

\vspace{1em}
\noindent\begin{locator}[P2, 6]\end{locator} {\eva tShol Shor ShkShy okol daiin otShor olsar}\\
\begin{locator}[P2, 7]\end{locator} {\eva otor oiin Sho Shol qokol daiin sol daiin ylg}\\
\begin{locator}[P2, 8]\end{locator} {\eva qokaiin dain qokor okal g dam chor olol dam}\\
\begin{locator}[P2, 9]\end{locator} {\eva otShy dal dar oldar ar or qoto l chees g}\\
\begin{locator}[P2, 10]\end{locator} {\eva dor chy kShol chol cThol otol oloiir}\\
\begin{locator}[P2, 11]\end{locator} {\eva y okaiin doroiin olols oiin ol cheen ols}\\
\begin{locator}[P2, 12]\end{locator} {\eva olaiior oloro eeeoly}\\


%%%%%%%
% 24R %
%%%%%%%
\clearpage
\subsection*{Folio 24, Recto}
\begin{locator}[P1, 1]\end{locator} {\eva porory chor opchar She cheol daiin or}\\
\begin{locator}[P1, 2]\end{locator} {\eva qotaiin char odain okaiIKhal oky}\\
\begin{locator}[P1, 3]\end{locator} {\eva y cThar cThal okol qotar cKhy}\\
\begin{locator}[P1, 4]\end{locator} {\eva or chcKhaly cThar eeor chees da}\\
\begin{locator}[P1, 5]\end{locator} {\eva qdar cho r chey cThy cTh\textcolor{orange}{e$\smile$}keom}\\
\begin{locator}[P1, 6]\end{locator} {\eva oeeeos eteor otal qocThol qoky}\\
\begin{locator}[P1, 7]\end{locator} {\eva q\textcolor{orange}{*}kar chtar s cheor cThol qodol}\\
\begin{locator}[P1, 8]\end{locator} {\eva ychor s om qoear daiin qokeol}\\
\begin{locator}[P1, 9]\end{locator} {\eva odaiin cKham qodai\textcolor{orange}{IK}hy dol dal}\\
\begin{locator}[P1, 10]\end{locator} {\eva q*or cFhar chor s am chotaiin dy}\\
\begin{locator}[P1, 11]\end{locator} {\eva sar cheoiees okeer cheor qokain}\\
\begin{locator}[P1, 12]\end{locator} {\eva qokchy qotchy tol tod cKhy}\\
\begin{locator}[P1, 13]\end{locator} {\eva oees ol s chey chcTh sar}\\
\begin{locator}[P1, 14]\end{locator} {\eva qor cheey qod char cThal}\\
\begin{locator}[P1, 15]\end{locator} {\eva ocKhoees oeees ol dal s}\\
\begin{locator}[P1, 16]\end{locator} {\eva Sham okeal dal dam dal}\\
\begin{locator}[P1, 17]\end{locator} {\eva sShey otam Sham cThom oky}\\
\begin{locator}[P1, 18]\end{locator} {\eva y cheol chol daiin chol s}\\
\begin{locator}[P1, 19]\end{locator} {\eva yol \textcolor{orange}{k}ol chol Shom otaIPhy}\\
\begin{locator}[T1, 20]\end{locator} {\eva samchorly}\\


%%%%%%%
% 24V %
%%%%%%%
\clearpage
\subsection*{Folio 24, Verso}
\begin{locator}[P1, 1]\end{locator} {\eva tchodar chocFhh\textcolor{orange}{y} opom Shod chcPhy opShody ocPhoraiin okokom}\\
\begin{locator}[P1, 2]\end{locator} {\eva ydals cKhor Shy cho dch\textcolor{orange}{o}r otol otaiir otchos okchom okcho}\\
\begin{locator}[P1, 3]\end{locator} {\eva ocThol odchees oesearees okam chcTh}\\
\begin{locator}[P1, 4]\end{locator} {\eva ydal Sh okol okaiin odaiin dlos}\\
\begin{locator}[P1, 5]\end{locator} {\eva oeor or\textcolor{orange}{o}iin tchar oro}\\

\vspace{1em}
\noindent\begin{locator}[P2, 6]\end{locator} {\eva tochol chor cfarasr}\\
\begin{locator}[P2, 7]\end{locator} {\eva ycheol daid dar olom}\\
\begin{locator}[P2, 8]\end{locator} {\eva kochky chcThy Shol sain}\\
\begin{locator}[P2, 9]\end{locator} {\eva ychol chol or chorom}\\
\begin{locator}[P2, 10]\end{locator} {\eva okoeo\textcolor{orange}{r} cKheod choy keol}\\
\begin{locator}[P2, 11]\end{locator} {\eva ydaiin cheor qodom okodaiin}\\
\begin{locator}[P2, 12]\end{locator} {\eva kSho foar cto Sho qokch ok\textcolor{orange}{$\smile$}okchal}\\
\begin{locator}[P2, 13]\end{locator} {\eva oeeey cheol chol odor Sho do otolodal}\\
\begin{locator}[P2, 14]\end{locator} {\eva doiir chee\textcolor{orange}{o}dam Sho Sho dy}\\
\begin{locator}[P2, 15]\end{locator} {\eva dchey kchod dchol ochdy}\\
\begin{locator}[P2, 16]\end{locator} {\eva otchol odaiim}\\


%%%%%%%
% 25R %
%%%%%%%
\clearpage
\subsection*{Folio 25, Recto}
\begin{locator}[P1, 1]\end{locator} {\eva fcholdy soShy daiin cky Shody daiin ocholdy cPholdy sy}\\
\begin{locator}[P1, 2]\end{locator} {\eva otor chor chrky chotchy Shair qod Sho chy kchy chkain}\\
\begin{locator}[P1, 3]\end{locator} {\eva qotchy qotShy cheesees Sheear s chain daiin chain dein}\\
\begin{locator}[P1, 4]\end{locator} {\eva dchcKhy ShocThy ytchey cThor s chan chaiin qotchain}\\
\begin{locator}[P1, 5]\end{locator} {\eva qotcheaiin dchain cThain daiin daiin cThain qotaiin}\\
\begin{locator}[P1, 6]\end{locator} {\eva okal chotaiin}\\
\begin{locator}[T1, 7]\end{locator} {\eva dair otaiir otosy}\\


%%%%%%%
% 25V %
%%%%%%%
\clearpage
\subsection*{Folio 25, Verso}
\begin{locator}[P1, 1]\end{locator} {\eva poeeaiin qoky Shy daiin qopchey otchey qofchor sos}\\
\begin{locator}[P1, 2]\end{locator} {\eva dchor cThor chor daiin s okeeaiin daiin cKhey daiin}\\
\begin{locator}[P1, 3]\end{locator} {\eva orcho kchor chol daiin ShcFhor daiin dShey daiity}\\
\begin{locator}[P1, 4]\end{locator} {\eva qok\textcolor{orange}{o}iin qokcho Shol daiin cKhear cKhol daiin chkear}\\
\begin{locator}[P1, 5]\end{locator} {\eva dar ch\textcolor{orange}{o}keey dShor dShey qochol dol cho daiin daiin}\\
\begin{locator}[P1, 6]\end{locator} {\eva qokcho r ochy qotchy qokoral cho\textcolor{orange}{*} chain deeaiir s}\\
\begin{locator}[P1, 7]\end{locator} {\eva o\textcolor{orange}{S}o chkey daiiol daiin ShcKh\textcolor{orange}{$\cancel{x}$} orchaiin}\\


%%%%%%%
% 26R %
%%%%%%%
\clearpage
\subsection*{Folio 26, Recto}
\begin{locator}[P1, 1]\end{locator} {\eva pSheoky odaiir qoy of\textcolor{orange}{Sh}od chypchey ypchedy ain chofochc\textcolor{orange}{P}hdy}\\
\begin{locator}[P1, 2]\end{locator} {\eva dchey \textcolor{orange}{d}aiin adeeody ykecThhy chedy ytedy dy checThedy lr}\\
\begin{locator}[P1, 3]\end{locator} {\eva oaiin ShcThy \textcolor{orange}{cTh}ed\textcolor{orange}{y} oloy ykeedy olchedy \textcolor{orange}{d}al y Sheey saiin s}\\
\begin{locator}[P1, 4]\end{locator} {\eva qokedy cheos ytedy qokedy ytedy chekedy daiin odam s aldy}\\
\begin{locator}[P1, 5]\end{locator} {\eva \textcolor{orange}{*}aiin Shedy \textcolor{orange}{ch}dy \textcolor{orange}{ch}dy schy daiin cThedy qokeedy qokedy cThhy}\\
\begin{locator}[P1, 6]\end{locator} {\eva rchedy qokedy}\\

\vspace{1em}
\noindent\begin{locator}[P2, 7]\end{locator} {\eva pcho qokedy dar Sheo ypchseds s aiin Shapchedyf\textcolor{orange}{ch}y dalchedy sa\textcolor{orange}{r}}\\
\begin{locator}[P2, 8]\end{locator} {\eva daiin Shedy qokeedy qoteedar s ok ol \textcolor{orange}{y}tedy qokchdy qokedy}\\
\begin{locator}[P2, 9]\end{locator} {\eva tcheo\textcolor{orange}{$\frown$}Shy \textcolor{orange}{dch}dy okedy chcKhy s dy dy ykeechy okeedy cheky}\\
\begin{locator}[P2, 10]\end{locator} {\eva Shese aiin Sheos cheody otal}\\


%%%%%%%
% 26V %
%%%%%%%
\clearpage
\subsection*{Folio 26, Verso}
\begin{locator}[P1, 1]\end{locator} {\eva pchedar qodary \textcolor{orange}{d}*iiin pcheety sair Shedy ypchedy ypchdy qopy Shdy}\\
\begin{locator}[P1, 2]\end{locator} {\eva saraiir chekedy qokedy otedy sar y etedy qokedy or a\textcolor{orange}{Sh}e alys chedy}\\
\begin{locator}[P1, 3]\end{locator} {\eva pchdar opar dar cheeol ofchdy otedy cKhdy odar chedy ytedy okchdy g}\\
\begin{locator}[P1, 4]\end{locator} {\eva ycKheody qokedy deey saldy okedor or \textcolor{orange}{ch}eos oraiin okeo chekaiin}\\
\begin{locator}[P1, 5]\end{locator} {\eva deeol \textcolor{orange}{ch}eody qoteedy qokody qotedy qotedy opchedy ofchy chs ar}\\
\begin{locator}[P1, 6]\end{locator} {\eva toeedy keody Shedy dar chedy sches or cheeky dar chey cheky ytchdy}\\
\begin{locator}[P1, 7]\end{locator} {\eva pchedy dar cheoet chy sair chees odaiiin chkeeey ykey Sheey}\\
\begin{locator}[P1, 8]\end{locator} {\eva t\textcolor{orange}{ch}edy okeeos cheeos ysaiin okcheey keody saiin cheeos qokes ory}\\
\begin{locator}[P1, 9]\end{locator} {\eva ySheey okeShy ShodypShey todydy}\\


%%%%%%%
% 27R %
%%%%%%%
\clearpage
\subsection*{Folio 27, Recto}
\begin{locator}[P1, 1]\end{locator} {\eva ksor Shey Shoteo chforaiin Shy Shod chary}\\
\begin{locator}[P1, 2]\end{locator} {\eva dy coain Shol dain dar Shokyd dchol cThey ds}\\
\begin{locator}[P1, 3]\end{locator} {\eva chol Shy keol chol \textcolor{orange}{c}hy Shol chy daiin chey dam}\\
\begin{locator}[P1, 4]\end{locator} {\eva qokey chor char chy dchy keey chos cThody}\\
\begin{locator}[P1, 5]\end{locator} {\eva dor chees cTho Shy cPhary daiin dair}\\
\begin{locator}[P1, 6]\end{locator} {\eva chy tchols chor chol chaiin Shy kchaldy}\\

\vspace{1em}
\noindent\begin{locator}[P2, 7]\end{locator} {\eva kchey chey kcheol pchy schey ly chals cham}\\
\begin{locator}[P2, 8]\end{locator} {\eva ytchy chy tchol dy tchey dain chol dal}\\
\begin{locator}[P2, 9]\end{locator} {\eva dchey keeod Sho\textcolor{orange}{$\frown$}tchey chol oty chy tol\textcolor{orange}{g}}\\
\begin{locator}[P2, 10]\end{locator} {\eva qotchey Shes s y chy tchey dy}\\
\begin{locator}[P2, 11]\end{locator} {\eva chol daiiin chees chos ctey dan}\\
\begin{locator}[P2, 12]\end{locator} {\eva dain cheokeey chey cThey otal}\\
\begin{locator}[T1, 13]\end{locator} {\eva okchodeey}\\


%%%%%%%
% 27V %
%%%%%%%
\clearpage
\subsection*{Folio 27, Verso}
\begin{locator}[P1, 1]\end{locator} {\eva fochof chof cho Sho soly Shol ytchor ofchory kchorchor}\\
\begin{locator}[P1, 2]\end{locator} {\eva dchy chkar otchy Shy Shy dchy dShy kchy cheo da\textcolor{orange}{i}dy dchy}\\
\begin{locator}[P1, 3]\end{locator} {\eva kchey kchy dchokchy dSho dc\textcolor{orange}{a}r chodchy etcheody Shld}\\
\begin{locator}[P1, 4]\end{locator} {\eva okcho chy kcheed chl chol kod o okSho do\textcolor{orange}{ch}ees\textcolor{orange}{g}}\\
\begin{locator}[P1, 5]\end{locator} {\eva qoky Shkeeo schodar Shkol chotchy cThodol}\\
\begin{locator}[P1, 6]\end{locator} {\eva dSho kchrrr okeedy dchschy sotchdy}\\
\begin{locator}[P1, 7]\end{locator} {\eva Sho Sho ykcho Shdy dchd chschs\textcolor{orange}{y} otchdy}\\
\begin{locator}[P1, 8]\end{locator} {\eva okShes okcho kShy}\\


%%%%%%%
% 28R %
%%%%%%%
\clearpage
\subsection*{Folio 28, Recto}
\begin{locator}[P1, 1]\end{locator} {\eva pchodar Shod chocPhy opchol daiin otchol chyqo ldy}\\
\begin{locator}[P1, 2]\end{locator} {\eva otchor otchor cThol cty cTheol dy dchar chakod dly}\\
\begin{locator}[P1, 3]\end{locator} {\eva qotchaiin Shor cThol cThol Shor chotchy tchodar}\\
\begin{locator}[P1, 4]\end{locator} {\eva choty chtol otol chotchy cThol otol choky qoty}\\
\begin{locator}[P1, 5]\end{locator} {\eva okSho otor chy kchoror chodaiin Sho cThody okoy}\\
\begin{locator}[P1, 6]\end{locator} {\eva qokchol qodaiin otcholchy daiin cho qokol okam}\\
\begin{locator}[P1, 7]\end{locator} {\eva Sho otor ShocKhy ShocThy otoldy dShor dol dar}\\
\begin{locator}[P1, 8]\end{locator} {\eva oschotShl daiin okchey kol daiin Shol dSho otaiin}\\
\begin{locator}[P1, 9]\end{locator} {\eva ytchol deey yteol deaiin}\\


%%%%%%%
% 28V %
%%%%%%%
\clearpage
\subsection*{Folio 28, Verso}
\begin{locator}[P1, 1]\end{locator} {\eva kShol qooiiin Shor p\textcolor{orange}{S}hoiiin Shepchy qoty dy Shory}\\
\begin{locator}[P1, 2]\end{locator} {\eva ykcholy qoty chy dy qokchol chor tchy qokchody cheor o}\\
\begin{locator}[P1, 3]\end{locator} {\eva chor chol chy choiin}\\

\vspace{1em}
\noindent\begin{locator}[P2, 4]\end{locator} {\eva tShoiin cheor chor o chty qotol Sheol Shor daiin qoty}\\
\begin{locator}[P2, 5]\end{locator} {\eva otol chol daiin chkaiin Shoiin qotchey qotShey daiiin}\\
\begin{locator}[T1, 6]\end{locator} {\eva daiin chkaiin}\\

\vspace{1em}
\noindent\begin{locator}[P3, 7]\end{locator} {\eva pchol \textcolor{orange}{o}iir chol tSho daiin Sho tco chy chtShy dai\textcolor{orange}{r} am}\\
\begin{locator}[P3, 8]\end{locator} {\eva okain chan chain cThor dain yk chy daiin cThol}\\
\begin{locator}[P3, 9]\end{locator} {\eva sor chear chl s \textcolor{orange}{S}holy dar}\\


%%%%%%%
% 29R %
%%%%%%%
\clearpage
\subsection*{Folio 29, Recto}
\begin{locator}[P1, 1]\end{locator} {\eva po\textcolor{orange}{s}aiin She aiin chep oty chy qotchy qoty cheecThy}\\
\begin{locator}[P1, 2]\end{locator} {\eva dShe ykchy choty oky chol chchoty choky chy ty dy}\\
\begin{locator}[P1, 3]\end{locator} {\eva qokchy qoty kchaiin Shear cThor dchor choly}\\
\begin{locator}[P1, 4]\end{locator} {\eva chocThy qoos chos}\\

\vspace{1em}
\noindent\begin{locator}[P2, 5]\end{locator} {\eva kcheol cheor Sheos Sheey teey do\textcolor{orange}{e} Shos oc\textcolor{orange}{T}y}\\
\begin{locator}[P2, 6]\end{locator} {\eva chokShy ShocThy Shor Shor daiin qokaiiin}\\
\begin{locator}[P2, 7]\end{locator} {\eva qokchy chol Shokchy qokaiin choe\textcolor{orange}{k}y choty kaiin}\\
\begin{locator}[P2, 8]\end{locator} {\eva Shor chor Sho Sheky Shy qoty kody daiin cThy}\\
\begin{locator}[P2, 9]\end{locator} {\eva qokShe qor chey kor cheod ychom}\\


%%%%%%%
% 29V %
%%%%%%%
\clearpage
\subsection*{Folio 29, Verso}
\begin{locator}[P1, 1]\end{locator} {\eva kooiin Shor chetchy olals Shytchy cTh y Shy cho Shy daiin}\\
\begin{locator}[P1, 2]\end{locator} {\eva qotcheaiin schol chol cThy chey cThold ytchor dary}\\
\begin{locator}[P1, 3]\end{locator} {\eva chol chol kor Shey odaiin qotchy taiin \textcolor{orange}{d} She otey sy}\\
\begin{locator}[P1, 4]\end{locator} {\eva ySho otShy okaiin cThy oltchy c\textcolor{orange}{T}os Shot Sho okaiin}\\

\vspace{1em}
\noindent\begin{locator}[P2, 5]\end{locator} {\eva tochon chain Shan Shoty chShy Shy}\\
\begin{locator}[P2, 6]\end{locator} {\eva dchor chol chaiin Shaiin cThod dar chs Shody}\\
\begin{locator}[P2, 7]\end{locator} {\eva qotcho kchor daiin ykaiin dy Shdy cho cThy Sheky}\\
\begin{locator}[P2, 8]\end{locator} {\eva otol Sho\textcolor{orange}{$\frown$}tchos cholody dain chcTh chy cThody dol}\\
\begin{locator}[P2, 9]\end{locator} {\eva Sho chokor chor chy ydaiin cho yk\textcolor{orange}{ee}n oaiin}\\

\vspace{1em}
\noindent\begin{locator}[P3, 10]\end{locator} {\eva qokoiin okaiin cho cTh chyd chykar}\\
\begin{locator}[P3, 11]\end{locator} {\eva daiin chteor ocThy dar chyt otody}\\
\begin{locator}[P3, 12]\end{locator} {\eva qekeochor otchey \textcolor{orange}{d} y}\\


%%%%%%%
% 30R %
%%%%%%%
\clearpage
\subsection*{Folio 30, Recto}
\begin{locator}[P1, 1]\end{locator} {\eva okchesy chey Shorchey fcheody Shey tchy cher d o Shey}\\
\begin{locator}[P1, 2]\end{locator} {\eva ychody chey chkeey kSho keeor cheor Shey She keeodol sy}\\
\begin{locator}[P1, 3]\end{locator} {\eva chodaiin chory chey doiin sa\textcolor{orange}{$\cancel{x}$}in chorain cheey keem}\\
\begin{locator}[P1, 4]\end{locator} {\eva qokechy qoko qopchar soin chan qotchaiin}\\
\begin{locator}[P1, 5]\end{locator} {\eva choko qokochy deeey dkeeor cheoldain chory}\\
\begin{locator}[P1, 6]\end{locator} {\eva dchorol Sho chor cho ro raiin dor chseeor dy}\\
\begin{locator}[P1, 7]\end{locator} {\eva qor chain cThorchy}\\

\vspace{1em}
\noindent\begin{locator}[P2, 8]\end{locator} {\eva opchol ol chesey scheo rchey okeal dcheo rchey}\\
\begin{locator}[P2, 9]\end{locator} {\eva dchey qochar chol keeaiin chcThey chor cheky}\\
\begin{locator}[P2, 10]\end{locator} {\eva chokchaiin dchor chkchean qotchy chcTho rchody}\\
\begin{locator}[P2, 11]\end{locator} {\eva qotchor cheor chey cheor chey so\textcolor{orange}{ee}n ydey sor daiin}\\
\begin{locator}[P2, 12]\end{locator} {\eva dcheey cKhey cKhey daiin chkeaiin dar chor ychdain}\\
\begin{locator}[P2, 13]\end{locator} {\eva chodaiin chtchey chear Shy keey}\\


%%%%%%%
% 30V %
%%%%%%%
\clearpage
\subsection*{Folio 30, Verso}
\begin{locator}[P1, 1]\end{locator} {\eva cThscThain Shosaiin chocThey Sho chepchy Shor Sheaiin}\\
\begin{locator}[P1, 2]\end{locator} {\eva qotchor chy She kchor chory keor ol chy daiin cTholdy}\\
\begin{locator}[P1, 3]\end{locator} {\eva chotchol daiin cThol doiin daiin chokeor dal chtoIThy}\\
\begin{locator}[P1, 4]\end{locator} {\eva otchey daiin chor checKhy qotchod daiin}\\
\begin{locator}[P1, 5]\end{locator} {\eva dain Shoty chkol ytor cheoldy}\\
\begin{locator}[P1, 6]\end{locator} {\eva qokeeor chokol chokeody dairy}\\
\begin{locator}[P1, 7]\end{locator} {\eva chokchy Sho kcho tcho cTho sos}\\
\begin{locator}[P1, 8]\end{locator} {\eva oyShy cheotol cPhoaiin cPhey}\\
\begin{locator}[P1, 9]\end{locator} {\eva qotchor chor Sheey cheol sos}\\
\begin{locator}[P1, 10]\end{locator} {\eva Sho Sheoldy otcheor daiin}\\
\begin{locator}[P1, 11]\end{locator} {\eva sol chokcheey daiin kch\textcolor{orange}{o}dy}\\


%%%%%%%
% 31R %
%%%%%%%
\clearpage
\subsection*{Folio 31, Recto}
\begin{locator}[P1, 1]\end{locator} {\eva keedey qofchedy Shees aiin qokeefcy chepakeo}\\
\begin{locator}[P1, 2]\end{locator} {\eva dcheey daiin okeedy qokees s aiin Sh\textcolor{orange}{eke}ey qef}\\
\begin{locator}[P1, 3]\end{locator} {\eva qokeey chey daiin qokeey rair chekey saiin d\textcolor{orange}{ey}}\\
\begin{locator}[P1, 4]\end{locator} {\eva qotar chy dar l\textcolor{orange}{o$\frown$}r ar cheey keeol ch\textcolor{orange}{e}d\textcolor{orange}{y} qokey}\\
\begin{locator}[P1, 5]\end{locator} {\eva daiin Sheoody qokeor}\\

\vspace{1em}
\noindent\begin{locator}[P2, 6]\end{locator} {\eva tShokeody qokedy qotedy chepar chedy r aiin dal}\\
\begin{locator}[P2, 7]\end{locator} {\eva ykeedar saiin checKhey Sheol qokedy ykeedy chedy ldy}\\
\begin{locator}[P2, 8]\end{locator} {\eva Shedy qokedy cheol cheod qokeody cheol checThy}\\
\begin{locator}[P2, 9]\end{locator} {\eva daiir Sheeo ShcThey okeol}\\

\vspace{1em}
\noindent\begin{locator}[P3, 10]\end{locator} {\eva tol Shso okedy okedy qokedy qokeedy dar ShedShey}\\
\begin{locator}[P3, 11]\end{locator} {\eva olSheol qokchy dal chey deey kchy keey okaii\textcolor{orange}{i}n ykeey}\\
\begin{locator}[P3, 12]\end{locator} {\eva qoaiin yches okedy Sheey chedaiin dar}\\
\begin{locator}[P3, 13]\end{locator} {\eva olSheor qoekedy okedy ykedy dals}\\
\begin{locator}[P3, 14]\end{locator} {\eva saii\textcolor{orange}{s} or chedy daiin okeedy}\\
\begin{locator}[P3, 15]\end{locator} {\eva yShedair Sheol\textcolor{orange}{$\frown$}cheky okedam}\\
\begin{locator}[T1, 16]\end{locator} {\eva oteor aiicThy}\\


%%%%%%%
% 31V %
%%%%%%%
\clearpage
\subsection*{Folio 31, Verso}
\begin{locator}[P1, 1]\end{locator} {\eva podair Sheedy otedy oteedy qotolcheo s arar oteey dkarar als}\\
\begin{locator}[P1, 2]\end{locator} {\eva ytchos oc\textcolor{orange}{K}hey okeeos cheody okeey k\textcolor{orange}{ee}ody daiin cheody keedy}\\
\begin{locator}[P1, 3]\end{locator} {\eva ykeeo daiin Shed\textcolor{orange}{o}l okedy okey keody okchey sair ok\textcolor{orange}{ch}s o lkedy}\\
\begin{locator}[P1, 4]\end{locator} {\eva dair cThedy qokedy okeody chedar oked al ocKhedy okeedy ota\textcolor{orange}{d}}\\
\begin{locator}[P1, 5]\end{locator} {\eva ykeos cheeoy ar aiin}\\

\vspace{1em}
\noindent\begin{locator}[P2, 6]\end{locator} {\eva pcheeody qop chedal \textcolor{orange}{ch}ecFhy chefchol or cheef alaiin opal Sheo otar}\\
\begin{locator}[P2, 7]\end{locator} {\eva ol\textcolor{orange}{k}eedam ches chol keeol checKhy okeol okal oky cheokar okor ary}\\
\begin{locator}[P2, 8]\end{locator} {\eva yk aiin chee\textcolor{orange}{$\frown$}kSheey ychek\textcolor{orange}{$\frown$}eeor cheor or checThy okechoked lchey okam}\\
\begin{locator}[P2, 9]\end{locator} {\eva ytecheol Sheoeky okeos aiin a\textcolor{orange}{I}Thedy chkaiin chetchey cTheey okear}\\
\begin{locator}[P2, 10]\end{locator} {\eva dar \textcolor{orange}{ch}oar al kar oeeeos cheos aiin o \textcolor{orange}{cKh}ey okeo kor o\textcolor{orange}{k}eol ain}\\
\begin{locator}[P2, 11]\end{locator} {\eva saiin ar cKheos chedy okeey qoear oraiin cheom}\\


%%%%%%%
% 32R %
%%%%%%%
\clearpage
\subsection*{Folio 32, Recto}
\begin{locator}[P1, 1]\end{locator} {\eva fchoiin Shykeody daiiody dain Sho tchy oty qopy}\\
\begin{locator}[P1, 2]\end{locator} {\eva okor okchor Sheor cKhy dal dShodar qotchol}\\
\begin{locator}[P1, 3]\end{locator} {\eva qokchor chor cThol chol dol dcheodain daiin}\\
\begin{locator}[P1, 4]\end{locator} {\eva schor dShor ytSho dain daiin choddal}\\
\begin{locator}[P1, 5]\end{locator} {\eva qotchy qokchy daiin}\\

\vspace{1em}
\noindent\begin{locator}[P2, 6]\end{locator} {\eva fcho tchey chedy}\\
\begin{locator}[P2, 7]\end{locator} {\eva otol dol ol dair}\\
\begin{locator}[P2, 8]\end{locator} {\eva qoar daiin dam}\\
\begin{locator}[P2, 9]\end{locator} {\eva dytchor dary}\\
\begin{locator}[P2, 10]\end{locator} {\eva dchor cTho daiin}\\
\begin{locator}[P2, 11]\end{locator} {\eva Shos chcKhol n}\\
\begin{locator}[P2, 12]\end{locator} {\eva Shodaiin ctol}\\
\begin{locator}[P2, 13]\end{locator} {\eva otcho\textcolor{orange}{$\frown$}r\textcolor{orange}{$\frown$}ol dain daiin cThy}\\
\begin{locator}[P2, 14]\end{locator} {\eva sche\textcolor{orange}{y} qot Shey daiin cThs}\\
\begin{locator}[P2, 15]\end{locator} {\eva qokchy qotol doiir ol}\\
\begin{locator}[P2, 16]\end{locator} {\eva otchol chey soty}\\
\begin{locator}[P2, 17]\end{locator} {\eva chokeol dchoty}\\
\begin{locator}[P2, 18]\end{locator} {\eva doiin Sho Shy}\\
\begin{locator}[P2, 19]\end{locator} {\eva dol dchol dan}\\


%%%%%%%
% 32V %
%%%%%%%
\clearpage
\subsection*{Folio 32, Verso}
\begin{locator}[P1, 1]\end{locator} {\eva kcheodaiin chol kechy qotaiin daiioam o chofchody}\\
\begin{locator}[P1, 2]\end{locator} {\eva daiin odar chy daiin chey tcho ldy dain teor}\\
\begin{locator}[P1, 3]\end{locator} {\eva Shor daiin chcKhy dSho dain daiin s Shokey ka}\\
\begin{locator}[P1, 4]\end{locator} {\eva or chr chor dor chaiin qotcho r chy dcho daiin}\\
\begin{locator}[P1, 5]\end{locator} {\eva chokchy daiin Shy chor qo kaiin dain dchol dorg}\\
\begin{locator}[P1, 6]\end{locator} {\eva okchan chol\textcolor{orange}{$\frown$}Shal dchcThy}\\

\vspace{1em}
\noindent\begin{locator}[P2, 7]\end{locator} {\eva kSho cPhor She Sheaiin otShchor dai\textcolor{orange}{i} ShcKhy s odan}\\
\begin{locator}[P2, 8]\end{locator} {\eva otchol daiin daiin cTho daiin qotaiin otchy d Shar}\\
\begin{locator}[P2, 9]\end{locator} {\eva qotchy cFhy skey chocThy daiin cThaiin daiin}\\
\begin{locator}[P2, 10]\end{locator} {\eva Sho keol chor chol daiin cPhol cThol da ar}\\
\begin{locator}[P2, 11]\end{locator} {\eva ol Sho chy}\\


%%%%%%%
% 33R %
%%%%%%%
\clearpage
\subsection*{Folio 33, Recto}
\begin{locator}[P1, 1]\end{locator} {\eva tShdar Shdor Shepchdy cPhody yfoldy qofocKhdy otchedy lfchdy daiin}\\
\begin{locator}[P1, 1]\end{locator} {\eva ytchedy qokar cheky okaldy qokaldy otor oldor qot\textcolor{orange}{a}r otar otardam}\\
\begin{locator}[P1, 1]\end{locator} {\eva loiin y cheky qokedy Shedy chdal chechdaiin qokchy ody chekeedy ykam}\\
\begin{locator}[P1, 1]\end{locator} {\eva taiin chekey or alaiiin saiin okaiin dar cheedy chkeey far aiin s}\\
\begin{locator}[P1, 1]\end{locator} {\eva pair or aiin otaiin ol kor aiin okal otal chdal Shekal q\textcolor{orange}{o}kar ota\textcolor{orange}{g}}\\
\begin{locator}[P1, 1]\end{locator} {\eva yteey Shody \textcolor{orange}{k}chedydy chekar okaiin okaiin daiin okal}\\
\begin{locator}[P1, 1]\end{locator} {\eva dar chos aiin or aiin cheekaiin okain ar okeeey}\\


%%%%%%%
% 33V %
%%%%%%%
\clearpage
\subsection*{Folio 33, Verso}
\begin{locator}[P1, 1]\end{locator} {\eva tar ar daiin ydain cThey dol\textcolor{orange}{s} Sheky ar aiin cs}\\
\begin{locator}[P1, 2]\end{locator} {\eva kchdy dam dy oky otal dain chdy ytam otam cham}\\
\begin{locator}[P1, 3]\end{locator} {\eva dar chcKhy dy dyky cKhdy oky dam okardy komdy}\\
\begin{locator}[P1, 4]\end{locator} {\eva tokar Shdy dal qok\textcolor{orange}{a}r Shd otody chedy ykedy dodl dain}\\
\begin{locator}[P1, 5]\end{locator} {\eva tchdy chody okaiin chcKhy dor arl cThy dy ty dy ykar cheky dy}\\
\begin{locator}[P1, 6]\end{locator} {\eva ycheo dor olaiin okar chdy chdy oldy okar chdy}\\

\vspace{1em}
\noindent\begin{locator}[P2, 7]\end{locator} {\eva tShdy Shefchdy ShcKhdy oltedy daiin oky cheol orain chdyShdy porar}\\
\begin{locator}[P2, 8]\end{locator} {\eva darar Sheey keedy okchy okar okedy chy daiin dy dy dar aiin okary}\\
\begin{locator}[P2, 9]\end{locator} {\eva sar or aiin chor or Shkair Shol or chcKhy ar aiin okain dal dy}\\
\begin{locator}[P2, 10]\end{locator} {\eva lchoar or chey lodaiin oor okeedy okaly}\\
\begin{locator}[P2, 11]\end{locator} {\eva tar al keey oram}\\


%%%%%%%
% 34R %
%%%%%%%
\clearpage
\subsection*{Folio 34, Recto}
\begin{locator}[P1, 1]\end{locator} {\eva pcheoepchy olar yl yfody okedody Shod ololdy dar ytey}\\
\begin{locator}[P1, 2]\end{locator} {\eva ytar Sheody olam ocThed\textcolor{orange}{y} otedy chdain oltey kchy ty}\\
\begin{locator}[P1, 3]\end{locator} {\eva qotedy chyty chdaly dar chd otedy qotol okedy dody ody kam}\\
\begin{locator}[P1, 4]\end{locator} {\eva ytedy daiin chey aiin Shy chcKhy} \textcolor{orange}{{\P}}\\

\vspace{1em}
\noindent\begin{locator}[P2, 5]\end{locator} {\eva qoteedy Shedy Shedy ol okes ar \textcolor{orange}{oltchedy otedy dam checThy}}\\
\begin{locator}[P2, 6]\end{locator} {\eva ote\textcolor{orange}{d}l chekey chetey oll chesy \textcolor{orange}{ykeedl chedy otey okaiin}}\\
\begin{locator}[P2, 7]\end{locator} {\eva qokedy dal chdy olchy ykch\textcolor{orange}{$\cancel{x}$}g \textcolor{orange}{daiin cheky fas aiir amg}}\\
\begin{locator}[P2, 8]\end{locator} {\eva Shotchy qoky olk checKhy \textcolor{orange}{ykedy qokchdy s ar oldam}}\\
\begin{locator}[P2, 9]\end{locator} {\eva ykchdy qod ar chc\textcolor{orange}{T} aIKhy \textcolor{orange}{ysair air chodar tam}}\\
\begin{locator}[T1, 10]\end{locator} {\eva \textcolor{orange}{orlchey todaly okaiin dardy}}\\

\vspace{1em}
\noindent\begin{locator}[P3, 11]\end{locator} {\eva tcheo olchcKhy oly otchdy chefalas}\\
\begin{locator}[P3, 12]\end{locator} {\eva qokeey Sh kaldaiin cheky cThdaly otchedy chty saiin}\\
\begin{locator}[P3, 13]\end{locator} {\eva char aiin okor ar tol qokar chcKhy chdal ked qokar ar daiin dam}\\
\begin{locator}[P3, 14]\end{locator} {\eva ykeo lor ochey oly okaly kechdy qokchdy chor ar aiiin daly}\\
\begin{locator}[P3, 15]\end{locator} {\eva or ar y kar ol al oky chody qokal chedy chcThedy cheky daram}\\
\begin{locator}[P3, 16]\end{locator} {\eva sair chekar cheky Shek cholchedy qokedy yk cheolchcThy}\\


%%%%%%%
% 34V %
%%%%%%%
\clearpage
\subsection*{Folio 34, Verso}
\begin{locator}[P1, 1]\end{locator} {\eva kschdy chdy chefchy Shdy qopchdy Shdydy chdalchdy ypchdy chcThdy spaiin}\\
\begin{locator}[P1, 2]\end{locator} {\eva tol qokchy dychedy okchy chcKhdy chdaiin cKhy loees ykar aiin oldam}\\
\begin{locator}[P1, 3]\end{locator} {\eva ytal \textcolor{orange}{Sh}or chdal olchdy char or ol kedaiiin chcThy okchdy chcKhy da\textcolor{orange}{r}am}\\
\begin{locator}[P1, 4]\end{locator} {\eva tchdaiin chekal Shedy qokedar chdaiin oldar qoldar chedy daiin otam}\\
\begin{locator}[P1, 5]\end{locator} {\eva lShaiir orair Shedy chechey dykey kair chedy qokar chekaly cholky}\\

\vspace{1em}
\noindent\begin{locator}[P2, 6]\end{locator} {\eva pchedar \textcolor{orange}{Sh}ear qokchdy qokees cheol ypchdaiin chedy lr ar chedain}\\
\begin{locator}[P2, 7]\end{locator} {\eva olchdaiin chedy chey keedy chy kedy dy qokedy okey sair chkain otain}\\
\begin{locator}[P2, 8]\end{locator} {\eva ySheos otar \textcolor{orange}{o}r choraiin cheky olchdaiin or oldar chdar okam}\\
\begin{locator}[P2, 9]\end{locator} {\eva lSheody cPhy qokeey keedy kchdy chedy qokedy chdy kal Shs oldaiin}\\
\begin{locator}[P2, 10]\end{locator} {\eva daiin chdy tedy kchdy okeedy checKhy chdy kain cheor or okedy okam}\\
\begin{locator}[P2, 11]\end{locator} {\eva yShos ody s aiin okoy okal Shedy}\\


%%%%%%%
% 35R %
%%%%%%%
\clearpage
\subsection*{Folio 35, Recto}
\begin{locator}[P1, 1]\end{locator} {\eva cThoo r choly cThy choty char dy}\\
\begin{locator}[P1, 2]\end{locator} {\eva qokeeaiin chokaiin qotchy daiin}\\
\begin{locator}[P1, 3]\end{locator} {\eva dchaiin cThey qotchey taiin cThory}\\
\begin{locator}[P1, 4]\end{locator} {\eva qotchy Shetchy cKhol cheey daiin dainl}\\
\begin{locator}[P1, 5]\end{locator} {\eva otchor Sho tcheey scheey daiin dainor}\\
\begin{locator}[P1, 6]\end{locator} {\eva schaiin char chan daiin}\\
\begin{locator}[P1, 7]\end{locator} {\eva Shosaiin tchor choky}\\
\begin{locator}[P1, 8]\end{locator} {\eva qo\textcolor{orange}{kee}y ky kaiin daiin}\\

\vspace{1em}
\noindent\begin{locator}[P2, 9]\end{locator} {\eva paiin chear aiin chear Shor\textcolor{orange}{$\frown$}chaiin \textcolor{orange}{$\cancel{x}\cancel{x}$}}\\
\begin{locator}[P2, 10]\end{locator} {\eva o aiin chaiin cKhy r chl s chochy daiin}\\
\begin{locator}[P2, 11]\end{locator} {\eva Shcheaiin chol cThaiin lchaiin lchal dal dair aldy n}\\
\begin{locator}[P2, 12]\end{locator} {\eva olor chy chaiin chy taiin kchey koldy chetchaiin}\\
\begin{locator}[P2, 13]\end{locator} {\eva qokoiin chaiin qokchaiin lShy lodaly oteey taiin}\\
\begin{locator}[P2, 14]\end{locator} {\eva cThol chol aiin qotchy otchor daiin Shol qotaiin}\\
\begin{locator}[P2, 15]\end{locator} {\eva ochor s chiin daiin ytain}\\


%%%%%%%
% 35V %
%%%%%%%
\clearpage
\subsection*{Folio 35, Verso}
\begin{locator}[P1, 1]\end{locator} {\eva par chor chocThy roaiin ar}\\
\begin{locator}[P1, 2]\end{locator} {\eva qotchy otchey kchor yty}\\
\begin{locator}[P1, 3]\end{locator} {\eva dchor choty chyty daiin}\\
\begin{locator}[P1, 4]\end{locator} {\eva ytchy qotchy dchy cThy}\\
\begin{locator}[P1, 5]\end{locator} {\eva dchokchy chocThy chcKhor}\\
\begin{locator}[P1, 6]\end{locator} {\eva Shol tcheey chkchee\textcolor{orange}{e}n chcThaiin}\\
\begin{locator}[P1, 7]\end{locator} {\eva tchotchor Shol Sho co kol daiin}\\
\begin{locator}[P1, 8]\end{locator} {\eva oaiin tchor cho chotchy dchol d}\\
\begin{locator}[P1, 9]\end{locator} {\eva qokchaiin cho kSho l choiin}\\
\begin{locator}[P1, 10]\end{locator} {\eva okcheey chosar Shory}\\
\begin{locator}[P1, 11]\end{locator} {\eva qotcheeaiin chodaiin}\\
\begin{locator}[P1, 12]\end{locator} {\eva dchaiin daiin daiin dal s}\\
\begin{locator}[P1, 13]\end{locator} {\eva ol char od ar chear}\\
\begin{locator}[P1, 14]\end{locator} {\eva tcheain Shy tar dain}\\
\begin{locator}[P1, 15]\end{locator} {\eva ykol cheol okchy tch}\\
\begin{locator}[P1, 16]\end{locator} {\eva ydaiin okeey daiin}\\
\begin{locator}[P1, 17]\end{locator} {\eva daiin dain chkaly choly}\\
\begin{locator}[P1, 18]\end{locator} {\eva daiin qokeeen chokeeo r}\\
\begin{locator}[P1, 19]\end{locator} {\eva schokey ykeol chol daiin}\\
\begin{locator}[P1, 20]\end{locator} {\eva sodaiin Shy dchy cKhy dan}\\
\begin{locator}[P1, 21]\end{locator} {\eva doiin chor chor}\\


%%%%%%%
% 36R %
%%%%%%%
\clearpage
\subsection*{Folio 36, Recto}
\begin{locator}[P1, 1]\end{locator} {\eva pchafdan qorain chcFhal soiin cPhor Shaiin cThy dair}\\
\begin{locator}[P1, 2]\end{locator} {\eva oral Shor ytaiin qotaiin qooldy chty chol dy tor}\\
\begin{locator}[P1, 3]\end{locator} {\eva qotchy tchy daiin \textcolor{orange}{d}aiin dolsain}\\

\vspace{1em}
\noindent\begin{locator}[P2, 4]\end{locator} {\eva podaiir cPhy qoypchol \textcolor{orange}{s}om \textcolor{orange}{S}y chy dchy fchom dar}\\
\begin{locator}[P2, 5]\end{locator} {\eva daiin qor chol cTholy s o r chy sy chytaroiin}\\
\begin{locator}[P2, 6]\end{locator} {\eva okaiin cThor ykaiiin s dain an dan}\\
\begin{locator}[P2, 7]\end{locator} {\eva qotol cThol okol dy okchy ytorory sold}\\
\begin{locator}[P2, 8]\end{locator} {\eva ytchor cThol chaiin yd}\\
\begin{locator}[P2, 9]\end{locator} {\eva ytodaly daiin otaro}\\


%%%%%%%
% 36V %
%%%%%%%
\clearpage
\subsection*{Folio 36, Verso}
\begin{locator}[P1, 1]\end{locator} {\eva pcharasy qoforom Shoaiin tchey chcKhhy otaiin cPhar daiin}\\
\begin{locator}[P1, 2]\end{locator} {\eva qotar chol daiin otaiin qotor ytar ochor chety cKhor dom}\\
\begin{locator}[P1, 3]\end{locator} {\eva dchytchy ytor s otaiin qopchor otar otchaiin s}\\
\begin{locator}[P1, 4]\end{locator} {\eva qotchor yky ty dy daiin cThor}\\

\vspace{1em}
\noindent\begin{locator}[P2, 5]\end{locator} {\eva tchor cKhoiin daiin cPhchar}\\
\begin{locator}[P2, 6]\end{locator} {\eva daiin cThor daiin dal dys}\\
\begin{locator}[P2, 7]\end{locator} {\eva qoky keol okchor os cho Shan}\\
\begin{locator}[P2, 8]\end{locator} {\eva ykShy ytchy dol yt\textcolor{orange}{o}dy yky}\\
\begin{locator}[P2, 9]\end{locator} {\eva okaiin ykcholqod chory}\\
\begin{locator}[P2, 10]\end{locator} {\eva ykchotchy daiin daiild}\\
\begin{locator}[P2, 11]\end{locator} {\eva oty chcThy ytoryd}\\
\begin{locator}[P2, 12]\end{locator} {\eva ytol kchy oty chd}\\
\begin{locator}[P2, 13]\end{locator} {\eva oky She cThol oty}\\
\begin{locator}[P2, 14]\end{locator} {\eva soiin chtain}\\


%%%%%%%
% 37R %
%%%%%%%
\clearpage
\subsection*{Folio 37, Recto}
\begin{locator}[P1, 1]\end{locator} {\eva tocPhol Shaiin qotor ofchor oty chory daiin otod or}\\
\begin{locator}[P1, 2]\end{locator} {\eva ykoiin cThor okaiin qotchy ytody qokaiin Sho ytaiin}\\
\begin{locator}[P1, 3]\end{locator} {\eva qoShy qokaiin cThol dy cKhor oky dy}\\

\vspace{1em}
\noindent\begin{locator}[P2, 4]\end{locator} {\eva pchotchy daiin cFhol dar chol daiin yd}\\
\begin{locator}[P2, 5]\end{locator} {\eva yky qokchy qotchor chkol otoly}\\
\begin{locator}[P2, 6]\end{locator} {\eva Shor Shol qokchy qotomody dchol daiin}\\
\begin{locator}[P2, 7]\end{locator} {\eva sor chey kor qokor daiin}\\

\vspace{1em}
\noindent\begin{locator}[P3, 8]\end{locator} {\eva koiin chorody qokaiin Shory otal Shor Sheor dar}\\
\begin{locator}[P3, 9]\end{locator} {\eva ykchody qotchy ykaiin chy qotor\textcolor{orange}{$\frown$}dy otcho lol daiin}\\
\begin{locator}[P3, 10]\end{locator} {\eva qoto okchochor dchor chy \textcolor{orange}{Shy} daiin ychey kol daiir}\\
\begin{locator}[P3, 11]\end{locator} {\eva okchor daiin cKhy dain d\textcolor{orange}{a}iin}\\


%%%%%%%
% 37V %
%%%%%%%
\clearpage
\subsection*{Folio 37, Verso}
\begin{locator}[P1, 1]\end{locator} {\eva kShody qocThy qotoldy chopdain sol}\\
\begin{locator}[P1, 2]\end{locator} {\eva dor chol cThor o\textcolor{orange}{r}cho chor daiin}\\
\begin{locator}[P1, 3]\end{locator} {\eva qokchon Shy chon daiin dy}\\
\begin{locator}[P1, 4]\end{locator} {\eva dShor dytory dShor daiin}\\
\begin{locator}[P1, 5]\end{locator} {\eva dchor qotol ykchon dain}\\
\begin{locator}[P1, 6]\end{locator} {\eva yokor ytchor saiin oty}\\
\begin{locator}[P1, 7]\end{locator} {\eva qotchor daiin}\\

\vspace{1em}
\noindent\begin{locator}[P2, 8]\end{locator} {\eva qotor choiin chetchy daiin}\\
\begin{locator}[P2, 9]\end{locator} {\eva dor chor Sho daiiin daiin}\\
\begin{locator}[P2, 10]\end{locator} {\eva soiin Shey o\textcolor{orange}{k}oiin chey tom}\\
\begin{locator}[P2, 11]\end{locator} {\eva qotoiin choror cThol daiin}\\
\begin{locator}[P2, 12]\end{locator} {\eva chor Sholy Sheaiin dotody}\\
\begin{locator}[P2, 13]\end{locator} {\eva sotoiiin}\\

\vspace{1em}
\noindent\begin{locator}[P3, 14]\end{locator} {\eva todain cPhaiin cPhorods}\\
\begin{locator}[P3, 15]\end{locator} {\eva soiiin cheoky daiin dain}\\
\begin{locator}[P3, 16]\end{locator} {\eva qotor daiin chotaiin}\\
\begin{locator}[P3, 17]\end{locator} {\eva sokchor qokoiiin ykeeols}\\
\begin{locator}[P3, 18]\end{locator} {\eva oyteey daiin daiin ody}\\
\begin{locator}[P3, 19]\end{locator} {\eva daiin \textcolor{orange}{q}koiin qotal daiin}\\
\begin{locator}[P3, 20]\end{locator} {\eva ychoc\textcolor{orange}{K}a\textcolor{orange}{h}y ykol daiin s}\\
\begin{locator}[P3, 21]\end{locator} {\eva oShcTho do daiin cThols}\\
\begin{locator}[P3, 22]\end{locator} {\eva qotol ytoiin chocTh\textcolor{orange}{h}y}\\
\begin{locator}[P3, 23]\end{locator} {\eva yto chol daiin}\\


%%%%%%%
% 38R %
%%%%%%%
\clearpage
\subsection*{Folio 38, Recto}
\begin{locator}[P1, 1]\end{locator} {\eva tolor chocKhy oky choiin okShol oly oky}\\
\begin{locator}[P1, 2]\end{locator} {\eva ok\textcolor{orange}{Sh}ey chodys ytoiin otaiin otaiin cThar}\\
\begin{locator}[P1, 3]\end{locator} {\eva qokor okaiin otaiin qokchol chokokor}\\
\begin{locator}[P1, 4]\end{locator} {\eva ychok chey chcKh chy chk\textcolor{orange}{o} r odaiin daiin sy}\\
\begin{locator}[P1, 5]\end{locator} {\eva okor chey kain chor cTho dain cKholdy}\\
\begin{locator}[P1, 6]\end{locator} {\eva ySho Sho kos daiin okoy chochor daiin}\\


%%%%%%%
% 38V %
%%%%%%%
\clearpage
\subsection*{Folio 38, Verso}
\begin{locator}[P1, 1]\end{locator} {\eva okchop chol Shotol oteol okeey s\hfill chor d aiin d}\\
\begin{locator}[P1, 2]\end{locator} {\eva choiin Shey keo Sho oiin s chol ldy\hfill okeaiin okom}\\
\begin{locator}[P1, 3]\end{locator} {\eva qokeey keor daiin okey keey daiin\hfill dair daiin s}\\
\begin{locator}[P1, 4]\end{locator} {\eva okeey d\textcolor{orange}{$\smile$}aiin qokeey chot\textcolor{orange}{o}y tody\hfill oky aiiin d}\\
\begin{locator}[P1, 5]\end{locator} {\eva qoty daiin chol oteeol dody\hfill cheod chody}\\
\begin{locator}[P1, 6]\end{locator} {\eva Sho keeey key tey daiin daiiin\hfill dain dain}\\
\begin{locator}[P1, 7]\end{locator} {\eva daiin qol chy dain * or daiin\hfill dai\textcolor{orange}{n} daldy}\\
\begin{locator}[P1, 8]\end{locator} {\eva o aiin char chShol chokaiin}\\

\subsubsection*{Notes}


%%%%%%%
% 39R %
%%%%%%%
\clearpage
\subsection*{Folio 39, Recto}
\begin{locator}[P1, 1]\end{locator} {\eva tedo chShd cPhhofy chdain Shey f\textcolor{orange}{Sh}y dy orain cheepaiin ofy Shey koly dy}\\
\begin{locator}[P1, 2]\end{locator} {\eva olchey cheky qokedy ShekShey qolain chc\textcolor{orange}{K}Shy chdy dair ShcKhey dold}\\
\begin{locator}[P1, 3]\end{locator} {\eva qokalchdy chekaiin checKhy dar Shed qokeedar \textcolor{orange}{ch}edy dar or cheey dy cKhy}\\
\begin{locator}[P1, 4]\end{locator} {\eva tohedy chdy olaiin chedy ShcKhdchy chol or ordy chees aly okalcheg}\\
\begin{locator}[P1, 5]\end{locator} {\eva dchdy chdy ykaiin}\\

\vspace{1em}
\noindent\begin{locator}[P2, 6]\end{locator} {\eva pchdaiin She dam qofchedy Shedy kchdy dydy opchekaiin ShocKh\textcolor{orange}{$\cancel{x}$}hy Shdalo ry}\\
\begin{locator}[P2, 7]\end{locator} {\eva dchdar chedy cholal qokedar chdy chcKhdy dar ar al ydy eeesaii\textcolor{orange}{i}n}\\
\begin{locator}[P2, 8]\end{locator} {\eva lchedy Shedy qokaiin chkeey fchedy okam chcFhhy saiin chc\textcolor{orange}{K}hy dairo\textcolor{orange}{g}}\\
\begin{locator}[P2, 9]\end{locator} {\eva tchedy Shol odal qokaiin Shdaiin}\\

\vspace{1em}
\noindent\begin{locator}[P3, 10]\end{locator} {\eva pchdar Shedy ar aiir okair ykeols Shedy qocKhdy laiin \textcolor{orange}{S}yky}\\
\begin{locator}[P3, 11]\end{locator} {\eva dShdy ar aiin y yk\textcolor{orange}{e}r chdy ykal olkaiin qokaldar cheol dar aiin}\\
\begin{locator}[P3, 12]\end{locator} {\eva tolchdaiin chcKhy saiin olor chls aiin oky ches aiin dal}\\
\begin{locator}[P3, 13]\end{locator} {\eva ytor chdar Shey qokaiin chor kar Sheolkedy otedy tedy saiin}\\
\begin{locator}[P3, 14]\end{locator} {\eva daiin ShcKhy chekl ol daiin chedy ykeey daiin otal chdam qokam}\\
\begin{locator}[P3, 15]\end{locator} {\eva qolkain olcheol daiin dar ol dld \textcolor{orange}{o}r ador aiin ofcheefar}\\
\begin{locator}[T1, 16]\end{locator} {\eva oteol cholkal qokal dar ykdy}\\

\subsubsection*{Notes}


%%%%%%%
% 39V %
%%%%%%%
\clearpage
\subsection*{Folio 39, Verso}
\begin{locator}[P1, 1]\end{locator} {\eva pdair chdy fdykain or air Sheykaiiin ofchy kar or aiin dol ky oShdy}\\
\begin{locator}[P1, 2]\end{locator} {\eva sor Shy kor chol qoty kchdy olky dor chdy ykar olkedaiin ody dy}\\
\begin{locator}[P1, 3]\end{locator} {\eva daiin chor okain okaiifchody saiin or aiin qokaiin ytodaiin okom}\\
\begin{locator}[P1, 4]\end{locator} {\eva y okeey chody cheor aiin okody chodal ykedy qokedy dal or aiin Shky}\\
\begin{locator}[P1, 5]\end{locator} {\eva ytedykor or Sheky kain otar or aiin okaiin cKhol ol kor otor opchy}\\
\begin{locator}[P1, 6]\end{locator} {\eva lkedy okchey Shor\textcolor{orange}{$\smile$}qoy\textcolor{orange}{$\frown$}kam cho cKhcFhhy or aly Shody}\\

\vspace{1em}
\noindent\begin{locator}[P2, 7]\end{locator} {\eva pardy Shedy qokar Sheedy oraly\hfill olaiir okar ar}\\
\begin{locator}[P2, 8]\end{locator} {\eva oekar aiin olkaiin olky dar ald\hfill Shek chek qokchy dar ain}\\
\begin{locator}[P2, 9]\end{locator} {\eva tar aiin dal ar ain cheor ydam\hfill Shody okal Shdy kShy or}\\
\begin{locator}[P2, 10]\end{locator} {\eva dar ar ykar or yky chdy fchor\hfill qokain ar Sheedy olchef}\\
\begin{locator}[P2, 11]\end{locator} {\eva sarol chedy Shekam qokar chl\hfill ykeedy chcKhy dalor dy}\\
\begin{locator}[P2, 12]\end{locator} {\eva paiin alaiin otal chd okar am\hfill okar cheodal ocKhy}\\
\begin{locator}[P2, 13]\end{locator} {\eva dain o\textcolor{orange}{I}Khedy otedy\hfill okedy lchdy okaiin daiiiny}\\
\begin{locator}[P2, 14]\end{locator} {\eva tar aiin okaiin cholody}\\

\subsubsection*{Notes}


%%%%%%%
% 40R %
%%%%%%%
\clearpage
\subsection*{Folio 40, Recto}
\begin{locator}[P1, 1]\end{locator} {\eva pchey keodar aldydy\hfill qoky okal Shdy olkedy opches ar ordaiin}\\
\begin{locator}[P1, 2]\end{locator} {\eva qokar okar okedy dar\hfill ykchey kaiin okos chedy okar aralos}\\
\begin{locator}[P1, 3]\end{locator} {\eva Shy qokal chdy chcKhd\hfill otor aiis oky okolchy qokar okam}\\
\begin{locator}[P1, 4]\end{locator} {\eva or aiin chekody das\hfill qokol okaiin okar oky okoldy ol}\\
\begin{locator}[P1, 5]\end{locator} {\eva lokar qokar okar\hfill okol ol chedy qokchd ar ar or dam}\\
\begin{locator}[P1, 6]\end{locator} {\eva tor or ar Shokor om\hfill olShedy qokam chdy kar oraiin}\\
\begin{locator}[P1, 7]\end{locator} {\eva yaiin chekaiin oky\hfill ycheey}\\

\vspace{1em}
\noindent\begin{locator}[P2, 8]\end{locator} {\eva kSheo k\textcolor{orange}{ch}ey dar aim\hfill kcheo cFhdy orain chefal daiin dm}\\
\begin{locator}[P2, 9]\end{locator} {\eva taiin ol olaiin or\hfill dain okaiin okaiin okaiin daram}\\
\begin{locator}[P2, 10]\end{locator} {\eva saiin olcheey chdy\hfill ychey karar oky ykedy okair ody}\\
\begin{locator}[P2, 11]\end{locator} {\eva toar ykaiin ory dal}\\

\subsubsection*{Notes}


%%%%%%%
% 40V %
%%%%%%%
\clearpage
\subsection*{Folio 40, Verso}
\begin{locator}[P1, 1]\end{locator} {\eva pchedain chefaiin oldy}\\
\begin{locator}[P1, 2]\end{locator} {\eva todar qokaiin ol aram}\\
\begin{locator}[P1, 3]\end{locator} {\eva s air ain okaiin okam}\\
\begin{locator}[P1, 4]\end{locator} {\eva taraiin okaiin chcK\textcolor{orange}{$\cancel{x}$}y}\\
\begin{locator}[P1, 5]\end{locator} {\eva solaiin okar oly chcKhy}\\
\begin{locator}[P1, 6]\end{locator} {\eva qoeed aiin ol \textcolor{orange}{ch}edy daiin}\\
\begin{locator}[P1, 7]\end{locator} {\eva Shody qokol chedy s ar}\\
\begin{locator}[P1, 8]\end{locator} {\eva chodaiin chkal ykedy okal}\\
\begin{locator}[P1, 9]\end{locator} {\eva tchy pchody pchdy k\textcolor{orange}{a}r ol}\\
\begin{locator}[P1, 10]\end{locator} {\eva tchkaiin tchedy qokaiin oraiin}\\
\begin{locator}[P1, 11]\end{locator} {\eva schedy qokol chedy dalor aIThy}\\
\begin{locator}[P1, 12]\end{locator} {\eva ycheekeey daiin okaiin}\\

\vspace{1em}
\noindent\begin{locator}[P2, 13]\end{locator} {\eva toees chedy kedy olfchedy qokedy\hfill daiin chefain}\\
\begin{locator}[P2, 14]\end{locator} {\eva saiin o lchey kchedy okar qokchdy dy\hfill qokees am chdy}\\
\begin{locator}[P2, 15]\end{locator} {\eva qokchey qody or aiiin okaiin o cKhy\hfill Sheod fai\textcolor{orange}{d}y}\\
\begin{locator}[P2, 16]\end{locator} {\eva Shol kedy lor ar okar qoky kedy r\hfill yteey qoka\textcolor{orange}{$\cancel{x}$}m}\\
\begin{locator}[P2, 17]\end{locator} {\eva tochey qokeedy qokaiin okeor qokar\hfill okees ar oky}\\
\begin{locator}[P2, 18]\end{locator} {\eva saiin otain chcKhy okal okair arol\hfill qokey okary}\\
\begin{locator}[T1, 19]\end{locator} {\eva pchedy chetar ofair arody}\\

\subsubsection*{Notes}


%%%%%%%
% 41R %
%%%%%%%
\clearpage
\subsection*{Folio 41, Recto}
\begin{locator}[P1, 1]\end{locator} {\eva p\textcolor{orange}{S}hey kedaleey oked Shek\textcolor{orange}{ech}y\hfill opShes ypchd qotche\textcolor{orange}{$\cancel{x}$}dy Shkakeedy}\\
\begin{locator}[P1, 2]\end{locator} {\eva ykeeo alShey \textcolor{orange}{o}kedy keShdy dy\hfill dor y\textcolor{orange}{ech}oky qokeed chpy qokedy dy}\\
\begin{locator}[P1, 3]\end{locator} {\eva qok ra\textcolor{orange}{y} y\textcolor{orange}{o}kedy ychdy\textcolor{orange}{$\frown$}kchdy\hfill qokedy qokedy Shekchy cheky daly}\\
\begin{locator}[P1, 4]\end{locator} {\eva chdchy ytcheeky ypchedy schdy\hfill ytedy cTh\textcolor{orange}{h}y chees oteey otal dam}\\
\begin{locator}[P1, 5]\end{locator} {\eva qotchy sal yteedy kchdy\hfill dchedy \textcolor{orange}{k}chdy dchedy dalain}\\

\vspace{1em}
\noindent\begin{locator}[P2, 6]\end{locator} {\eva Shedey polchedy qokeey chekedy ytey\hfill chkchod ypchedpy Shepy Shedy}\\
\begin{locator}[P2, 7]\end{locator} {\eva parchdy kchey yteedy qokeody\hfill ykedy chkedy qokedy chedy qokedy}\\
\begin{locator}[P2, 8]\end{locator} {\eva dair chedy chcKhy qokey lchdy\hfill qokedy qok\textcolor{orange}{a}l cheked\textcolor{orange}{y} qodor am}\\
\begin{locator}[P2, 9]\end{locator} {\eva qokeedy okedy chekedy chedy\hfill chcKheody chekal daiin cheo al okedy}\\
\begin{locator}[P2, 10]\end{locator} {\eva s\textcolor{orange}{$\frown$}Shok Shedy q\textcolor{orange}{o}kchdy dchdaldy\hfill ykedy qokeedy qokedy qoke\textcolor{orange}{d}}\\
\begin{locator}[P2, 11]\end{locator} {\eva ykeod ykeedy chekeedy cheked}\\

\subsubsection*{Notes}


%%%%%%%
% 41V %
%%%%%%%
\clearpage
\subsection*{Folio 41, Verso}
\begin{locator}[T1, 1]\end{locator} {\eva kee\textcolor{orange}{s}odal}\\
\begin{locator}[P1, 2]\end{locator} {\eva pcheody qofcheepy ofchdy cFhekchdy\hfill ypchedy chepchefy Shdchdy qotal dar}\\
\begin{locator}[P1, 3]\end{locator} {\eva dShedy tchey s\textcolor{orange}{$\frown$}aiin Shekey okedy okaly\hfill daiin okedy ykeeody choy \textcolor{orange}{t}eoy dam}\\
\begin{locator}[P1, 4]\end{locator} {\eva qokeody ok\textcolor{orange}{c}y qokeody oleeol lkedy\hfill lkeeody qokeedy okeey qokol Sheols}\\
\begin{locator}[P1, 5]\end{locator} {\eva ycheos olchey daiin or chol\textcolor{orange}{$\frown$}ol\textcolor{orange}{$\frown$}aiin\hfill oteedy qoteol oteodar orain}\\
\begin{locator}[P1, 6]\end{locator} {\eva todaiin ol cheos yteedy okal old\hfill oteol qokal or oteody}\\
\begin{locator}[P1, 7]\end{locator} {\eva ykeey okey ykeeol cKhdy chdal\hfill ykeo\textcolor{orange}{$\frown$}aiin okeody oly}\\
\begin{locator}[P1, 8]\end{locator} {\eva daiin olkeeo lkol\textcolor{orange}{$\frown$}chedy okeey}\\

\subsubsection*{Notes}


%%%%%%%%%%%%
% ANALYSIS %
%%%%%%%%%%%%
\clearpage
\section{Analysis}\label{sec:analysis}


%%%%%%%%%%%%%
% FREQUENCY %
%%%%%%%%%%%%%
\subsection{Glyph Frequencies}\label{sec:frequency}
Based on a transliteration by Takahashi (see Section \ref{sec:introduction}), a first basic analysis of frequencies can be performed, by simply counting each Latin letter of the transliteration.
This results in Table \ref{tab:count_translit}, listing all letters that are used.

\begin{table}[ht]
\center
\begin{tabular}{ccrr}
   \hline
   Letter       &   Glyph      & \multicolumn{2}{c}{Count}       \\
                &              & Takahashi   & Takahashi/Bauer   \\
   \hline\hline
   \texttt{o}   &   {\eva o}   & 25,468      &                   \\
   \texttt{e}   &   {\eva e}   & 20,070      &                   \\
   \texttt{h}   &   {\eva h}   & 17,856      &                   \\
   \texttt{y}   &   {\eva y}   & 17,655      &                   \\
   \texttt{a}   &   {\eva a}   & 14,281      &                   \\
   \texttt{c}   &   {\eva c}   & 13,314      &                   \\
   \texttt{d}   &   {\eva d}   & 12,973      &                   \\
   \texttt{i}   &   {\eva i}   & 11,660      &                   \\
   \texttt{l}   &   {\eva l}   & 10,518      &                   \\
   \texttt{k}   &   {\eva k}   &  9,996      &                   \\
   \texttt{r}   &   {\eva r}   &  7,456      &                   \\
   \texttt{n}   &   {\eva n}   &  6,141      &                   \\
   \texttt{t}   &   {\eva t}   &  5,968      &                   \\
   \texttt{q}   &   {\eva q}   &  5,423      &                   \\
   \texttt{S}   &   {\eva S}   &  4,501      &                   \\
   \texttt{s}   &   {\eva s}   &  2,886      &                   \\
   \texttt{p}   &   {\eva p}   &  1,406      &                   \\
   \texttt{m}   &   {\eva m}   &  1,116      &                   \\
   \texttt{T}   &   {\eva T}   &    976      &                   \\
   \texttt{K}   &   {\eva K}   &    938      &                   \\
   \texttt{f}   &   {\eva f}   &    425      &                   \\
   \texttt{P}   &   {\eva P}   &    224      &                   \\
   \texttt{g}   &   {\eva g}   &     96      &                   \\
   \texttt{F}   &   {\eva F}   &     80      &                   \\
   \texttt{I}   &   {\eva I}   &     72      &                   \\
   \texttt{x}   &   {\eva x}   &     35      &                   \\
   \texttt{v}   &   {\eva v}   &      9      &                   \\
   \texttt{z}   &   {\eva z}   &      2      &                   \\
   \hline
\end{tabular}
\caption{Letter frequencies in Takahashi's transliteration.}
\label{tab:count_translit}
\end{table}

Determining what counts as \textit{one} glyph, though, is in some cases an interpretative act.
It seems straightforward to assume that {\eva c} and {\eva h} are one glyph (i.e., {\eva ch}).
Likewise, {\eva S} and {\eva h} could be interpreted as another glyph (i.e., {\eva Sh}).
Additionally, one \textit{might} interpret instances of {\eva cTh}, {\eva cKh}, {\eva cPh}, or {\eva cFh} as representing {\eva ch} (or {\eva Sh}), on the one hand, and {\eva t}, {\eva k}, {\eva p}, or {\eva f}, on the other.
Concerning those ``gallows'' characters, as they have come to be called, it is not clear whether {\eva t} and {\eva k} or {\eva p} and {\eva f} are variations of the same glyphs, only diverging in one small loop (as \cite{tiltman_voynich_1967} seems to have interpreted them).
In the following, the four gallows will be treated as distinct glyphs, as has become customary.
Accordingly, Table \ref{tab:count_c} gives frequencies for several {\eva c} combinations, taking into account four forms of ``pedestalled gallows''.
The table also includes three instances that bear some resemblance to Latin abbreviations (also see \cite[p. 95]{dimperio_voynich_1978}).

\begin{table}[ht]
\center
\begin{tabular}{ccrr}
   \hline
   Letter(s)      & Glyph        & \multicolumn{2}{c}{Count}       \\
                  &              & Takahashi   & Takahashi/Bauer   \\
   \hline\hline
   \texttt{ch}    & {\eva ch}    & 11,008      &                   \\
   \texttt{Sh}    & {\eva Sh}    &  4,501      &                   \\
   \texttt{cTh}   & {\eva cTh}   &    950      &                   \\
   \texttt{cKh}   & {\eva cKh}   &    906      &                   \\
   \texttt{cPh}   & {\eva cPh}   &    216      &                   \\
   \texttt{cFh}   & {\eva cFh}   &     74      &                   \\
   \texttt{c}     & {\eva c}     &    143      &                   \\
   \texttt{co}    & {\eva co}    &      9      &                   \\
   \texttt{cy}    & {\eva cy}    &      7      &                   \\
   \hline
\end{tabular}
\caption{Glyph frequencies in Takahashi's transliteration for several {\eva c} combinations.}
\label{tab:count_c}
\end{table}

Furthermore, while there are instances of a single {\eva i}, it is not clear whether instances like {\eva ii} and {\eva iii} or combinations with {\eva n} like {\eva in} or {\eva iin} represent separate glyphs.
Table \ref{tab:count_i} gives frequencies for several of those {\eva i} combinations.
Assuming there are three glyphs {\eva i}, {\eva in}, and {\eva iin}, it is of course difficult to determine when there actually is a {\eva in} or a {\eva iin}.

\begin{table}[ht]
\center
\begin{tabular}{ccrr}
   \hline
   Letter(s)        & Glyph          & \multicolumn{2}{c}{Count}       \\
                    &                & Takahashi   & Takahashi/Bauer   \\
   \hline\hline
   \texttt{i}       & {\eva i}       &   590       &                   \\
   \texttt{ii}      & {\eva ii}      &   195       &                   \\
   \texttt{iii}     & {\eva iii}     &    10       &                   \\
   \texttt{n}       & {\eva n}       &   148       &                   \\
   \texttt{in}      & {\eva in}      & 1,752       &                   \\
   \texttt{iin}     & {\eva iin}     & 4,076       &                   \\
   \texttt{iiin}    & {\eva iiin}    &   154       &                   \\
   \texttt{iiiin}   & {\eva iiiin}   &     2       &                   \\
   \hline
\end{tabular}
\caption{Glyph frequencies in Takahashi's transliteration for several {\eva i} combinations.}
\label{tab:count_i}
\end{table}

This being said, I tentatively stipulate a set of 22 glyphs.
Table \ref{tab:count_glyph} gives frequencies for those glyphs.
They are counted according to the following rules:

\begin{itemize}
   \item The glyphs {\eva a}, {\eva d}, {\eva e}, {\eva g}, {\eva I}, {\eva l}, {\eva m}, {\eva o}, {\eva q}, {\eva r}, {\eva s}, {\eva v}, {\eva x}, and {\eva y} are simply counted as before. Hence, frequencies are identical with Table \ref{tab:count_translit}.
   \item The glyph {\eva Sh} is counted every time {\eva S} and {\eva h} appear consecutively.
   \item The glyph {\eva ch} is counted every time {\eva c} and {\eva h} appear consecutively, as well as in cases where there is {\eva T}, {\eva K}, {\eva P}, or {\eva F} between them.
   \item The glyph {\eva t} is counted every time {\eva t} or {\eva T} appears.
   \item The glyph {\eva k} is counted every time {\eva k}, {\eva K}, or {\eva z} appears.
   \item The glyph {\eva p} is counted every time {\eva p} or {\eva P} appears.
   \item The glyph {\eva f} is counted every time {\eva f} or {\eva F} appears.
   \item The glyph {\eva in} is counted every time {\eva i} and {\eva n} appear consecutively without there being a preceding second {\eva i}.
   \item The glyph {\eva iin} is counted every time {\eva i}, {\eva i}, and {\eva n} appear consecutively.
   \item The glyph {\eva i} is counted every time it is not part of {\eva in} or {\eva iin} as well as in instances of {\eva I}.
\end{itemize}

\begin{table}[ht]
\center
\begin{tabular}{ccrrrr}
   \hline
   Letter(s)      & Glyph        & \multicolumn{4}{c}{Count (Percent)}                                     \\
                  &              & \multicolumn{2}{c}{Takahashi}   & \multicolumn{2}{c}{Takahashi/Bauer}   \\
   \hline\hline
   \texttt{o}     & {\eva o}     &  25,468   & (15.636)            &       &                               \\
   \texttt{e}     & {\eva e}     &  20,070   & (12.322)            &       &                               \\
   \texttt{y}     & {\eva y}     &  17,655   & (10.839)            &       &                               \\
   \texttt{a}     & {\eva a}     &  14,281   &  (8.768)            &       &                               \\
   \texttt{ch}    & {\eva ch}    &  13,154   &  (8.076)            &       &                               \\
   \texttt{d}     & {\eva d}     &  12,973   &  (7.965)            &       &                               \\
   \texttt{k}     & {\eva k}     &  10,934   &  (6.713)            &       &                               \\
   \texttt{l}     & {\eva l}     &  10,518   &  (6.458)            &       &                               \\
   \texttt{r}     & {\eva r}     &   7,456   &  (4.578)            &       &                               \\
   \texttt{t}     & {\eva t}     &   6,944   &  (4.263)            &       &                               \\
   \texttt{q}     & {\eva q}     &   5,423   &  (3.329)            &       &                               \\
   \texttt{Sh}    & {\eva Sh}    &   4,501   &  (2.763)            &       &                               \\
   \texttt{iin}   & {\eva iin}   &   4,232   &  (2.598)            &       &                               \\
   \texttt{s}     & {\eva s}     &   2,886   &  (1.772)            &       &                               \\
   \texttt{in}    & {\eva in}    &   1,752   &  (1.076)            &       &                               \\
   \texttt{p}     & {\eva p}     &   1,630   &  (1.001)            &       &                               \\
   \texttt{i}     & {\eva i}     &   1,240   &  (0.761)            &       &                               \\
   \texttt{m}     & {\eva m}     &   1,116   &  (0.685)            &       &                               \\
   \texttt{f}     & {\eva f}     &     505   &  (0.310)            &       &                               \\
   \texttt{g}     & {\eva g}     &     96    &  (0.059)            &       &                               \\
   \texttt{x}     & {\eva x}     &     35    &  (0.021)            &       &                               \\
   \texttt{v}     & {\eva v}     &      9    & (<0.001)            &       &                               \\
   \hline
                  &              & 162,878   &                     &       &                               \\
   \hline
\end{tabular}
\caption{Glyph frequencies in Takahashi's transliteration}
\label{tab:count_glyph}
\end{table}


%%%%%%%%%%%
% ENTROPY %
%%%%%%%%%%%
\subsection{Entropy}\label{sec:entropy}
Next, let's take a look at the manuscript's entropy.
To this end, I utilize the conceptualization of information entropy by \citet{shannon_mathematical_1948}, basically measuring the variability of glyphs in the manuscript.
Entropy $H$ can be defined as shown in (\ref{eq:shannon}), where $p_{z}$ denotes the probability of $z$, with $z$ being an element of the alphabet $Z$.

\begin{equation}\label{eq:shannon}
   H=-\sum_{z\in Z}p_{z}\log_{2}p_{z}
\end{equation}

\noindent To calculate the entropy, I suppose the above stipulated glyphs to form alphabet $Z$ (see Table \ref{tab:count_glyph}).
Accordingly, changes to the corpus are made so that each stipulated glyph has a single corresponding character in the transliteration.
Changes are made according to the following rules:

\begin{itemize}
   \item The glyph {\eva Sh} is represented by \texttt{Sh} in EVA. For calculating the entropy, every time \texttt{S} and \texttt{h} appear consecutively, they are replaced by \texttt{1}
   \item The glyph {\eva ch} is represented by \texttt{ch} in EVA. Every time \texttt{c} and \texttt{h} appear consecutively, they are replaced by \texttt{2}. Following the above stated rules for glyph counting, {\eva ch} is also believed to be present when {\eva c} and {\eva h} are separated by {\eva T}, {\eva K}, {\eva P}, or {\eva F}, represented by \texttt{T}, \texttt{K}, \texttt{P}, or \texttt{F} in EVA. In those cases, it is believed that {\eva t}, {\eva k}, {\eva p}, or {\eva f} are also present. They are represented by \texttt{t},\texttt{k}, \texttt{p}, or \texttt{f}  in EVA. Hence, they are replaced by \texttt{2t}, \texttt{2k}, \texttt{2p}, or \texttt{2f}.
   \item The glyph {\eva z}, represented by \texttt{z} in EVA, is counted as {\eva k}, represented by \texttt{k} in EVA. Hence, every \texttt{z} is replaced by \texttt{k}.
   \item The glyph {\eva iin} is represented by \texttt{iin} in EVA. Every time \texttt{i}, \texttt{i}, and \texttt{n} appear consecutively, they are replaced by \texttt{3}.   
   \item The glyph {\eva in} is represented by \texttt{in} in EVA. Every time \texttt{i} and \texttt{n} appear consecutively, they are replaced by \texttt{4}.
   \item Lastly, the glyph {\eva I}, represented by \texttt{I} in EVA, is counted as {\eva i}, represented by \texttt{i} in EVA. Hence, every \texttt{I} is replaced by \texttt{i}.
\end{itemize}

\noindent Calculating the entropy for this converted corpus results in $H=3.866$.
As comparisons, Table \ref{tab:entropy} also offers the entropy values for the unconverted transliteration as well as for a randomly generated text consisting of 22 different characters and being 234,956 characters long (just as the converted transliteration).

\begin{table}[ht]
\center
\begin{tabular}{lr}
   \hline
   Corpus                        & \multicolumn{1}{c}{$H$}   \\
   \hline\hline
   Takahashi                     & 4.021                     \\
   Takahashi (converted)         & 3.866                     \\
   Takahashi/Bauer               &                           \\
   Takahashi/Bauer (converted)   &                           \\
   Random text                   & 4.459                     \\
   \hline
\end{tabular}
\caption{Entropy ($H$) for different transliterations.}
\label{tab:entropy}
\end{table}


%%%%%%%%%%%%%%
% REFERENCES %
%%%%%%%%%%%%%%
\clearpage
\printbibliography

\end{document}