% BibTeX
% XeLaTeX

\documentclass{scrarticle}

\usepackage[english]{babel}
\usepackage[natbib,authordate,backend=bibtex]{biblatex-chicago}
   \bibliography{references}
\usepackage[a4paper,top=2cm,bottom=2cm,left=3cm,right=3cm,marginparwidth=1.75cm]{geometry}
\usepackage{fontspec}
   \newfontfamily\eva{fairfax_eva_hd.ttf}
\usepackage{xcolor}

\newenvironment{locator}{\tiny\sffamily}

\deffootnote{1.5em}{1em}{\makebox[1.5em][l]{\thefootnotemark}}
   \setlength{\skip\footins}{1.5em}
   \setlength{\footnotesep}{1em}

\title{Beinecke MS 408}
\date{}

\begin{document}
\maketitle


%%%%%%%%%%%%%%%%
% INTRODUCTION %
%%%%%%%%%%%%%%%%
\section{Introduction}\label{sec:introduction}
In Section \ref{sec:analysis}, a frequency analysis of glyphs is presented.

In Section \ref{sec:transcription}, mainly to promote easier readability and comparability, a transcription of the manuscript is provided, based on a transliteration by \citet{takahashi_voynich_2004}.
Takahashi's transliteration is obtained in an automatically capitalized version of the \textit{European} or \textit{Extensible Voynich Alphabet} (EVA), using the \textit{Interlinear Transcription Archive Extractor} by \citet{schwerdtfeger_voynich_2004}.\footnote{\citet{schwerdtfeger_voynich_2004} builds on an online archive of Voynich transcriptions that was maintained by Jorge \citet{stolfi_voynich_1998}. For a general overview of transcription efforts, see \citet{zandbergen_text_2023}.}
It is then converted back into glyphs using a font by \citet{bettencourt_voynich_2019}.

Using the manuscript's reproduction in \citet{clemens_voynich_2016} and the scans provided by  Yale University's \citet{beinecke_voynich_2004}, Takahashi's transliteration is compared with what I am able to make out of the manuscript.
Glyphs that I interpret in another way than Takahashi are printed in \textcolor{orange}{orange}, showing my interpretation.
Removed spaces are marked with \textcolor{orange}{$\frown$} and added spaces are indicated by \textcolor{orange}{$\smile$}.
For sake of simplicity, places, where the writing is interrupted by illustrations, are also marked by spaces.
Especially the recognition of spaces in running text is sometimes very difficult; therefore, my adjustments in this -- just as in any other -- regard should be taken with a certain amount of caution.
In total, I made changes in 95 places of Takahashi's transliteration.

Each line of the transcription is preceded by a locator in square brackets, informing the reader about the element a line is part of (i.e., paragraph, title, or label) and assigning the line a number.
For example, a line located at [P2, 9] is part of the page's second paragraph (P2) and is counted as the overall ninth line (9) on this page.


%%%%%%%%%%%%
% ANALYSIS %
%%%%%%%%%%%%
\clearpage
\section{Analysis}\label{sec:analysis}
\subsection{Glyph Frequencies}
Based on a transliteration by Takahashi (see Section \ref{sec:introduction}), a first basic analysis of frequencies can be performed, simply counting each latin letter of the transliteration.
This results in Table \ref{tab:count_translit}, listing all letters that are used.

\begin{table}[ht]
\center
\begin{tabular}{lcr}
   \hline
   Letter   &   Glyph      & Count    \\
   \hline\hline
   o        &   {\eva o}   & 25,468   \\
   e        &   {\eva e}   & 20,070   \\
   h        &   {\eva h}   & 17,856   \\
   y        &   {\eva y}   & 17,655   \\
   a        &   {\eva a}   & 14,281   \\
   c        &   {\eva c}   & 13,314   \\
   d        &   {\eva d}   & 12,973   \\
   i        &   {\eva i}   & 11,660   \\
   l        &   {\eva l}   & 10,518   \\
   k        &   {\eva k}   &  9,996   \\
   r        &   {\eva r}   &  7,456   \\
   n        &   {\eva n}   &  6,141   \\
   t        &   {\eva t}   &  5,968   \\
   q        &   {\eva q}   &  5,423   \\
   S        &   {\eva S}   &  4,501   \\
   s        &   {\eva s}   &  2,886   \\
   p        &   {\eva p}   &  1,406   \\
   m        &   {\eva m}   &  1,116   \\
   T        &   {\eva T}   &    976   \\
   K        &   {\eva K}   &    938   \\
   f        &   {\eva f}   &    425   \\
   P        &   {\eva P}   &    224   \\
   g        &   {\eva g}   &     96   \\
   F        &   {\eva F}   &     80   \\
   I        &   {\eva I}   &     72   \\
   x        &   {\eva x}   &     35   \\
   v        &   {\eva v}   &      9   \\
   z        &   {\eva z}   &      2   \\
   \hline
\end{tabular}
\caption{Letter frequencies in Takahashi's transliteration.}
\label{tab:count_translit}
\end{table}

Determining what counts as \textit{one} glyph, though, is in some cases an interpretative act.
It seems straightforward to assume that {\eva c} and {\eva h} are one glyph (i.e., {\eva ch}).
Likewise, {\eva S} and {\eva h} could be interpreted as another glyph (i.e., {\eva Sh}).
Additionally, one \textit{might} interpret instances of {\eva cTh}, {\eva cKh}, {\eva cPh}, or {\eva cFh} as representing {\eva ch} (or {\eva Sh}), on the one hand, and {\eva t}, {\eva k}, {\eva p}, or {\eva f}, on the other.
Concerning those ``gallows'' characters, as they have come to be called, it is not clear whether {\eva t} and {\eva k} or {\eva p} and {\eva f} are variations of the same glyphs, only diverging in one small loop (as \cite{tiltman_voynich_1967} interpreted them).
In the following, the four gallows will be treated as distinct glyphs, as has become customary.
Accordingly, Table \ref{tab:count_c} gives frequencies for several {\eva c} combinations, taking into account four forms of ``pedestalled gallows''.
The table also includes three instances that bear some resemblance to Latin abbreviations (also see \cite[p. 95]{dimperio_voynich_1978}).

\begin{table}[ht]
\center
\begin{tabular}{lr}
   \hline
   Glyph        & Count    \\
   \hline\hline
   {\eva ch}    & 11,008   \\
   {\eva Sh}    &  4,501   \\
   {\eva cTh}   &    950   \\
   {\eva cKh}   &    906   \\
   {\eva cPh}   &    216   \\
   {\eva cFh}   &     74   \\
   {\eva c}     &    143   \\
   {\eva co}    &      9   \\
   {\eva cy}    &      7   \\
   \hline
\end{tabular}
\caption{Glyph frequencies in Takahashi's transliteration for several {\eva c} combinations.}
\label{tab:count_c}
\end{table}

Furthermore, while there are instances of a single {\eva i}, it is not clear whether instances like {\eva ii} and {\eva iii} or combinations with {\eva n} like {\eva in} or {\eva iin} represent separate glyphs.
Table \ref{tab:count_i} gives frequencies for several of those {\eva i} combinations.
Assuming there are three glyphs {\eva i}, {\eva in}, and {\eva iin}, it is of course difficult to determine when there actually is a {\eva in} or a {\eva iin}.

\begin{table}[ht]
\center
\begin{tabular}{lr}
   \hline
   Glyph          & Count   \\
   \hline\hline
   {\eva i}       &   590   \\
   {\eva ii}      &   195   \\
   {\eva iii}     &    10   \\
   {\eva n}       &   148   \\
   {\eva in}      & 1,752   \\
   {\eva iin}     & 4,076   \\
   {\eva iiin}    &   154   \\
   {\eva iiiin}   &     2   \\
   \hline
\end{tabular}
\caption{Glyph frequencies in Takahashi's transliteration for several {\eva i} combinations.}
\label{tab:count_i}
\end{table}

This being said, I tentatively stipulate a set of 23 glyphs.
Table \ref{tab:count_glyph} gives frequencies for those glyphs.
They are counted as follows:

\begin{itemize}
   \item The glyphs {\eva a}, {\eva d}, {\eva e}, {\eva g}, {\eva I}, {\eva l}, {\eva m}, {\eva o}, {\eva q}, {\eva r}, {\eva s}, {\eva v}, {\eva x}, and {\eva y} are simply counted as before. Hence, frequencies are identical with Table \ref{tab:count_translit}.
   \item The glyph {\eva Sh} is counted every time {\eva S} and {\eva h} appear consecutively.
   \item The glyph {\eva ch} is counted every time {\eva c} and {\eva h} appear consecutively, as well as in cases where there is {\eva T}, {\eva K}, {\eva P}, or {\eva F} between them.
   \item The glyph {\eva t} is counted every time {\eva t} or {\eva T} appears.
   \item The glyph {\eva k} is counted every time {\eva k}, {\eva K}, or {\eva z} appears.
   \item The glyph {\eva p} is counted every time {\eva p} or {\eva P} appears.
   \item The glyph {\eva f} is counted every time {\eva f} or {\eva F} appears.
   \item The glyph {\eva in} is counted every time {\eva i} and {\eva n} appear consecutively without there being a preceding second {\eva i}.
   \item The glyph {\eva iin} is counted every time {\eva i}, {\eva i}, and {\eva n} appear consecutively.
   \item The glyph {\eva i} is counted every time it is not part of {\eva in} or {\eva iin}.
\end{itemize}

\begin{table}[ht]
\center
\begin{tabular}{lrr}
   \hline
   Glyph        & Count     & Percent   \\
   \hline\hline
   {\eva o}     &  25,468   & 15.636    \\
   {\eva e}     &  20,070   & 12.322    \\
   {\eva y}     &  17,655   & 10.839    \\
   {\eva a}     &  14,281   &  8.768    \\
   {\eva ch}    &  13,154   &  8.076    \\
   {\eva d}     &  12,973   &  7.965    \\
   {\eva k}     &  10,934   &  6.713    \\
   {\eva l}     &  10,518   &  6.458    \\
   {\eva r}     &   7,456   &  4.578    \\
   {\eva t}     &   6,944   &  4.263    \\
   {\eva q}     &   5,423   &  3.329    \\
   {\eva Sh}    &   4,501   &  2.763    \\
   {\eva iin}   &   4,232   &  2.598    \\
   {\eva s}     &   2,886   &  1.772    \\
   {\eva in}    &   1,752   &  1.076    \\
   {\eva p}     &   1,630   &  1.001    \\
   {\eva i}     &   1,168   &  0.717    \\
   {\eva m}     &   1,116   &  0.685    \\
   {\eva f}     &     505   &  0.310    \\
   {\eva g}     &     96    &  0.059    \\
   {\eva I}     &     72    &  0.044    \\
   {\eva x}     &     35    &  0.021    \\
   {\eva v}     &      9    & <0.001    \\
   \hline
                & 162,878   &           \\
   \hline
\end{tabular}
\caption{Glyph frequencies in Takahashi's transliteration}
\label{tab:count_glyph}
\end{table}


%%%%%%%%%%%%%%%%%
% TRANSCRIPTION %
%%%%%%%%%%%%%%%%%
\clearpage
\section{Transcription}\label{sec:transcription}


%%%%%%
% 1R %
%%%%%%
\subsection*{Folio 1, Recto}
\begin{locator}[P1, 1]\end{locator} {\eva fachys ykal ar ataiin Shol Shory cThres y kor Sholdy}\\
\begin{locator}[P1, 2]\end{locator} {\eva sory cKhar or y kair chtaiin Shar are cThar cThar dan}\\
\begin{locator}[P1, 3]\end{locator} {\eva syaiir Sheky or ykaiin Shod cThoary cThes daraiin s\textcolor{orange}{y}}\\
\begin{locator}[P1, 4]\end{locator} {\eva \textcolor{orange}{d}oiin oteey oteos roloty cTh\textcolor{orange}{i}ar daiin o\textcolor{orange}{k}aiin or okan}\\
\begin{locator}[P1, 5]\end{locator} {\eva dair\textcolor{orange}{$\frown$}y chear cThaiin cPhar cFhaiin}\\
\begin{locator}[T1, 6]\end{locator} {\eva ydaraiShy}\\

\vspace{1em}
\noindent\begin{locator}[P2, 7]\end{locator} {\eva \textcolor{orange}{ü} odar \textcolor{orange}{S$\frown$}y Shol cPhoy oydar Sh*\textcolor{orange}{$\frown$}s cFhoaiin Shodary}\\
\begin{locator}[P2, 8]\end{locator} {\eva yShey Shody okchoy otchol chocThy oschy dain chor kos}\\
\begin{locator}[P2, 9]\end{locator} {\eva daiin Shos cFhol Shody}\\
\begin{locator}[T2, 10]\end{locator} {\eva dain os teody}\\

\vspace{1em}
\noindent\begin{locator}[P3, 11]\end{locator} {\eva \textcolor{orange}{ü} ydain cPhesaiin ol\textcolor{orange}{$\frown$}s cPhey ytain ShoShy cPhodales}\\
\begin{locator}[P3, 12]\end{locator} {\eva okSho kShoy otairin oteol okan Shodain scKhey daiin}\\
\begin{locator}[P3, 13]\end{locator} {\eva Shoy cKhey kodaiin cPhy cPhodaiils cThey She oldain d}\\
\begin{locator}[P3, 14]\end{locator} {\eva dain oiin chol odaiin chodain chdy okain dan cThy kod}\\
\begin{locator}[P3, 15]\end{locator} {\eva daiin ShcKhey \textcolor{orange}{cKh}or chor Shey kol chol chol kor chal}\\
\begin{locator}[P3, 16]\end{locator} {\eva Sho chol Shodan kShy kchy dor chodaiin Sho kchom}\\
\begin{locator}[P3, 17]\end{locator} {\eva ycho tchey ch\textcolor{orange}{e}kain Sheo pShol dydyd cThy daicThy}\\
\begin{locator}[P3, 18]\end{locator} {\eva yto Shol She kodShey cPhealy dasain dain cKhyds}\\
\begin{locator}[P3, 19]\end{locator} {\eva dchar ShcThaiin okaiir\textcolor{orange}{$\frown$}chey rchy potol cThols d\textcolor{orange}{a}octa}\\
\begin{locator}[P3, 20]\end{locator} {\eva Shok chor chey dain cKhey}\\
\begin{locator}[T3, 21]\end{locator} {\eva otol daiiin}\\

\vspace{1em}
\noindent\begin{locator}[P4, 22]\end{locator} {\eva cPho Shaiin Shokcheey chol tShodeesy Shey pydeey chy ro d\textcolor{orange}{ar}}\\
\begin{locator}[P4, 23]\end{locator} {\eva *doin chol dain cThal dar Shear kaiin dar Shey cThar}\\
\begin{locator}[P4, 24]\end{locator} {\eva cho*o kaiin Shoaiin okol daiin far cThol daiin cTholdar}\\
\begin{locator}[P4, 25]\end{locator} {\eva ycheey okay oky daiin okchey kokaiin \textcolor{orange}{of}chol k\textcolor{orange}{ad}chy dal}\\
\begin{locator}[P4, 26]\end{locator} {\eva d\textcolor{orange}{che}o Shody koShey cThy okchey keey keey dal chtor}\\
\begin{locator}[P4, 27]\end{locator} {\eva \textcolor{orange}{ch}o chol chok choty chotey}\\
\begin{locator}[T4, 28]\end{locator} {\eva dchaiin}\\


%%%%%%
% 1V %
%%%%%%
\clearpage
\subsection*{Folio 1, Verso}
\begin{locator}[P1, 1]\end{locator} {\eva kchsy chadaiin ol oltchey char cFhar am}\\
\begin{locator}[P1, 2]\end{locator} {\eva ytee\textcolor{orange}{y} char or ochy dcho lkody okodar chody}\\
\begin{locator}[P1, 3]\end{locator} {\eva do cKhy cKho cKhy Shy dkSheey cThy kotchody dal}\\
\begin{locator}[P1, 4]\end{locator} {\eva dol chokeo dair dam sochey chokody}\\

\vspace{1em}
\noindent\begin{locator}[P2, 5]\end{locator} {\eva potoy Shol dair cPhoal dar chey tody otoaiin ShoShy}\\
\begin{locator}[P2, 6]\end{locator} {\eva choky chol cThol Shol okal dolchey chodo lol chy cThy}\\
\begin{locator}[P2, 7]\end{locator} {\eva qo ol choee\textcolor{orange}{s} cheol dol cThey ykol dol dolo ykol do lchody}\\
\begin{locator}[P2, 8]\end{locator} {\eva okol\textcolor{orange}{$\frown$}Shol kol kechy chol ky chol cThol chody chol daiin}\\
\begin{locator}[P2, 9]\end{locator} {\eva Shor okol chol dol ky dar Shol dchor otcho dar Shody}\\
\begin{locator}[P2, 10]\end{locator} {\eva taor chotchey dal chody schody pol chodar}\\


%%%%%%
% 2R %
%%%%%%
\clearpage
\subsection*{Folio 2, Recto}
\begin{locator}[P1, 1]\end{locator} {\eva kydainy ypchol daiin otchal ypchaiin cKholsy}\\
\begin{locator}[P1, 2]\end{locator} {\eva dorchory chkar s Shor cThy cTh}\\
\begin{locator}[P1, 3]\end{locator} {\eva qotaiin cThey y chor chy ydy chaiin}\\
\begin{locator}[P1, 4]\end{locator} {\eva chaindy chtod dy cPhy dals chokaiin d}\\
\begin{locator}[P1, 5]\end{locator} {\eva otochor al Shodaiin chol dan ytchaiin dan}\\
\begin{locator}[P1, 6]\end{locator} {\eva saiin daind dkol sor ytoldy dchol dchy cThy}\\
\begin{locator}[P1, 7]\end{locator} {\eva Shor cKhy daiiny chol dan}\\

\vspace{1em}
\noindent\begin{locator}[L1, 8]\end{locator} {\eva \textcolor{orange}{ytoaile*}}\\
\begin{locator}[L2, 9]\end{locator} {\eva \textcolor{orange}{***an}}\\

\vspace{1em}
\noindent\begin{locator}[P2, 10]\end{locator} {\eva kydain Shaiin qoy s Shol fodan ykSh olSheey daiildy}\\
\begin{locator}[P2, 11]\end{locator} {\eva dlsSho kol Sheey qokey ykody so chol yky dain daiirol}\\
\begin{locator}[P2, 12]\end{locator} {\eva qoky cholaiin Shol Sheky daiin cThey keol saiin saiin}\\
\begin{locator}[P2, 13]\end{locator} {\eva ychain dal chy dalor Shan dan olsaiin Sheey cKhor}\\
\begin{locator}[P2, 14]\end{locator} {\eva okol chy chor cThor yor an chan saiin chety chyky sal}\\
\begin{locator}[P2, 15]\end{locator} {\eva Sho ykeey chey daiin chcThy}\\


%%%%%%
% 2V %
%%%%%%
\clearpage
\subsection*{Folio 2, Verso}
\begin{locator}[P1, 1]\end{locator} {\eva kooiin cheo pchor otaiin o dain chor dair Shty}\\
\begin{locator}[P1, 2]\end{locator} {\eva kcho kchy Sho Shol qotcho loeees qoty chor daiin}\\
\begin{locator}[P1, 3]\end{locator} {\eva otchy chor lShy chol chody chodain chcThy daiin}\\
\begin{locator}[P1, 4]\end{locator} {\eva Sho cholo cheor chodaiin}\\

\vspace{1em}
\noindent\begin{locator}[P2, 5]\end{locator} {\eva kchor Shy daiiin chcKhoy s Shey dor chol daiin}\\
\begin{locator}[P2, 6]\end{locator} {\eva dor chol chor chol keol chy chty daiin otchor chan}\\
\begin{locator}[P2, 7]\end{locator} {\eva daiin chotchey qoteeey chokeos chees chr cheaiin}\\
\begin{locator}[P2, 8]\end{locator} {\eva chokoiShe chor cheol chol dolody}\\


%%%%%%
% 3R %
%%%%%%
\clearpage
\subsection*{Folio 3, Recto}
\begin{locator}[P1, 1]\end{locator} {\eva tSheos qopal chol cThol daimm}\\
\begin{locator}[P1, 2]\end{locator} {\eva ycheor chor dam qotcham cham}\\
\begin{locator}[P1, 3]\end{locator} {\eva ochor qocheor chol daiin cThy}\\
\begin{locator}[P1, 4]\end{locator} {\eva schey chor chal cham cham cho}\\
\begin{locator}[P1, 5]\end{locator} {\eva qokol chololy s cham cThol}\\
\begin{locator}[P1, 6]\end{locator} {\eva ychtaiin chor cThom otal\textcolor{orange}{$\frown$}dam}\\
\begin{locator}[P1, 7]\end{locator} {\eva otchol qodaiin chom Shom damo}\\
\begin{locator}[P1, 8]\end{locator} {\eva ySheor chor chol oky damo}\\
\begin{locator}[P1, 9]\end{locator} {\eva Sho *or Sheoldam otchody ol}\\
\begin{locator}[P1, 10]\end{locator} {\eva ydas chol cThom}\\

\vspace{1em}
\noindent\begin{locator}[P2, 11]\end{locator} {\eva pcheol Shol sols Sheol Shey}\\
\begin{locator}[P2, 12]\end{locator} {\eva okadaiin qokchor qoschodam ocThy}\\
\begin{locator}[P2, 13]\end{locator} {\eva qokeey qot Shey qokody qok\textcolor{orange}{Sh}ey cheody}\\
\begin{locator}[P2, 14]\end{locator} {\eva chor qodair okeey qokeey}\\

\vspace{1em}
\noindent\begin{locator}[P3, 15]\end{locator} {\eva tSheoarom Shor or chor olchsy chom otchom oporar}\\
\begin{locator}[P3, 16]\end{locator} {\eva oteol chol s cheol ekShy qokeom qokol daiin soleeg}\\
\begin{locator}[P3, 17]\end{locator} {\eva soeom okeom yteody qokeeo\textcolor{orange}{$\frown$}dal sam}\\

\vspace{1em}
\noindent\begin{locator}[P4, 18]\end{locator} {\eva pcheoldom Shodaiin qopchor qopol opchol qoty otolom}\\
\begin{locator}[P4, 19]\end{locator} {\eva otchor ol cheor qoeor dair qoteol qosaiin chor cThy}\\
\begin{locator}[P4, 20]\end{locator} {\eva ycheor chol odaiin chol s aiin okolor am}\\


%%%%%%
% 3V %
%%%%%%
\clearpage
\subsection*{Folio 3, Verso}
\begin{locator}[P1, 1]\end{locator} {\eva koaiin cPhor qotoy Sha cKhol ykoaiin s oly}\\
\begin{locator}[P1, 2]\end{locator} {\eva daiidy qot\textcolor{orange}{ch}ol okchor okor olytol dol dar}\\
\begin{locator}[P1, 3]\end{locator} {\eva okom chol Shol seees chom cheeykam okai}\\
\begin{locator}[P1, 4]\end{locator} {\eva qodar \textcolor{orange}{ch}s \textcolor{orange}{ch}y kcheol okal do\textcolor{orange}{$\frown$}r chear een}\\
\begin{locator}[P1, 5]\end{locator} {\eva y\textcolor{orange}{ch}ear otchal \textcolor{orange}{ch}or \textcolor{orange}{ch}ar cKhy}\\
\begin{locator}[P1, 6]\end{locator} {\eva or cheor kor chodaly chom}\\

\vspace{1em}
\noindent\begin{locator}[P2, 7]\end{locator} {\eva tchor otcham chor cFham s}\\
\begin{locator}[P2, 8]\end{locator} {\eva ykchy kchom chor ch\textcolor{orange}{cKh}ol oka}\\
\begin{locator}[P2, 9]\end{locator} {\eva ytcheear okeol cThodoaly chor cThy}\\
\begin{locator}[P2, 10]\end{locator} {\eva ochor daiin qokShol daiim chol okary}\\
\begin{locator}[P2, 11]\end{locator} {\eva Sho ShocKho cKhy tchor chodaiin chom}\\
\begin{locator}[P2, 12]\end{locator} {\eva oSh chodair ytchy tchor kcham s}\\
\begin{locator}[P2, 13]\end{locator} {\eva Shar Shkaiin qokchy yty cThal chky}\\
\begin{locator}[P2, 14]\end{locator} {\eva dain Sheam yteam}\\


%%%%%%
% 4R %
%%%%%%
\clearpage
\subsection*{Folio 4, Recto}
\begin{locator}[P1, 1]\end{locator} {\eva kodalchy chpady Sheol ol Sheey qotey doiin chor ytoy}\\
\begin{locator}[P1, 2]\end{locator} {\eva dchor chol Shol cThol Shtchy chaiin \textcolor{orange}{*}s choraiin chom}\\
\begin{locator}[P1, 3]\end{locator} {\eva otchol chol chy chaiin qotaiin daiin Shain}\\
\begin{locator}[P1, 4]\end{locator} {\eva qotchol chy yty daiin okaiin cThy}\\

\vspace{1em}
\noindent\begin{locator}[P2, 5]\end{locator} {\eva pydaiin qotchy dy tydy}\\
\begin{locator}[P2, 6]\end{locator} {\eva chor Shytchy dy tche\textcolor{orange}{e}y}\\
\begin{locator}[P2, 7]\end{locator} {\eva qotaiin cThol daiin cThom}\\
\begin{locator}[P2, 8]\end{locator} {\eva Shor Shol Shol cThy cPholdy}\\
\begin{locator}[P2, 9]\end{locator} {\eva daiin cKhochy tchy koraiin}\\
\begin{locator}[P2, 10]\end{locator} {\eva odal Shor ShyShol cPhaiin}\\
\begin{locator}[P2, 11]\end{locator} {\eva qotchoiin Sheyr qoty}\\
\begin{locator}[P2, 12]\end{locator} {\eva soiin chaiin chaiin}\\
\begin{locator}[P2, 13]\end{locator} {\eva daiin cThey}\\


%%%%%%
% 4V %
%%%%%%
\clearpage
\subsection*{Folio 4, Verso}
\begin{locator}[P1, 1]\end{locator} {\eva pchooiin kSheo kchoy chopchy dolds dlod}\\
\begin{locator}[P1, 2]\end{locator} {\eva ol chey chy cThy Shkchor Sheo cheory choldy}\\
\begin{locator}[P1, 3]\end{locator} {\eva Sho Sho chaiin Shaiin daiin qodaiin o ar am}\\
\begin{locator}[P1, 4]\end{locator} {\eva qokShy qocThy choteol daiin cThey choaiin}\\
\begin{locator}[P1, 5]\end{locator} {\eva Shor Sheey cto otoiin Shey qotchoiin chodain}\\
\begin{locator}[P1, 6]\end{locator} {\eva ytchoy Shokchy cPhody}\\

\vspace{1em}
\noindent\begin{locator}[P2, 7]\end{locator} {\eva torchy Sheeor chor Shokchy cPhydy}\\
\begin{locator}[P2, 8]\end{locator} {\eva olaen chor cThol Sho otor cThory}\\
\begin{locator}[P2, 9]\end{locator} {\eva qooko iiincheom chcThy Shoky daiin}\\
\begin{locator}[P2, 10]\end{locator} {\eva otaiin Sheo okeody chol chokeody}\\
\begin{locator}[P2, 11]\end{locator} {\eva Sho kcheor Shody Shtaiin qotol daiin}\\
\begin{locator}[P2, 12]\end{locator} {\eva qokoy Sho okeol s keey Shar char ody}\\
\begin{locator}[P2, 13]\end{locator} {\eva Shody s cheor chokody Shodaiin qoty}\\
\begin{locator}[P2, 14]\end{locator} {\eva ochody chykey chtody}\\


%%%%%%
% 5R %
%%%%%%
\clearpage
\subsection*{Folio 5, Recto}
\begin{locator}[P1, 1]\end{locator} {\eva kShody fchoy chkoy oaiin oar olsy chody dkShy dy}\\
\begin{locator}[P1, 2]\end{locator} {\eva ochey okey qokaiin Sho cKhoy cThey chey oka*or otol}\\
\begin{locator}[P1, 3]\end{locator} {\eva qoaiin otan chy daiin oteeen cho cThy otchy qotcho dy}\\
\begin{locator}[P1, 4]\end{locator} {\eva otain Sheody chan s cheor chocThy}\\

\vspace{1em}
\noindent\begin{locator}[P2, 5]\end{locator} {\eva tShy Shody qoaiin cholols Sho qotcheo daiin Shodaiin}\\
\begin{locator}[P2, 6]\end{locator} {\eva Sho cheor chey qoeeey qoykeeey qoeor cThy ShotShy dy}\\
\begin{locator}[P2, 7]\end{locator} {\eva qotoeey keey cheo kchy Shody}\\


%%%%%%
% 5V %
%%%%%%
\clearpage
\subsection*{Folio 5, Verso}
\begin{locator}[P1, 1]\end{locator} {\eva kocheor chor ytchey pShod chols chodaiin ytoiiin daiin}\\
\begin{locator}[P1, 2]\end{locator} {\eva dchol \textcolor{orange}{S}y chol otaiin dain cThor chots ychopordg}\\
\begin{locator}[P1, 3]\end{locator} {\eva qotcho ytor daiin daiin otchor daii\textcolor{orange}{n} qo darchor do}\\
\begin{locator}[P1, 4]\end{locator} {\eva qotor Shees otol ykoiin Shol daiin cThor okchy taiin}\\
\begin{locator}[P1, 5]\end{locator} {\eva Shokeeol chor cheotol otchol daiin dal chol chotaiin}\\
\begin{locator}[P1, 6]\end{locator} {\eva otol chol dairodg}\\


%%%%%%
% 6R %
%%%%%%
\clearpage
\subsection*{Folio 6, Recto}
\begin{locator}[P1, 1]\end{locator} {\eva foar y Shol cholor cPhol chor chcKh chopchol otcham}\\
\begin{locator}[P1, 2]\end{locator} {\eva daiin chcKhy chor chor kar cThy cThor chotols}\\
\begin{locator}[P1, 3]\end{locator} {\eva poeear kShor choky os cheoee\textcolor{orange}{e}s ykeor ytaiin dar}\\
\begin{locator}[P1, 4]\end{locator} {\eva dar cho s Sheor chocThy otcham yaiir chy}\\
\begin{locator}[P1, 5]\end{locator} {\eva tar okoiin Shees ytaly cThaiin odam}\\
\begin{locator}[P1, 6]\end{locator} {\eva or al daiin cKham okom cThaiin ydaiin}\\
\begin{locator}[P1, 7]\end{locator} {\eva daiin qodaiin cho s chol okaiin s}\\
\begin{locator}[P1, 8]\end{locator} {\eva ychol cKhor pchar Sheo cKhaiin}\\
\begin{locator}[P1, 9]\end{locator} {\eva dar Sh\textcolor{orange}{e}ol skaiiodar otaiin chory}\\
\begin{locator}[P1, 10]\end{locator} {\eva tchor cTheod chy Shor odShe od}\\
\begin{locator}[P1, 11]\end{locator} {\eva ychar olchad ol chokaiin}\\
\begin{locator}[P1, 12]\end{locator} {\eva or Shol cThom chor cThy}\\
\begin{locator}[P1, 13]\end{locator} {\eva qocThol \textcolor{orange}{y}odaiin cThy}\\
\begin{locator}[P1, 14]\end{locator} {\eva ySho taiin y kaiim}\\


%%%%%%
% 6V %
%%%%%%
\clearpage
\subsection*{Folio 6, Verso}
\begin{locator}[P1, 1]\end{locator} {\eva koar y sar \textcolor{orange}{c}heekar qoar Shor chapchy s chear char otchy}\\
\begin{locator}[P1, 2]\end{locator} {\eva oees chor chcKhy qoekchar cheas odaii\textcolor{orange}{i}n kchey chor chaiin}\\
\begin{locator}[P1, 3]\end{locator} {\eva qoair cKhy chol oochocKhy chekchoy cKhy okol rychos}\\
\begin{locator}[P1, 4]\end{locator} {\eva y ShcKhy ytchoy sos y\textcolor{orange}{$\frown$}dady dchy dey okody ytody}\\
\begin{locator}[P1, 5]\end{locator} {\eva dair \textcolor{orange}{Sha} chodam dam okor oty doldom}\\

\vspace{1em}
\noindent\begin{locator}[P2, 6]\end{locator} {\eva tchody ShocThol chocThey s}\\
\begin{locator}[P2, 7]\end{locator} {\eva ychos ychol daiin cThol dol}\\
\begin{locator}[P2, 8]\end{locator} {\eva ychor chor okchey qokom}\\
\begin{locator}[P2, 9]\end{locator} {\eva oeeo dal chor cThom s}\\
\begin{locator}[P2, 10]\end{locator} {\eva qokch\textcolor{orange}{od} ychear kchdy}\\
\begin{locator}[P2, 11]\end{locator} {\eva lor char otam cThom dy}\\
\begin{locator}[P2, 12]\end{locator} {\eva ytchos Shy qokam cThy}\\
\begin{locator}[P2, 13]\end{locator} {\eva yodaiin cThy s chor oees or}\\
\begin{locator}[P2, 14]\end{locator} {\eva qokor chol cThol tchalody}\\
\begin{locator}[P2, 15]\end{locator} {\eva chocKhy s os chy sain or}\\
\begin{locator}[P2, 16]\end{locator} {\eva ochy cThar cThar cThy}\\
\begin{locator}[P2, 17]\end{locator} {\eva y chaiir cKhal cThodam dy}\\
\begin{locator}[P2, 18]\end{locator} {\eva ytcho\textcolor{orange}{$\frown$}cThol ches cThor}\\
\begin{locator}[P2, 19]\end{locator} {\eva ocholy kchos chy dor}\\
\begin{locator}[P2, 20]\end{locator} {\eva dchor choldar okol daiin}\\
\begin{locator}[P2, 21]\end{locator} {\eva ycheor chor ocTham}\\


%%%%%%
% 7R %
%%%%%%
\clearpage
\subsection*{Folio 7, Recto}
\begin{locator}[P1, 1]\end{locator} {\eva fchodaiin Shopchey qko Shey qoos Sheey ch\textcolor{orange}{a}rochy}\\
\begin{locator}[P1, 2]\end{locator} {\eva dcheey keor Shor dold dchey kchey otchy cheody}\\
\begin{locator}[P1, 3]\end{locator} {\eva oeeees cheodaiin Sheey ytcheey qotchy chald}\\
\begin{locator}[P1, 4]\end{locator} {\eva qokcho cho lochey daiin ychey kchos odaiin}\\
\begin{locator}[P1, 5]\end{locator} {\eva oaiir otaiin}\\

\vspace{1em}
\noindent\begin{locator}[P2, 6]\end{locator} {\eva kSholo\textcolor{orange}{Sh}ey qotoees chkoldy otchor choaiin}\\
\begin{locator}[P2, 7]\end{locator} {\eva dShoy cThol chol otchol dain Shody Shol chotchy}\\
\begin{locator}[P2, 8]\end{locator} {\eva okchey deeeese choty qokchy Shol keey choty dain}\\
\begin{locator}[P2, 9]\end{locator} {\eva qokechy olchoiin chol cPhey ShcKhy chochy kchod}\\
\begin{locator}[P2, 10]\end{locator} {\eva schain chor daiin chcKhy}\\


%%%%%%
% 7V %
%%%%%%
\clearpage
\subsection*{Folio 7, Verso}
\begin{locator}[P1, 1]\end{locator} {\eva polyShy Shey tchody qopchy otShol dy\textcolor{orange}{$\frown$}daiin tShodody}\\
\begin{locator}[P1, 2]\end{locator} {\eva chochy cThy daiin qoky chcPhhy daiin cThol cThy cThd}\\
\begin{locator}[P1, 3]\end{locator} {\eva qokchy dykchy chkeey kShy ky ty dor cheey ol\textcolor{orange}{$\frown$}cheol\textcolor{orange}{$\frown$}dy}\\
\begin{locator}[P1, 4]\end{locator} {\eva choteeen oeear choschy dain Sho\textcolor{orange}{$\frown$}kShy Shol deees\textcolor{orange}{$\frown$}dol}\\
\begin{locator}[P1, 5]\end{locator} {\eva dchodaiin qotchy cheey tcheey}\\

\vspace{1em}
\noindent\begin{locator}[P2, 6]\end{locator} {\eva kchor Sheod Sheodaiin Shodaiin okSho\textcolor{orange}{$\frown$}lShol dai\textcolor{orange}{r} qos}\\
\begin{locator}[P2, 7]\end{locator} {\eva okSho\textcolor{orange}{$\frown$}deeen chor\textcolor{orange}{$\frown$}cheor odaiin Shotch\textcolor{orange}{*} dol dol dor aiin}\\
\begin{locator}[P2, 8]\end{locator} {\eva qoteeeo rcho\textcolor{orange}{$\frown$}cheeody qotchey tey okchor daiin}\\
\begin{locator}[P2, 9]\end{locator} {\eva Sho keeo daiir chokchy dor deol dy dol daiin}\\


%%%%%%
% 8R %
%%%%%%
\clearpage
\subsection*{Folio 8, Recto}
\begin{locator}[P1, 1]\end{locator} {\eva pShol chor otShal chopy cPhol chody Shy cFhodar Shor}\\
\begin{locator}[P1, 2]\end{locator} {\eva tchty Sh kcheals Sho okche do dchy dain al}\\
\begin{locator}[P1, 3]\end{locator} {\eva chodar Shy ry chodaiin Shokchy chor dy}\\
\begin{locator}[P1, 4]\end{locator} {\eva qotor chor chor Sheey dchol Shesed chofchy dam}\\
\begin{locator}[P1, 5]\end{locator} {\eva okchey do r cheeey dy ky scho chky ckooaiin ch\textcolor{orange}{o} taiin}\\
\begin{locator}[P1, 6]\end{locator} {\eva toSh ckcheey koltoldy Shy cho\textcolor{orange}{e}ty cheeody sol}\\
\begin{locator}[P1, 7]\end{locator} {\eva choto kchoan choor dain}\\
\begin{locator}[T1, 8]\end{locator} {\eva \textcolor{orange}{d}cho daiin}\\

\vspace{1em}
\noindent\begin{locator}[P2, 1]\end{locator} {\eva tchoep Sho pcheey pchey ofchey dSheey Shol\textcolor{orange}{$\frown$}daiin Shor}\\
\begin{locator}[P2, 9]\end{locator} {\eva daiin cheey teeodan dy cheocThy okSheo dol dair\textcolor{orange}{g}}\\
\begin{locator}[P2, 10]\end{locator} {\eva Shol cheodaiin daiin do ytchody chot choty otariin}\\
\begin{locator}[P2, 11]\end{locator} {\eva qochodaiin Shotokody chotol}\\
\begin{locator}[T2, 12]\end{locator} {\eva okokchod\textcolor{orange}{g}}\\

\vspace{1em}
\noindent\begin{locator}[P3, 13]\end{locator} {\eva cTho cThey Shol chofydy Sho chey kShey lody cholal}\\
\begin{locator}[P3, 14]\end{locator} {\eva dchey cKhol chol chey kchs chy \textcolor{orange}{cT}odaiin dol daiiirchy cKhy}\\
\begin{locator}[P3, 15]\end{locator} {\eva ychey kchokchy chsey kchy scheaiin cThaichar cThy dar}\\
\begin{locator}[P3, 16]\end{locator} {\eva chol dchy qokar chl aiin chean c\textcolor{orange}{K}y char chaiin}\\
\begin{locator}[P3, 17]\end{locator} {\eva okar cPhaiin chaiin el daiin chor cha rchealcham}\\
\begin{locator}[P3, 18]\end{locator} {\eva sair cheain cPhol dar Shol kaiin Shol kaiin daikam}\\
\begin{locator}[P3, 19]\end{locator} {\eva or chokesey Shey okal chal}\\
\begin{locator}[T3, 20]\end{locator} {\eva schol sair}\\


%%%%%%
% 8V %
%%%%%%
\clearpage
\subsection*{Folio 8, Verso}
\begin{locator}[P1, 1]\end{locator} {\eva cThod soocTh sol Shol otol chol opcheaiin opydaiin saiin}\\
\begin{locator}[P1, 2]\end{locator} {\eva ShcThal sar chor Sheaiin Shor chykchy otaiin cty}\\
\begin{locator}[P1, 3]\end{locator} {\eva qody cheal sy chory chear Shol chaiin Shaiin dolar}\\
\begin{locator}[P1, 4]\end{locator} {\eva dShol Shol d\textcolor{orange}{a}l chean cThar Shealy daiin chary}\\
\begin{locator}[P1, 5]\end{locator} {\eva chol chol dar otchar etaiin cThol dar}\\
\begin{locator}[P1, 6]\end{locator} {\eva daiin cThan ytchy chey kaiin dain ar}\\
\begin{locator}[P1, 7]\end{locator} {\eva Sho kchol dar Shey cThar chotain ry}\\
\begin{locator}[P1, 8]\end{locator} {\eva okchol kSh chol chol chol cThaiin dain}\\
\begin{locator}[P1, 9]\end{locator} {\eva Shol orchl chokchy chol cThor chaiin}\\
\begin{locator}[P1, 10]\end{locator} {\eva scharchy oeesody kchey pchy cPharom}\\
\begin{locator}[P1, 11]\end{locator} {\eva sorain}\\

\vspace{1em}
\noindent\begin{locator}[P2, 12]\end{locator} {\eva pchar cho rol dal Shear ch\textcolor{orange}{ch}otaiin chal daiin}\\
\begin{locator}[P2, 13]\end{locator} {\eva kchor otchar oky chokain keoky otorchy satar}\\
\begin{locator}[P2, 14]\end{locator} {\eva Shor okol lokaiin Shol kol char cThey tchy cKham}\\
\begin{locator}[P2, 15]\end{locator} {\eva or chol cheen chcky chor cheain char cheeky chor ry}\\
\begin{locator}[P2, 16]\end{locator} {\eva chor chear chear oteey dchor chodey cho raiin}\\
\begin{locator}[P2, 17]\end{locator} {\eva dain chear daiin}\\


%%%%%%
% 9R %
%%%%%%
\clearpage
\subsection*{Folio 9, Recto}
\begin{locator}[P1, 1]\end{locator} {\eva tydlo choly cThor orchey s Shy odaiin sary Shor cThy}\\
\begin{locator}[P1, 2]\end{locator} {\eva oykeey chol ytaiin okchody toeoky okoiin dy or chaiin}\\
\begin{locator}[P1, 3]\end{locator} {\eva toiin cPhy qotod otaiin cThy okor chey cThod ram}\\
\begin{locator}[P1, 4]\end{locator} {\eva yShy chokcho chcThod Shor Shaiin otar dor ytol dayty}\\
\begin{locator}[P1, 5]\end{locator} {\eva daiin chor sor cThy chokoiin Shol dSholdy otchol ot dy}\\

\vspace{1em}
\noindent\begin{locator}[P2, 6]\end{locator} {\eva pShoain cThyaiin okaiir cFhodoral Shar sy Shydal chdy}\\
\begin{locator}[P2, 7]\end{locator} {\eva or chol chytchy tchol ytor qotol chyky chodar aiin}\\
\begin{locator}[P2, 8]\end{locator} {\eva qotcho qokchy cThey koraiin okain d dal s olSho cThy}\\
\begin{locator}[P2, 9]\end{locator} {\eva ocho cThy choc\textcolor{orange}{T}oy chodykchy saiin dchy daiin}\\
\begin{locator}[T1, 10]\end{locator} {\eva ytchas oraiin chk\textcolor{orange}{o}r}\\


%%%%%%
% 9V %
%%%%%%
\clearpage
\subsection*{Folio 9, Verso}
\begin{locator}[P1, 1]\end{locator} {\eva fochor oporody opy Shor daiin qopchypcho qofol Shol cFhol daiin}\\
\begin{locator}[P1, 2]\end{locator} {\eva dchor qoaiin chkaiin cThor chol chor cPhol dy oty qokaiin dy}\\
\begin{locator}[P1, 3]\end{locator} {\eva ykey chor ykaiin daiin cThy otaiin oky oeees daiin}\\
\begin{locator}[P1, 4]\end{locator} {\eva ytey tchy kaiin cThor otol oty toldy}\\

\vspace{1em}
\noindent\begin{locator}[P2, 5]\end{locator} {\eva pchor ypcheey qotor ypchy olcFholy te ar chty daiiin}\\
\begin{locator}[P2, 6]\end{locator} {\eva odol choy kSheody chody dain otchy cThod yk\textcolor{orange}{o}}\\
\begin{locator}[P2, 7]\end{locator} {\eva qochol chol ctchy daiin otal dor daim}\\
\begin{locator}[P2, 8]\end{locator} {\eva soiin daiin qokcho rokyd daly}\\
\begin{locator}[P2, 9]\end{locator} {\eva daiin chy tor chyty dary ytoldy}\\
\begin{locator}[P2, 10]\end{locator} {\eva oty kchol chol chy kyty}\\
\begin{locator}[P2, 11]\end{locator} {\eva ychor chShoty oky kaiin}\\
\begin{locator}[P2, 12]\end{locator} {\eva chkaiin cKhy chor}\\


%%%%%%%
% 10R %
%%%%%%%
\clearpage
\subsection*{Folio 10, Recto}
\begin{locator}[P1, 1]\end{locator} {\eva pchocThy Shor ocThody chorchy pchodol chopchal ypch kom}\\
\begin{locator}[P1, 2]\end{locator} {\eva dchey cThoor char chty os chair otytchol oky daiin etyd}\\
\begin{locator}[P1, 3]\end{locator} {\eva qotor otchy daiin chocThy qotchy chol or yty dy dy}\\
\begin{locator}[P1, 4]\end{locator} {\eva sor chaiin chcThy cTho cKhy or aiin chtchor doiir ody}\\
\begin{locator}[P1, 5]\end{locator} {\eva qokchy qotchol chol cThy}\\

\vspace{1em}
\noindent\begin{locator}[P2, 6]\end{locator} {\eva ych\textcolor{orange}{s}or cThy chor cThaiin qocTholy dy chy taiin Shy}\\
\begin{locator}[P2, 7]\end{locator} {\eva dchy qokchol ykchaiin yty daiin cTh dain dair am}\\
\begin{locator}[P2, 8]\end{locator} {\eva qotchor chor otol chol cholor chol daiin dar}\\
\begin{locator}[P2, 9]\end{locator} {\eva oykchor Shor chor chy kaiiin dy chodaiin}\\
\begin{locator}[P2, 10]\end{locator} {\eva oqot\textcolor{orange}{o}r otor cFhy cThor osain ytoiin}\\
\begin{locator}[P2, 11]\end{locator} {\eva rotcho Shor qoty qotor cThy d otar}\\
\begin{locator}[P2, 12]\end{locator} {\eva rodaiin daiin qotchy qotor}\\


%%%%%%%
% 10V %
%%%%%%%
\clearpage
\subsection*{Folio 10, Verso}
\begin{locator}[P1, 1]\end{locator} {\eva paiin daiin Sheo pcheey qoty daiin cThor otydy sain}\\
\begin{locator}[P1, 2]\end{locator} {\eva dain daiin cKhy chcThor choiin qot chodaiin cThy daiin}\\
\begin{locator}[P1, 3]\end{locator} {\eva dSho ytey kchol olty chol dy}\\

\vspace{1em}
\noindent\begin{locator}[P2, 4]\end{locator} {\eva qotchytor Shoiin daiin qotchey ShcThey ytor dain}\\
\begin{locator}[P2, 5]\end{locator} {\eva Sho ykeey daiin qotchy qotor chol daiin qokchyky}\\
\begin{locator}[P2, 6]\end{locator} {\eva Shoiin chor ShcThy qoty qotoiin qokol choraiin}\\
\begin{locator}[P2, 7]\end{locator} {\eva qokol chyky chol cheky daiin dain chcKhan}\\


%%%%%%%
% 11R %
%%%%%%%
\clearpage
\subsection*{Folio 11, Recto}
\begin{locator}[P1, 1]\end{locator} {\eva tShol schoal cFhy Shfydaiin cPhy Shey tchody Shoyty}\\
\begin{locator}[P1, 2]\end{locator} {\eva socThody qodor y kShy daiin ytchy ytchoky kchol daiin}\\
\begin{locator}[P1, 3]\end{locator} {\eva qoty chol cThy dor ykychy choty dain chaiin daiin ded}\\
\begin{locator}[P1, 4]\end{locator} {\eva dchol chy kchy dy daiin}\\

\vspace{1em}
\noindent\begin{locator}[P2, 5]\end{locator} {\eva tchol Shor Shor dky \textcolor{orange}{S}cPhy daiin cThy dy chodl daiin}\\
\begin{locator}[P2, 6]\end{locator} {\eva odl ds y otol chaiin ykchor dair chody cThy s daiin}\\
\begin{locator}[P2, 7]\end{locator} {\eva qotchy okchol cThy dy}\\


%%%%%%%
% 11V %
%%%%%%%
\clearpage
\subsection*{Folio 11, Verso}
\begin{locator}[P1, 1]\end{locator} {\eva poldchody ShcPhy Shordy qoty Shol cPhar dan y}\\
\begin{locator}[P1, 2]\end{locator} {\eva Shol dy chcKhy ShcThy daiin dam ykchy dain dchy}\\
\begin{locator}[P1, 3]\end{locator} {\eva otchor dy kchy tchy \textcolor{orange}{d}ar qokchd oky chol dy dy}\\
\begin{locator}[P1, 4]\end{locator} {\eva qokchor chololer chyky dchy qoky cTho tchey tan}\\
\begin{locator}[P1, 5]\end{locator} {\eva soydy qoteey qot\textcolor{orange}{$\smile$}chor dy ddy cThor Shy arg}\\
\begin{locator}[P1, 6]\end{locator} {\eva ycheor kSho dor cThey schold}\\


%%%%%%
% 12 %
%%%%%%
\clearpage
\subsection*{Folio 12}
Folio 12 is missing from the manuscript.


%%%%%%%
% 13R %
%%%%%%%
\clearpage
\subsection*{Folio 13, Recto}
\begin{locator}[P1, 1]\end{locator} {\eva torShor opchy Shol dy qopchy Shol opchor dypchy dchm}\\
\begin{locator}[P1, 2]\end{locator} {\eva dchol chol dol Shkchy ydal Shy ykchy qot\textcolor{orange}{$\smile$}ey daiin s y}\\
\begin{locator}[P1, 3]\end{locator} {\eva s y dchor Shaiin oeees ykor chor ytShy ykchy kchy dar}\\
\begin{locator}[P1, 4]\end{locator} {\eva qodchy ytchy otchor}\\

\vspace{1em}
\noindent\begin{locator}[P2, 5]\end{locator} {\eva Shorodo Shy tShy kchol dpchy qopchy otchol cFhol dy}\\
\begin{locator}[P2, 6]\end{locator} {\eva tchor dor daiin qotchol okchy okchor oiin chcKhy d}\\
\begin{locator}[P2, 7]\end{locator} {\eva dchy qoky chol dy qokhy d oldy okchor doaiin}\\
\begin{locator}[P2, 8]\end{locator} {\eva Shochy qokchy torchy kcc\textcolor{orange}{K}y s okchey daiin}\\
\begin{locator}[P2, 9]\end{locator} {\eva oldy Shey chol doiin ykoly okchal daldy}\\
\begin{locator}[P2, 10]\end{locator} {\eva sotchy kchy okorory}\\


%%%%%%%
% 13V %
%%%%%%%
\clearpage
\subsection*{Folio 13, Verso}
\begin{locator}[P1, 1]\end{locator} {\eva koair chtoiin otchy kchod otol otchy ocThos}\\
\begin{locator}[P1, 2]\end{locator} {\eva oko qokol chodal otchol cPhol choty}\\
\begin{locator}[P1, 3]\end{locator} {\eva qokchy qokod chy otchy cThody}\\
\begin{locator}[P1, 4]\end{locator} {\eva ols chey okos oaiin okShy qoc\textcolor{orange}{K}y}\\
\begin{locator}[P1, 5]\end{locator} {\eva qoky daiin}\\

\vspace{1em}
\noindent\begin{locator}[P2, 6]\end{locator} {\eva foldaiin olcPhy Shol dy oty Shor qotyd dairo d}\\
\begin{locator}[P2, 7]\end{locator} {\eva dain okal chy qokchory dchy koky daiin Sho\textcolor{orange}{n}}\\
\begin{locator}[P2, 8]\end{locator} {\eva otchy daiin y dain ykol okchy okald d ytaiin}\\
\begin{locator}[P2, 9]\end{locator} {\eva tchtod otal cThor ytal y cho t\textcolor{orange}{o}l Sho qocThy}\\
\begin{locator}[P2, 10]\end{locator} {\eva y ol chy kchey kchor dal}\\


%%%%%%%
% 14R %
%%%%%%%
\clearpage
\subsection*{Folio 14, Recto}
\begin{locator}[P1, 1]\end{locator} {\eva pchodaiin chopol Shoiin daiin dain}\\
\begin{locator}[P1, 2]\end{locator} {\eva o ykeey soiiin chok qokchy da okol}\\
\begin{locator}[P1, 3]\end{locator} {\eva ydaiin olchy kchor daiin olol}\\
\begin{locator}[P1, 4]\end{locator} {\eva ochkch\textcolor{orange}{o}r kol Shy daiin dorody}\\
\begin{locator}[P1, 5]\end{locator} {\eva qokchol dar dal\textcolor{orange}{o} qotolo}\\
\begin{locator}[P1, 6]\end{locator} {\eva ychol oir okor choor ocKhy}\\
\begin{locator}[P1, 7]\end{locator} {\eva otcho dain chcKhy}\\

\vspace{1em}
\noindent\begin{locator}[P2, 8]\end{locator} {\eva soShy fchol Shor cheor ykaiin s}\\
\begin{locator}[P2, 9]\end{locator} {\eva sody chody otchody qotchy koiin sy Shoty dy}\\
\begin{locator}[P2, 10]\end{locator} {\eva qotchor chod Shoty chody dol dy dy okchy dy}\\
\begin{locator}[P2, 11]\end{locator} {\eva dchokchy schol dy Shey dar qoty ykeey ky}\\
\begin{locator}[P2, 12]\end{locator} {\eva oeeen chey \textcolor{orange}{k}eor chey tchy ky chodalg}\\
\begin{locator}[P2, 13]\end{locator} {\eva sodaiin chy kchy kchy ykeody}\\


%%%%%%%
% 14V %
%%%%%%%
\clearpage
\subsection*{Folio 14, Verso}
\begin{locator}[P1, 1]\end{locator} {\eva pdychoiin yfodain otyShy dy ypchor daiin kol ydain}\\
\begin{locator}[P1, 2]\end{locator} {\eva okchor dchy tShy oky chy cThy otchy ty chol daiin}\\
\begin{locator}[P1, 3]\end{locator} {\eva ychy dy daiin chcThy ykykaiin dytchy y\textcolor{orange}{K}chy ky dy}\\
\begin{locator}[P1, 4]\end{locator} {\eva yty\textcolor{orange}{$\smile$}chy kSho ykShy ShokShor yty darody dyotyds}\\
\begin{locator}[P1, 5]\end{locator} {\eva okShy daiin okchor chky qotchy daiin cThor oty}\\
\begin{locator}[P1, 6]\end{locator} {\eva qoty choky cThy chokchy dydydy chcKhy dchyd n}\\
\begin{locator}[P1, 7]\end{locator} {\eva oykShy choty dydy odyd otchy okchy dShy dardy}\\
\begin{locator}[P1, 8]\end{locator} {\eva chokShor daiin okShody daiin dol dair dam}\\
\begin{locator}[P1, 9]\end{locator} {\eva dykchy cTholdg dchcKhy}\\


%%%%%%%
% 15R %
%%%%%%%
\clearpage
\subsection*{Folio 15, Recto}
\begin{locator}[P1, 1]\end{locator} {\eva tShor Shey tchaly Shy chtols Shey daiin}\\
\begin{locator}[P1, 2]\end{locator} {\eva otchor qokchor oly okor Shy koly}\\
\begin{locator}[P1, 3]\end{locator} {\eva qokaiin qotchy tydy daiin chol cThy}\\
\begin{locator}[P1, 4]\end{locator} {\eva scheaiin chodaiin chl sol cKhaiin sal}\\
\begin{locator}[P1, 5]\end{locator} {\eva qotchy r Shor cThy daiin cThy dy}\\
\begin{locator}[P1, 6]\end{locator} {\eva dchy kokaiin chdy saiin okear}\\
\begin{locator}[P1, 7]\end{locator} {\eva daiin Shkaiin cThy Sho keocThy}\\
\begin{locator}[P1, 8]\end{locator} {\eva ShocThy tol kaiin s dain cTholy}\\
\begin{locator}[P1, 9]\end{locator} {\eva ocThain qokaiin chos odaiin cThl s y}\\
\begin{locator}[P1, 10]\end{locator} {\eva ychain chcKhhy okShy saiiin dolchds}\\
\begin{locator}[P1, 11]\end{locator} {\eva okaiin otaiin chl sy chor choross}\\
\begin{locator}[P1, 12]\end{locator} {\eva qotor Shor tcheor chy cThaiin Shan}\\
\begin{locator}[P1, 13]\end{locator} {\eva ykShol dor Sheey cThy dain sky Shor Shoty}\\
\begin{locator}[P1, 14]\end{locator} {\eva otcho kchy chol daiin cThar ytol dor dom}\\
\begin{locator}[P1, 15]\end{locator} {\eva qotchor chaiin chy kol\textcolor{orange}{$\smile$}daky}\\


%%%%%%%
% 15V %
%%%%%%%
\clearpage
\subsection*{Folio 15, Verso}
\begin{locator}[P1, 1]\end{locator} {\eva poror orShy choiin dtchan opchor dy}\\
\begin{locator}[P1, 2]\end{locator} {\eva *chor or oro r aiin cThy \textcolor{orange}{t}ain dar}\\
\begin{locator}[P1, 3]\end{locator} {\eva cThor daiin qokor okeor okaiin}\\
\begin{locator}[P1, 4]\end{locator} {\eva doiin choky Shol qoky qotchod}\\
\begin{locator}[P1, 5]\end{locator} {\eva otchor chor chor ytchor cThy s}\\
\begin{locator}[P1, 6]\end{locator} {\eva qotchey choty kaiin otchy r aiin}\\
\begin{locator}[P1, 7]\end{locator} {\eva coy choiin Sho s chy s chy tor ols}\\
\begin{locator}[P1, 8]\end{locator} {\eva ytchor chor ol oiin oty Shol daiin}\\
\begin{locator}[P1, 9]\end{locator} {\eva otcholocThol chol chol chody kan}\\
\begin{locator}[P1, 10]\end{locator} {\eva sor chor cThoiin cThy qokaiin}\\
\begin{locator}[P1, 11]\end{locator} {\eva soloiin cheor chol daiin cThy}\\
\begin{locator}[P1, 12]\end{locator} {\eva daiin cThor chol chor}\\


%%%%%%%
% 16R %
%%%%%%%
\clearpage
\subsection*{Folio 16, Recto}
\begin{locator}[P1, 1]\end{locator} {\eva pocheody qopchey sykaiin opchy dor ychy daiin dy chor orom}\\
\begin{locator}[P1, 2]\end{locator} {\eva ychykchy otly kol Shor ody otody qoy oeesordy}\\
\begin{locator}[P1, 3]\end{locator} {\eva ydor Sheal okchy qoy koiin choky ykair}\\
\begin{locator}[T1, 4]\end{locator} {\eva dainod ychealod}\\

\vspace{1em}
\noindent\begin{locator}[P2, 5]\end{locator} {\eva tchor chor chs ykch ShocThy opchy ty ky}\\
\begin{locator}[P2, 6]\end{locator} {\eva oShaiin dyky oeees deeeod aiin dtoaiin}\\
\begin{locator}[P2, 7]\end{locator} {\eva daiin dalchy dyky schy saiin doal qoky}\\
\begin{locator}[P2, 8]\end{locator} {\eva Shotchy ydain yky Shody otol daiin}\\
\begin{locator}[P2, 9]\end{locator} {\eva saiin ytaiin}\\

\vspace{1em}
\noindent\begin{locator}[P3, 10]\end{locator} {\eva toror dalydal opchy fchol ypchoc\textcolor{orange}{F}y okal}\\
\begin{locator}[P3, 11]\end{locator} {\eva sokchy qokol choty\textcolor{orange}{$\smile$}okchy cThy chy kchy}\\
\begin{locator}[P3, 12]\end{locator} {\eva dychokchy ShcThy ShtShy Sho tchokyd}\\
\begin{locator}[P3, 13]\end{locator} {\eva qokchor dl dy Shey}\\



%%%%%%%
% 16V %
%%%%%%%
\clearpage
\subsection*{Folio 16, Verso}
\begin{locator}[P1, 1]\end{locator} {\eva pchraiin otchor chpchol chpchey s pchocty}\\
\begin{locator}[P1, 2]\end{locator} {\eva ytchor y ky chokchy qokchocThor Shory}\\
\begin{locator}[P1, 3]\end{locator} {\eva ykchy dy choy qoty chy kchy koShet}\\
\begin{locator}[P1, 4]\end{locator} {\eva dchol chcThody cPhod chotol dal}\\
\begin{locator}[P1, 5]\end{locator} {\eva ytchy chyty chor chol ytchy dan}\\
\begin{locator}[P1, 6]\end{locator} {\eva sor chk\textcolor{orange}{o}r oty chk\textcolor{orange}{o}r chol dairin}\\

\vspace{1em}
\noindent\begin{locator}[P2, 7]\end{locator} {\eva pchocThy chypchy qotchy chcFhhy sy}\\
\begin{locator}[P2, 8]\end{locator} {\eva daiin chol y daiin chcThy qotchar chor Sholo}\\
\begin{locator}[P2, 9]\end{locator} {\eva dShy okaiin okaiin chol chor cThor ty chody}\\
\begin{locator}[P2, 10]\end{locator} {\eva qokchy chydy ykchy chcKhy otain cThor cThy}\\
\begin{locator}[P2, 11]\end{locator} {\eva okytaiin chkchy saiin}\\
\begin{locator}[P2, 12]\end{locator} {\eva daiin yky otor chody}\\
\begin{locator}[P2, 13]\end{locator} {\eva sokar oaorar}\\


%%%%%%%
% 17R %
%%%%%%%
\clearpage
\subsection*{Folio 17, Recto}


%%%%%%%%%%%%%%
% REFERENCES %
%%%%%%%%%%%%%%
\clearpage
\printbibliography

\end{document}